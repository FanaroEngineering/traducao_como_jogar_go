\chapter{O Fim do Jogo}

\begin{itemize}
    \item[\textbf{Regra 8}] Duas passagens de turno consecutivas finalizam a partida. (Ou um dos jogadores pode desistir.)
    \item[\textbf{Regra 9}] No final da partida, toda pedra que não conseguir se salvar é removida do tabuleiro como prisioneira do adversário.
    \item[\textbf{Regra 10}] A pontuação de um jogador é o número de intersecções vazias que ele cercou menos o número de prisioneiros que ele perdeu para o adversário. A pontuação mais alta vence.
\end{itemize}

% TODO: Adicionar um comentário de que essas são as regras japonesas... Há outras...

A partida no \emph{Dia. 35} está quase finalizada. Ambos Preto e Branco asseguraram seus respectivos territórios. Durante essa partida, Branco capturou diversos prisioneiros --- indicados pelas sete pedras pretas acima dos diagramas ---, e Preto capturou três pedras brancas.

Preto 1 no \emph{Dia. 36} não conquista nenhum ponto, mas ameaça a captura de uma pedra branca e o início de um ko, então Branco conecta com 2. Preto 3 e Branco 4 não ganham pontos também. Há pontos neutros que são jogados ao final da partida, somente para mais claramente delimitar os territórios. Não há mais pontos a serem disputados, então Preto 5 e Branco 6 são passagens de turno. De acordo com a \emph{Regra 8}, isso finaliza a partida.

A pedra branca marcada no \emph{Dia. 36} está morta: ela não consegue evitar de ser capturada. Portanto, de acordo com a \emph{Regra 9}, ela é removida do tabuleiro como prisioneira. O resultado dessa operação é visualizado no \emph{Dia. 37}. Note que Preto possui quatro prisioneiros brancos.

Vamos contar a pontuação.

Preto cercou os pontos vazios marcados com \textbf{X} no \emph{Dia. 38}, e Branco cercou os pontos vazios marcados com \textbf{X} no \emph{Dia. 39}. Esses pontos vazios cercados são denominados de \emph{território}.

Para subtrair prisioneiros do território, os prisioneiros são colocados no tabuleiro dentro dos territórios pretos e brancos, como mostrado pelas pedras marcadas no \emph{Dia. 40}. Conte os territórios para cada lado e verá que Branco possui 6 pontos enquanto que Preto possui 5 pontos. Branco vence por um ponto.

Essas são as dez regras necessárias para se jogar Go. Há algumas poucas posições especiais e kos que são determinados por adjudicação. Essas posições são conhecidas como Precedentes da Nihon Kiin, mas elas ocorrem muito raramente em partidas.

As regras aqui apresentadas são as japonesas. Há outras também, a mais importante dentre essas outras sendo as regras chinesas. A maioria dos jogadores no ocidente utilizam as regras japonesas. Porém, se você for jogar Go na China, você terá de aprender sobre as regras chinesas. A principal diferença entre elas é o procedimento de contagem ao final da partida, mas, para basicamente todos os fins práticos, ambas são equivalentes. Isto é, elas não mudam a natureza do jogo. Para uma exposição dos diversos conjuntos de regras, assim como os Precedentes da Nihon Kiin, veja o livro \emph{The Go Player's Almanac}.