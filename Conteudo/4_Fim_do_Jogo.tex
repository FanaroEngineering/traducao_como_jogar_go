\chapter{O Fim do Jogo}

\begin{itemize}
    \item[\textbf{Regra 8}] Duas passagens de turno consecutivas finalizam a partida. (Ou um dos jogadores pode desistir, e é possível desistir a qualquer momento, mesmo após um passe adversário.)
    \item[\textbf{Regra 9}] Ao final da partida, toda pedra que não conseguir se salvar é removida do tabuleiro como prisioneira do adversário.
    \item[\textbf{Regra 10}] A pontuação de um jogador é o número de intersecções vazias que ele cercou \emph{menos} o número de prisioneiros que ele perdeu para o adversário. A pontuação mais alta vence --- note que a diferença de pontos é irrelevante para o status final da partida.
\end{itemize}

\pagebreak

A partida no \emph{Dia.\@~35} está quase finalizada. Ambos Preto e Branco asseguraram seus respectivos territórios. Durante essa partida, Branco capturou diversos prisioneiros, mais precisamente, sete; e Preto capturou três pedras brancas.

\begin{figure}[h!]
    \centering
    \begin{subfigure}[t]{.3\textwidth}
        \centering
        \captionsetup{justification=raggedright,singlelinecheck=false,margin={.05in,.05in}}
        \includegraphics[width=1\textwidth]{4 - Dia 35}
        \caption*{\emph{Dia.\@~35. A partida está quase finalizada}}
    \end{subfigure}
    \hfill
    \begin{subfigure}[t]{.3\textwidth}
        \centering
        \captionsetup{justification=raggedright,singlelinecheck=false,margin={.05in,.05in}}
        \includegraphics[width=1\textwidth]{4 - Dia 36}
        \caption*{\emph{Dia.\@~36. Preenchendo os pontos neutros}}
    \end{subfigure}
    \hfill
    \begin{subfigure}[t]{.3\textwidth}
        \centering
        \captionsetup{justification=raggedright,singlelinecheck=false,margin={.05in,.05in}}
        \includegraphics[width=1\textwidth]{4 - Dia 37}
        \caption*{\emph{Dia.\@~37. Removendo a pedra morta}}
    \end{subfigure}
    \vspace*{.5cm}
    \captionsetup{justification=centering}
    \caption*{\emph{Como capturas, Preto possui três pedras; e Branco, sete.}}
\end{figure}
 
Preto 1 no \emph{Dia.\@~36} não conquista nenhum ponto, mas ameaça a captura de uma pedra branca e o início de um ko, então Branco conecta com 2. Preto 3 e Branco 4 não ganham pontos também. Há pontos neutros que são jogados ao final da partida, somente para mais claramente delimitar os territórios. Não há mais pontos a serem disputados, então Preto 5 e Branco 6 são passagens de turno. De acordo com a \emph{Regra 8}, isso finaliza a partida.

A pedra branca marcada no \emph{Dia.\@~36} está morta: ela não consegue evitar de ser capturada. Portanto, de acordo com a \emph{Regra 9}, ela é removida do tabuleiro como prisioneira. O resultado dessa operação é visualizado no \emph{Dia.\@~37}. Note que Preto possui quatro prisioneiros brancos.

Contemos a pontuação.

\pagebreak

\begin{figure}[h!]
    \centering
    \begin{subfigure}[t]{.3\textwidth}
        \centering
        \captionsetup{justification=raggedright,singlelinecheck=false,margin={.05in,.05in}}
        \includegraphics[width=1\textwidth]{4 - Dia 38}
        \caption*{\emph{Dia.\@~38. O território preto}}
    \end{subfigure}
    \hfill
    \begin{subfigure}[t]{.3\textwidth}
        \centering
        \captionsetup{justification=raggedright,singlelinecheck=false,margin={.05in,.05in}}
        \includegraphics[width=1\textwidth]{4 - Dia 39}
        \caption*{\emph{Dia.\@~39. O território branco}}
    \end{subfigure}
    \hfill
    \begin{subfigure}[t]{.3\textwidth}
        \centering
        \captionsetup{justification=raggedright,singlelinecheck=false,margin={.05in,.05in}}
        \includegraphics[width=1\textwidth]{4 - Dia 40}
        \caption*{\emph{Dia.\@~40. O território é preenchido com prisioneiros}}
    \end{subfigure}
    \vspace*{.5cm}
    \captionsetup{justification=centering}
    \caption*{\emph{Como capturas, Preto possui três pedras; e Branco, sete.}}
\end{figure}

Preto cercou os pontos vazios marcados com $\times$ no \emph{Dia.\@~38}, e Branco cercou os pontos vazios marcados com $\times$ no \emph{Dia.\@~39}. Esses pontos vazios cercados são denominados de \emph{território}.

Para subtrair prisioneiros do território, os prisioneiros são colocados no tabuleiro dentro dos territórios pretos e brancos, como mostrado pelas pedras marcadas no \emph{Dia.\@~40}. Conte os territórios para cada lado e verá que Branco possui 6 pontos enquanto que Preto possui 5 pontos. Branco vence por um ponto.

\bigskip

Essas são as dez regras necessárias para se jogar Go. Há, porém, algumas poucas posições especiais e kos que são determinados por adjudicação. Essas posições são conhecidas como \emph{Precedentes da Nihon Kiin}, mas elas ocorrem muito raramente em partidas.

As regras aqui apresentadas são as japonesas. Há outras também, a mais importante dentre essas outras sendo as regras chinesas. A maioria dos jogadores no ocidente utiliza as regras japonesas. Porém, se você for jogar Go na China, terá de aprender, claro sobre as regras chinesas. A principal diferença é o procedimento de contagem ao final da partida, mas, para basicamente quase todos os fins práticos, ambas são equivalentes. Isto é, elas não mudam a natureza do jogo. Para uma exposição dos diversos conjuntos de regras, assim como os Precedentes da Nihon Kiin, veja o livro \emph{The Go Player's Almanac}~\cite{bozulich_almanac}.