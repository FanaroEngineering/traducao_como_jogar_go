\chapter{O Equipamento}

Go é geralmente jogado em um tabuleiro com 19 linhas verticais e 19 horizontais, constituindo uma grade de 361 intersecções, como se pode ver pelo Diagrama 1. O tabuleiro possui nove pontos grifados com pequenos círculos. Esses círculos são chamados de pontos-estrela.

As peças ou pedras são tipicamente duplo-convexas, para facilitar o manuseio quando de muitas pedras em uma mesma área. Um conjunto de Go padrão possui 181 pedras pretas e 180 pedras brancas.

\begin{center}
    \shortstack{\showgoban\\\emph{Dia. 1}}
\end{center}

As pedras são colocadas em containeres chamados de potes. Há dois potes em um conjunto. Um pote é para as pedras pretas, e o outro, para as brancas.

A foto na capa mostra um exemplo de tabuleiro de Go com pernas em um ambiente tradicional japonês. O equipamento de Go pode ser obtido em uma grande gama de qualidade, desde conjuntos muito baratos àqueles custando dezenas de milhares de reais. É possível encontrar uma lista de equipamentos visitando o site da Kiseido: \href{https://www.kiseido.com}{\path{kiseido.com}}~\cite{kiseido}. Na verdade, você nem sequer precisa de um conjunto de Go para estudar ou jogar. É possível simplesmente baixar gratuitamente um programa chamado \emph{Go Write 2} que permite jogar e analisar posições (\href{https://www.gowrite.net/GOWrite2_download.html}{\path{gowrite.net/GOWrite2_download.html}})~\cite{gowrite}. Você também pode jogar partidas online sem custos com oponentes do mundo inteiro, através do servidor KGS (\href{https://www.gokgs.com}{\path{gokgs.com}})~\cite{kgs}. A força dos jogadores lá abrange desde iniciantes até profissionais.

O jogo de Go também pode ser jogado em pequenos tabuleiros, sem nenhuma mudança nas regras, e iniciantes são encorajados a jogar suas primeiras partidas em tabuleiros 9\(\times\)9. Já que uma partida 9\(\times\)9 pode ser finalizada em aproximadamente 10 minutos, esta é uma boa maneira de iniciantes se familiarizarem com as regras e as táticas básicas. Você talvez queira, a partir daí, então jogar algumas partidas em um tabuleiro $13\times13$ antes de progredir para o padrão oficial $19\times19$.