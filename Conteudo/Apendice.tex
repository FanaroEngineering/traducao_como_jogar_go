\chapter{Catálogo Kiseido em Português}\label{ap:pt}

\emph{Note, porém, que, basicamente, nenhum dos livros abaixo foram traduzidos para o português. O catálogo abaixo é apenas uma tradução somente dos títulos em inglês.}

\bigskip

\chapter{Catálogo Kiseido em Inglês}\label{ap:en}

\emph{Books marked with an asterisk are also available on iOS via the Go Books~\cite{gobooks} app by SmartGo~\cite{smartgo}.}

\section{Introductory and General}

\begin{longtable}{l|p{50mm}|p{50mm}} 
    \hline
    \textbf{Code} & \textbf{Title} & \textbf{Author(s)} \\
    \hline \hline
    K50* & Go --- A Complete Introduction to the Game & Cho Chikun \\
    \hline
    K40 & The Go Player's Almanac 2001 & Richard Bozulich (editor) \\
    \hline
    K30 & Japanese Prints and the World of Go & William Pinckard and Kitagawa Akiko \\
    \hline
\end{longtable}


\section{Problem Books for Beginners}

\emph{Graded Go Problems for Beginners}, by Kano Yoshinori:

\begin{longtable}{c|c|l} 
    \hline
    \textbf{Code} & \textbf{Volume} & \textbf{Title} \\
    \hline \hline
    K46* & 1 & Introductory Problems (35-25 kyu) \\
    \hline
    K47* & 2 & Elementary Problems (25-20 kyu) \\
    \hline
    K48* & 3 & Intermediate Problems (20-15 kyu) \\
    \hline
    K49* & 4 & Advanced Problems (15-1 kyy) \\
    \hline
\end{longtable}


\section{Elementary Go Series}

For more than 30 years, the \emph{Elementary Go Series} has been the standard texts for Go players who want to get a firm grasp of the fundamentals of Go. Not only is the theory of Go elaborated on, the reader is also given problems to show how these theoretical concepts are applied in actual games.

\begin{longtable}{c|c|p{45mm}|p{45mm}} 
    \hline
    \textbf{Code} & \textbf{Volume} & \textbf{Title} &\textbf{Author(s)} \\
    \hline \hline
    K10* & 1 & In the Beginning --- The Opening in the Game of Go & Ishigure Ikuro \\
    \hline
    K11* & 2 & 38 Basic Josekis & Kosugi Kiyoshi and James Davies \\
    \hline
    K12* & 3 & Tesuji & James Davies \\
    \hline
    K13* & 4 & Life and Death & James Davies \\
    \hline
    K14* & 5 & Attack and Defense & Ishida Akira and James Davies \\
    \hline
    K15* & 6 & The Endgame & Ogawa Tomoko and James Davies \\
    \hline
    K16* & 7 & Handicap Go & Nagahara Yoshiaki and Richard Bozulich \\
    \hline
\end{longtable}

\section{Get Strong at Go Series}

A series of problem books covering every phase of the game, from the opening to the endgame. Each book contains 170 or more problems ranging in difficulty from elementary to advanced. Thus, they can be used by players ranging in strength from 20 kyu to dan-level. By studying Go in this problem format, you will not only learn basic principles as to why moves are made but also train yourself in thinking through and analyzing positions. You will encounter a great many of the same or similar patterns that will arise in your own games. We guarantee that diligent study of this entire series will lay the foundation for becoming a truly strong player.

\bigskip

\emph{All the books in this series are authored by Richard Bozulich.}

\begin{longtable}{l|c|l} 
    \hline
    \textbf{Code} & \textbf{Volume} & \textbf{Title} \\
    \hline \hline
    K51 & 1 & Get Strong at the Opening \\
    \hline
    K52 & 2 & Get Strong at Joseki I \\
    \hline
    K53 & 3 & Get Strong at Joseki II \\
    \hline
    K54 & 4 & Get Strong at Joseki III \\
    \hline
    K55 & 5 & Get Strong at Invading \\
    \hline
    K56 & 6 & Get Strong at Tesuji \\
    \hline
    K57* & 7 & Get Strong at the Endgame \\
    \hline
    K58* & 8 & Get Strong at Life and Death \\
    \hline
    K59* & 9 & Get Strong at Handicap Go \\
    \hline
    K60 & 10 & Get Strong at Attacking \\
    \hline
\end{longtable}

\section{Mastering the Basics}

A series of books, especially written for high-kyu players, for mastering the basic techniques of Go. Each book in this series consists of hundreds of problems designed to hammer home the fundamental concepts of Go theory and technique. A thorough and patient study of this series is the fastest way to advance through the kyu ranks.

\begin{longtable}{l|c|p{55mm}|p{35mm}} 
    \hline
    \textbf{Code} & \textbf{Volume} & \textbf{Title} & \textbf{Author(s)} \\
    \hline \hline
    K71* & 1 & 501 Opening Problems & Richard Bozulich and Rob van Zeijst \\
    \hline
    K72 & 2 & 1001 Life-and-Death Problems & Richard Bozulich \\
    \hline
    K73* & 3 & Making Good Shape & Rob van Zeijst and Richard Bozulich \\
    \hline
    K74* & 4 & 501 Tesuji Problems & Richard Bozulich \\
    \hline
    K75* & 5 & The Basics of Go Strategy & Richard Bozulich \\
    \hline
    K76* & 6 & All About Ko & Rob van Zeijst and Richard Bozulich \\
    \hline
    K77* & 7 & Attacking and Defending Moyos & Rob van Zeijst and Richard Bozulich \\
    \hline
    K78* & 8 & Fight Like a Pro --- The Secrets of Kiai & Rob van Zeijst and Richard Bozulich \\
    \hline
    K79* & 9 & An Encyclopedia of Go Principles & Richard Bozulich \\
    \hline
    K80 & 10 & Close Encounters with the Middle Game & Michiel Eijkhout \\
    \hline
\end{longtable}

\section{Elementary Books}

\begin{longtable}{l|p{55mm}|p{55mm}} 
    \hline
    \textbf{Code} & \textbf{Title} & \textbf{Author(s)} \\
    \hline \hline
    K02* & Basic Techniques of Go & Nagahara Yoshiaki and Haruyama Isamu \\
    \hline
    K28 & Lessons in the Fundamentals of Go & Kageyama Toshiro \\
    \hline
    K36 & Opening Theory Made Easy & Otake Hideo \\
    \hline
    K82* & Handicap Go Strategy and the Sanrensei Opening & Rob van Zeijst and Richard Bozulich \\
    \hline
    K83 & The Basic Principles of the Opening and Middle Game & Rob van Zeijst and Richard Bozulich \\
    \hline
    K84* & The Basics of Life and Death & Rob van Zeijst and Richard Bozulich \\
    \hline
    K85 & A Survey of Basic Tesujis & Richard Bozulich \\
    \hline
\end{longtable}

\section{Advanced Books}

