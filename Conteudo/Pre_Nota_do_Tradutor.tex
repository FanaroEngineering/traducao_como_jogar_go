\chapter{Nota do Tradutor}

Primeiramente, agradeço muito ao Richard Bozulich e ao James Davies por não somente terem escrito este livro mas, também, terem me liberado a traduzi-lo. Espero que seja o primeiro de muitos livros da excelente editora Kiseido~\cite{kiseido} traduzidos ao português.

\bigskip
\bigskip

A tradução deste livro foi feita com base em meu conhecimento do jogo. Sou um jogador amador de força aproximada mínima de 1 dan\footnote{Seria supostamente equivalente ao de faixa preta em artes marciais. Mas essa é uma simplificação talvez grosseira e muito discutível, uma comparação à la banana versus maçã.}, um nível visto como de maestria do jogo, apesar de que, claro, sempre há um peixe maior. Desde que comecei a jogar, no final de 2012, já participei de torneios tanto no Brasil quanto na Europa, Coreia e China, além de muitos torneios online também. Toda essa experiência me ajuda a discernir o que é bom do que é ruim, o que é útil do que é perda de tempo.

É por isso que corri atrás de conteúdo da editora Kiseido, que vem trazendo ao Ocidente, em inglês, conhecimento de Go já há décadas. No caso deste livro, temos uma introdução sólida e sistemática ao jogo, com muita concisão. Nele, o leitor encontrará, além das regras, conceitos e técnicas utilizados por desde iniciantes até grandes mestres de Go; e táticas e estratégias que residem no coração jogo.

Porém, não tenha pressa. Go demanda muita prática, e esta é muito divertida e, principalmente, recompensadora, por encontrar bastantes analogias filosóficas e psicológicas na vida real. Além disso, internalizar os conceitos deste livro é algo que não só demanda tempo, mas que também é feito em ciclos: mesmo os grandes mestres (re)aprendem com os fundamentos todos os dias.

Os benefícios do Go são muitos em outras áreas da vida também. Go pode ser utilizado para se aprender e se exercitar conceitos de matemática, estatística, filosofia, história, ciência da computação e muitos outros tópicos. E, pessoalmente, eu o vejo como o melhor simulador psicológico acelerado que há!\footnote{Para mais informações sobre Go e motivação, sugiro este artigo: \href{https://fanaro.io/articles/why_play_go/why_play_go.html}{\path{fanaro.io/articles/why_play_go/why_play_go.html}}~\cite{why_play_go}, apesar de, infelizmentente, só estar disponível em inglês.}

\bigskip
\bigskip

Após o término deste livro, infelizmente, não há muito mais conteúdo em português\footnote{Isso é válido somente para quando da publicação deste livro. Talvez, anos depois, já haja mais material disponível.}, pelo menos não em livros ou outras vias impressas mais tradicionais. Tentarei continuar criando mais traduções de livros de Go, mas não sei a que velocidade ou até que ponto. No entanto, já há conteúdo online disponível para variados níveis, em português, espalhado pela internet. E também vale ressaltar que mesmo livros e vídeos em línguas estrangeiras muitas vezes são compreensíveis, uma vez que diagramas de Go são uma linguagem à parte e universal.

Algumas dessas outras fontes com mais conteúdo são --- há já muitos canais de YouTube e de Twitch em diversas línguas também ---:

\begin{longtable}{p{45mm}|p{55mm}} 
 \hline
 \textbf{Nome ou URL} & \textbf{Descrição} \\
 \hline \hline
 \href{https://online-go.com}{\path{online-go.com}}~\cite{ogs} & o servidor OGS é essencialmente uma versão mais moderna do servidor KGS, citado no livro. A interface é mais bem feita e há muitos outros recursos, incluindo tutoriais e josekipédia. Este é o servidor que eu mais recomendo para iniciantes \\
 \hline
 \href{https://facebook.com/groups/gobrasil}{\path{facebook.com/groups/gobrasil}}~\cite{facebook_go_brasil} & grupo \emph{Go Brasil} no Facebook, com mais de 1200 integrantes. Também há um grupo respectivo de Whatsapp bastante ativo, mas será preciso demandar pelo grupo de Facebook como acessá-lo \\
 \hline
 \href{https://nihonkiin.com.br}{\path{nihonkiin.com.br}}~\cite{brasil_nihon_kiin} & site da Brasil Nihon Kiin, associação nipo-brasileira de Go, que, também, existe fisicamente em São Paulo-SP \\
 \hline
 \href{http://www.conscienciadoxadrez.com.br/}{\path{http://www.conscienciadoxadrez.com.br}} & Uma das lojas que mais vende equipamentos de Go, apesar de que a Brasil Nihon Kiin também os vende. Inevitavelmente, porém, será necessário buscar lojas estrangeiras para se obter os melhores equipamentos \\
 \hline
 \href{https://shin.gokgs.com/}{\path{shin.gokgs.com}}~\cite{shinkgs} & uma nova fachada para o servidor KGS, totalmente integrada a navegadores \\
 \hline
 \emph{Go --- Coleção Jogos de Tabuleiro}~\cite{go_sao_paulo} & um livro similar a este, criado pela Secretaria Municipal de São Paulo em conjunto com a Brasil Nihon Kiin, também de São Paulo \\
 \hline
 \href{https://eidogo.com}{\path{http://eidogo.com}}~\cite{eidogo} & Provavelmente a melhor referência para se explorar josekis, recomendo-o mais do que a própria Josekipedia~\cite{josekipedia} e a página de josekis do OGS \\
 \hline
 \href{https://fanaro.io}{\path{fanaro.io}}~\cite{fanaroio} & meu site, com conteúdo não só de Go mas também de software, psicologia, filosofia, etc. \\
 \hline
 \href{https://youtube.com/c/PhilippeFanaro}{\path{youtube.com/c/PhilippeFanaro}}~\cite{fanaro_youtube} & meu canal de YouTube, focado majoritariamente em Go \\
 \hline
 \href{https://twitch.tv/fanaro009}{\path{twitch.tv/fanaro009}}~\cite{fanaro_twitch} & meu canal de Twitch, onde jogo e ensino ao vivo, ocasionalmente \\
 \hline
\end{longtable}

\pagebreak

Por fim, gostaria de comentar que muitos dos termos técnicos de Go traduzidos por mim ao longo do livro encontram, talvez, suas primeiras instâncias em português. Ou seja, não os leve tanto a ferro e fogo, inclusive pois a maioria dos jogadores veteranos na verdade utiliza as respectivas versões japonesas. Tanto o autor original quanto eu, o tradutor, optamos por minimizar a carga de vocabulário para quem está começando.

\bigskip
\bigskip

Para mais informações\footnote{Eu também ofereço aulas particulares no momento, e posso oferecer direções para quem estiver se sentindo perdido na jornada de aprendizado no Go.}, ou caso queira fornecer feedback, melhorias ou comentários, envie um email para \emph{\href{mailto:philippefanaro@gmail.com}{philippefanaro@gmail.com}} \cite{fanaro_email}. Este livro/projeto também é código-aberto, ou seja, é possível examinar em detalhes como ele foi feito, através da ferramenta \LaTeX~\cite{latex}, pelo link \href{https://github.com/FanaroEngineering/traducao_como_jogar_go}{\path{github.com/FanaroEngineering/traducao_como_jogar_go}}~\cite{repo_github}.

E, como adendo final, a partida exposta na capa deste livro advém de um dos livros de Go mais antigos ainda existentes, e foi disputada entre um príncipe-embaixador japonês e o jogador mais forte da China, ao redor do ano de 1100 d.C. Porém, propositalmente, o diagrama da capa está incompleto, faltando somente o movimento final, uma das jogadas mais brilhantes de todos os tempos. Você consegue adivinhá-la? É um desafio até mesmo para grandes mestres, então não se desaponte se não conseguir solucionar este problema depois de ler este livro. No entanto, você certamente será capaz de compreender o valor desta obra-prima quase milenar depois de ler este livro! Caso queira saber mais, é só me mandar uma mensagem!\footnote{Eu cheguei a escrever um artigo sobre esta jogada no meu site (\href{https://fanaro.io/articles/tesuji_mor/tesuji_mor.html}{\path{fanaro.io/articles/tesuji_mor/tesuji_mor.html}}~\cite{tesuji_mor}).}.

\bigskip
\bigskip

\emph{Sua jornada só está começando, e um guia tão completo quanto este será de grande valia na sua aventura pelo jogo e respectiva filosofia.}

\bigskip
\bigskip

Philippe Fanaro

31 de Novembro de 2021