\chapter[Glossário de Termos de Go]{Glossário de Termos de Go Utilizados Neste Livro}

\begin{longtable}{l|p{85mm}} 
 \hline
 \textbf{Termo} & \textbf{Significado} \\
 \hline \hline
 \textbf{atari} & quando todas as intersecções exceto uma diretamente adjacente à pedra ou grupo de pedras são ocupadas pelo oponente, a pedra ou grupo é dita estar sob ``atari"\footnote{Sim, este termo de Go inspirou o nome do console icônico de videogames, Atari.}, e o oponente pode capturá-la no próximo movimento \\ 
 \hline
 \textbf{dan} & uma classificação de força de jogadores indo de shodan (1 dan) até 9 dan \\
 \hline
 \textbf{gote} & um movimento que o oponente pode ignorar. Um movimento gote é geralmente defensivo, então não representa nenhuma ameaça ao adversário \\
 \hline
 \textbf{joseki} & durante a abertura, conflitos locais frequentemente surgem, começando tipicamente nos cantos e se desenvolvendo para os lados. Esses conflitos locais se chamam josekis \\
 \hline
 \textbf{komi} & compensação de pontos adicionada a um dos lados no final da partida. Tipicamente, 6.5 pontos ao território do Branco em uma partida igualitária, para compensar a desvantagem de ser a segunda cor a jogar. \\
 \hline
 \textbf{kyu} & uma classificação amadora de jogadores indo desde aproximadamente 35 kyu (um total iniciante) até 1 kyu, o mais alto nível antes de shodan (1 dan) \\
 \hline
 \textbf{nigiri} & um procedimento para escolher quem joga primeiro em uma partida igual. Veja o \autoref{chap:10:niveis_compensacoes}. \\
 \hline
 \textbf{Nihon Kiin} & a Associação de Go do Japão. Uma fundação sem fins lucrativos baseada em Tóquio com cerca de 300 profissionais ativos. Somente jogadores pertencentes a essa organização ou à Associação Ocidental de Go do Japão são permitidos a competir em torneios japoneses patrocinados por jornais ou empresas \\
 \hline
 \textbf{tesuji} & um movimento habilidoso que conquista um objetivo tático claro \\
 \hline
 \textbf{sanrensei} & um padrão de abertura em que um jogador ocupa três pontos-estrela em um dos lados do tabuleiro \\
 \hline
 \textbf{san-san} & o ponto 3-3 em qualquer um dos cantos \\
 \hline
 \textbf{sente} & um movimento que o oponente não pode ignorar, caso contrário ele sofrerá uma perda inaceitável \\
 \hline
 \textbf{seki} & vida mútua. Uma situação em que nenhum dos dois grupos de cores opostas possui dois olhos, mas nenhum dos lados pode atacar sem perder suas próprias pedras \\
 \hline
\end{longtable}