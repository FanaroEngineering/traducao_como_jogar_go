\chapter{Movimentos de Fim de Jogo}

A posição no \emph{Dia. 1} abaixo é a mesma que o \emph{Dia. 6} que no \autoref{chap:estrat_comp} após Branco ter jogado 13 (a pedra marcada). Quando Preto desce para 1 (14 no \emph{Dia. 6} do \autoref{chap:estrat_comp}), Branco não deveria jogar em outro lugar, com 2 por exemplo, já que Preto invadirá com 3, e Branco não conseguirá defender seu território.

O melhor que ele pode fazer é empurrar com 4 no \emph{Dia. 2} e pular para 6. Porém, Preto pode reduzir Branco rastejando com 7 a 9. Após os movimentos até 13, o território branco basicamente desapareceu. Branco investiu movimentos demais nesta área para deixar isso acontecer.

Sendo assim, quando Preto desce para 1 no \emph{Dia. 3}, Branco não possui nenhuma outra escolha senão defender com 2. Quando Preto desce para 3, ele precisa bloquear com 4 também.

Movimentos como Preto 1 e 3 são geralmente jogados no final do meio de jogo quando o território de cada lado já essencialmente determinado. Eles são utilizados para expandir seu território ao mesmo tempo que se diminiu o do adversário. Em seguida\ldots

Preto joga 5 e 7 no \emph{Dia. 4}, forçando Preto a defender com 6 e 8. Ele continua com o mesmo tipo de movimento na parte inferior do lado direito com 9 e 11, e Branco precisa defender com 12, terminando em \emph{gote}. Em outras palavras, Branco precisa fazer o último movimento defensivo.

Por que Branco 8 e 12 seria necessário?

Depois de Branco 12 no \emph{Dia. 4}, Preto possui sente. Isto é, ele possui a iniciativa --- ele não precisa defender e pode jogar onde quiser. Nesta posição, ele talvez queira jogar em \textbf{A} ou \textbf{B} e conseguiu uma boa vantagem nesta parte do tabuleiro.

Se Branco omitir 8 no \emph{Dia. 4} e se aproximar da pedra preta no canto inferior esquerdo com 1 no \emph{Dia. 5}, Preto cortará com 2. Se Branco tentar escapar com 3, Preto faz atari de novo com 4 e, então, enreda com 6. As pedras pretas não conseguem fugir. Se Branco \textbf{A}, Preto faz atari em \textbf{B}; se Branco \textbf{C}, Preto \textbf{D}.

\section{Cálculos de Fim de Jogo}

Restrinjamos nossa atenção ao lado direito do tabuleiro, assumindo que os outros territórios no tabuleiro já foram decididos. Qual é o valor dos movimentos Preto 5 e 7? E qual é o valor dos movimentos Preto 9 e 11?

Supondo que Branco jogue como aqui, no \emph{Dia. 7} antes de Preto. Se Branco fizer 5, Preto precisa responder com 6. Branco conecta com 7 e Preto precisa defender com 8. Comparado a \emph{Dia. 6}, Branco ganhou dois pontos marcados com \textbf{X}, e Preto perdeu dois pontos em 6 e 8. Isso é um ganho de quatro pontos para Branco.

Branco também termina com sente, então ele pode ainda jogar 9 e 11, forçando Preto a responder com 10 e 12. Comparado ao \emph{Dia. 6}, Branco ganha outros dois pontos, marcados com \(\increment\), e Preto perde dois pontos 10 e 12. Novamente, Branco ganha quatro pontos.

Portanto, qual seja o lado que jogue seus movimentos primeiro, o ganho será de oito pontos.

No entanto, a posição no \emph{Dia. 1} sugiu na abertura, então Preto talvez não desça para 1. Em seu lugar, ele talvez reforce sua posição no canto superior esquerdo com 1 no \emph{Dia. 8}. Branco talvez pense que descer para 2 é um grande movimento, já que ameaça a invasão do canto Preto, mas Preto ignoraria  tal ameaça e asseguraria o canto superior esquerdo com 3 ou \textbf{A}.

Se Branco der continuidade à ameaça de invasão ao canto superior direito preto com 4 e 6, Preto assegurará o canto superior esquerdo com 5 e 7. A invasão branca no canto superior direito é de aproximadamente 30 pontos, mas Preto 1 a 7 delimitam um território enorme à esquerda. É difícil estimar quão grande será esse território. Branco provavelmente conseguirá invadi-lo e criar um grupo vivo, mas Preto será capaz de garantir mais de 60 pontos no processo. Sendo assim, Preto fica feliz com a troca.

O que estamos querendo enfatizer é que movimentos como Preto 1 no \emph{Dia. 1} e Branco 2 no \emph{Dia. 8} são geralmente deixados para o fim de jogo. Na abertura, há sempre movimentos maiores a serem feitos. É claro que Preto 1 no \emph{Dia. 1} é uma grande ameaça, então, do ponto de vista do Branco, ele não pode ignorá-lo. Mas, do ponto de vista do Preto, ele pode facilmente se dar ao luxo de ignorar Branco 2 no \emph{Dia. 8}, já que há pontos maiores em disputa do que simplesmente defender o canto.

O fim de jogo é um aspecto importante do jogo a se dominar. A margem de vitória ou derrota é frequentemente muito pequena, e manobras hábeis de fim de jogo podem virar a partida a seu favor.

A Kiseido já publicou dois excelentes livros sobre esse assunto. Para uma introdução detalhada ao assunto, você deveria ler o \emph{The Endgame}~\cite{tomoko_bozulich_endgame}, o Volume 6 da \emph{Elementary Go Series} da Kiseido. Outro livro é o \emph{Get Strong at the Endgame}~\cite{bozulich_endgame}, Volume 7 da \emph{Ge Strong at Go Series} da Kiseido. Esse livro provê cálculos de fim de jogo para 101 posições básicas. Há também 120 problemas onde você deverá encontrar o tesuji de fim de jogo --- um movimento brilhante que cumpre um papel tático claro --- que maximizará seu lucro. Finalmente, há 70 problemas em tabuleiros 9\(\times\)9 e 13\(\times\)13 que lhe possiblitarão um teste de habilidade no fim de jogo.