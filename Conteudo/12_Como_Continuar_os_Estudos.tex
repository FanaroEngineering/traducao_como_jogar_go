\chapter{Como Continuar os Estudos}

Depois de ter lido este livro e ter jogado vários jogos, você terá adquirido conhecimento básico do jogo e estará bem encaminhado para se tornar um jogador de Go. Porém, para se tornar um jogador forte, você precisará de duas coisas: aprender princípios básicos de estratégia de abertura; e desenvolver sua habilidade analítica. Para a abertura, você deveria começar com o Volume 1 da \emph{Elementary Go Series} da Kiseido, \emph{In the Beginning: The Opening in the Game of Go}~\cite{ikure_in_the_beginning}. Após o estudo desse livro, você terá aprendido os princípios básicos da abertura.

Juntamente com o livro citado acima, você também deveria estudar todos os problemas da série de três volumes \emph{Graded Go Problems for Beginners}~\cite{yoshinori_bozulich_ggpb}. (Os 46 problemas na primeira parte deste livro foram tirados do Volume 1 dessa série.) Os primeiros problemas são extremamente fáceis, mas, gradualmente, eles se tornam mais difíceis. Após ter estudado os quatro volumes dessa série, se você for capaz de voltar e resolver todos os 1,387 problemas só de relance, sua habilidade analítica será de 1 dan.

Você também deveria continuar seu estudo da abertura aprendendo um básico de josekis. Durante a abertura, conflitos críticos frequentemente surgem, iniciando-se dos cantos e se desenvolvendo ao longo dos lados. Esses conflitos são denominados josekis. Nós apresentamos diversos josekis no \autoref{chap:6:estrat_abertura} sobre estratégia de abertura, mas há muitos outros a serem estudados. O segundo volume da \emph{Elementary Go Series}, \emph{38 Basic Josekis}~\cite{ishida_yoshio_basic_joseki_dictionary}, fornecerá a fundação sobre a qual você poderá expandir seu conhecimento de joseki. Você poderá, então, utilizar os dois volumes do \emph{The 21st Century Dictionary of Basic Josekis}~\cite{takao_shinji_21st_century_joseki_dictionary} como referência para incrementar seu repertório de josekis, conforme necessário.

Para o meio de jogo, a melhor introdução é o \emph{Attack and Defense}~\cite{ishida_akira_james_davies}, o Volume 5 da \emph{Elementary Go Series} da Kiseido. Tesujis --- movimentos hábeis que cumprem um objetivo tático claro, que jogadores de xadrez talvez se refiram como ``brilhantismos'' --- também constituem um aspecto importante do meio de jogo. Uma boa introdução ao assunto é o terceiro volume da série \emph{Elementary Go Series}, \emph{Tesuji}~\cite{davies_tesuji}. Você deveria em seguida trabalhar sobre os \emph{501 Tesuji Problems}, no quarto volume da \emph{Mastering the Basics Series}~\cite{bozulich_501_tesuji}, também da Kiseido. Os problemas nesse livro cobrem todos os 45 tesujis básicos que talvez apareçam nos seus jogos. A maestria desses problemas aumentará grandemente a sua força no Go.

Para melhorar o seu poder de meio de jogo, assim como sua compreensão geral de estratégia no Go, é necessário conhecer os princípios fundamentais do jogo. Estes são cobertos no nono volume da \emph{Mastering the Basics Series} da Kiseido, \emph{An Encyclopedia of Go Principles}~\cite{bozulich_encyclopedia_principles}.

Se você quiser se tornar um jogador sériamente comprometido, estudo constante de vida e morte é primordial. O estudo de vida e morte vai lhe treinar a `ler' a posição três ou quatro movimentos à frente. Nós recomendamos que você comece com o \emph{The Basics of Life and Death}~\cite{zeijst_bozulich_basics_of_life_and_death}. Esse livro é uma introdução minuciosa sobre o tópico. Na Parte I, o conceito de vida e morte --- espaço de olho --- é explicado. Em seguida, os três tesujis mais importantes de vida e morte são introduzidos, isto é, os tesujis de: posicionamento, hane e sacrifício. Ao longo do caminho, outro conceito importante é apresentado: escassez de liberdades. Esse conceito é importante não somente em vida e morte, mas, também, em lutas de meio de jogo. Finalmente, a regra do quatro-curvado-no-canto é analisada. Intercalados entre as seções, temos 51 problemas que reforçarão esses conceitos e lhe ajudarão a internalizá-los. A Parte II consiste de 177 posições de vida e morte básicas que surgem em josekis e conflitos de canto e de lado. Essas posições não são uma coleção aleatória de problemas vida e morte. Cada uma é um padrão básico que pode facilmente surgir nos seus jogos. Internalizar essas posições e saber como gerenciá-las vai aumentar dramaticamente a sua força.

A Kiseido e o SmartGo~\cite{smartgo} publicam um número grande de livros de problemas de vida e morte, então há material suficiente para manter até mesmo o mais ávido solucionador de problemas entretido por muitos anos.

Enquanto você estuda todos esses livros e adquire o conhecimento teórico que eles provêm, você deveria jogar quantos jogos puder, para ganhar experiência prática no tabuleiro. Hoje em dia, um jogador pode convenientemente jogar Go na internet através do KGS (\href{https://www.gokgs.com}{gokgs.com}) gratuitamente. Você conseguirá encontrar jogadores desde iniciantes até profissionais. Há também um sistema de nível próprio por lá, então você conseguirá traçar o seu progresso conforme você escala a montanha kyu até chegar aos níveis dans e além.

Por fim, reproduzir partidas de profissionais é outra boa maneira de se melhorar. Isso vai lhe oferecer uma intuição para o fluxo da partida, além de bons exemplos de jogadas. O melhor livro para esse propósito é \emph{Invincible: The Games of Shusaku}~\cite{power_invincible}. Todos os aprendizes de profissionais são demandados a estudar as partidas de Shusaku.