\chapter{Vida e Morte}

Nos \emph{Dias. 15 a 18} no \autoref{chap:cinco}, nós brevemente explicamos a diferença entre olhos reais e olhos falsos. Neste capítulo, mostraremos técnicas para a criação de olhos falsos nos grupos adversários, e explicaremos o conceito de olhos falsos.

\section{Olhos Falsos}

\emph{Dia. 1.} O grupo preto, que está confinado ao canto, possui somente um olho real, e um olho falso --- o ponto em \textbf{A} ---, portanto ele está morto. No final da partida, se Preto se recusar a aceitar que este grupo está morto, Branco pode demonstrá-lo através dos movimentos no \emph{Dia. 2}.

\emph{Dia. 2.} Teoricamente, no final da partida, Branco poderá capturar Preto com 1 a 5. Jogadores experientesnão jogariam tais movimentos, eles reconheceriam tal grupo como morto.

\emph{Dia. 3.} Esse grupo preto não está morto ainda, mas Branco pode matá-lo.

\emph{Dia. 4.} Branco sacrifica uma pedra com 1, fazendo com que o segundo olho preto se torne falso. Se Preto capturar essa pedra com \textbf{A}, ele acabará com uma posição que é essencialmente idêntica ao \emph{Dia. 1}.

\emph{Dia. 5.} Para fazer dois olhos, Preto precisa conectar em 1. O grupo preto não poderá, então, mais ser morto.

\subsection{Exemplo 1}

O grupo preto no \emph{Dia. 1} está instável. Ele viverá ou morrerá, dependendo de qual será o próximo movimento.

Se Branco jogar primeiro, ele matará Preto com a colocação de 1 no \emph{Dia. 2}. Se Preto conectar com 2, Branco toma o ponto-chave de 3. Após a captura preta com 4\ldots

Branco sacrifica uma pedra com 5 no \emph{Dia. 3}, criando um olho falso naquele ponto. Preto não consegue mais fazer um segundo olho, portanto seu grupo está morto.

Se Preto jogar primeiro, ele poderá fazer um segundo olhopara seu grupo pela conexão de 1 no \emph{Dia. 4}. Se Branco \textbf{A}, Preto captura em \textbf{B} e obtém seus dois olhos.

\subsection{Exemplo 2}

O grupo preto no \emph{Dia. 1} está instável, incompleto. Ele viverá ou morrerá baseado no próximo movimento.

Se Branco jogar primeiro, ele poderá matar o grupo preto com o sacrifício de 1 no \emph{Dia. 2}. O ponto 1 agora é um olho falso. Capturar com 2 não ajuda Preto a criar um olho. Seu grupo está morto a partir daí.

O ponto \textbf{A} no \emph{Dia. 3} é um olho falso. Apesar de que Branco não precisa jogar lá, ele pode forçar Preto a jogar naquele ponto com o atari em \textbf{B}, se desafiado a demonstrar que o grupo Preto está morto.

Se Preto jogar primeiro, ele pode fazer um segundo olho para seu grupo conectando em 1 no \emph{Dia. 4}.

\subsection{Exemplo 3}

O grupo preto no \emph{Dia. 1} está incompleto. Vida ou morte dependem da próxima jogada.

Se Branco jogar primeiro, ele poderá jogar 1 no \emph{Dia 2} e transformar o olho em \textbf{A} em um olho falso. Preto está morto.

Se for turno preto, ele pode tornar o ponto \textbf{A} no \emph{Dia. 3} em um olho verdadeiro conectando em 1. 

\section{Espaço de Olho e Forma de Olho}

Quando um grupo cerca um espaço contíguo aberto de vários pontos, a questão de se esse espaço será suficientemente grande para assegurar dois olhos surge.

\emph{Dia. 1.} O grupo preto possui um espaço de três intersecções como olho, e seu status é instável.

\emph{Dia. 2.} Se Preto jogar 1, ele está vivo, pois possui dois olhos separados.

\emph{Dia. 3.} Porém, se Branco jogar 1, Preto está morto.

\emph{Dias. 4 e 5.} Os movimentos até Branco 11 nesses dois diagramas provam que o grupo Preto está morto. Branco simplesmente preenche todas as liberdades pretas mostradas. Preto não tem como se defender.

Aqui seguem mais alguns exemplos.

\emph{Dia. 6.} O grupo preto possui um olho de quatro espaços, e está vivo já.

\emph{Dia. 7.} Se Branco jogar em 1, Preto joga 2 e, novamente, ele está vivo com dois olhos separados.

\emph{Dia. 8.} Similarmente, se Branco jogar 1, Preto jogará 2 e, novamente, ele estará vivo com dois olhos separados.

\emph{Dia. 9.} Após Branco 1, Preto não pode passar ou ignorar o movimento branco. Branco continuará com 3 e Preto morrerá.

\emph{Dia. 10.} Branco pode demonstrar que Preto está morto jogando 5 a 9. Depois da captura por Preto em 10\ldots

\emph{Dia. 11.} Branco joga em 11, e a situação se torna clara: Preto não possui dois olhos.

\emph{Dia. 12.} O grupo preto está vivo ou morto, dependendo de quem for a vez.

\emph{Dia. 13.} Se for turno branco, e ele jogar em 1, o grupo preto está morto.

\emph{Dia. 14.} Se for turno preto, ele pode fazer três olhos jogando em 1.

\emph{Dia. 15.} O grupo preto é um quadrado de quatro espaços, e está morto já.

\emph{Dia. 16.} Mesmo se Preto jogar primeiro, tudo que ele pode fazer é jogar em 1 (ou qualquer outro ponto simétrico) e ameaçar fazer dois olhos jogando em 2. No entanto, Branco pode jogar 2 primeiro, e Preto é deixado sem uma resposta. Se ele jogar qualquer um entre \textbf{A} ou \textbf{B}, ele coloca todo seu grupo em atari.

\emph{Dia. 17.} Se Branco for desafiado a provar que o grupo Preto está morto, tudo que ele precisa fazer é jogar o atari em 4. Se Preto 5 capturar, Branco 6 joga novamente em 4, e todo o grupo preto inevitavelmente será capturado.

\emph{Dia. 18.} O grupo preto está vivo ou morto, dependendo de quem for o turno.

\emph{Dia. 19.} Se Branco jogar primeiro, ele pode matar Preto com 1. Esse movimento põe as cinco pedras pretas em atari. Se Preto conectar em \textbf{A}, todas as pedras estarão em atari. Após Branco 1, o grupo preto está morto, e a posição já está sedimentada.

\emph{Dia. 20.} Se for turno preto, ele pode viver jogando no ponto-chave de 1. Preto está vivo em seki. A posição fica como está. No final da partida, as três pedras brancas permanecerão no tabuleiro, e tanto Preto quanto Branco receberão zero pontos de território nesta região.

\emph{Dia. 21.} Se Branco resistir respondendo Preto 1, no \emph{Dia. 20}, com 2, Preto captura as quatro pedras brancas com 3.

\emph{Dia. 22.} Esse é o resultado da captura das pedras brancas no \emph{Dia. 21}. Preto está vivo com quatro pontos de território mais quatro pedras capturadas. Se Branco \textbf{A}, Preto ganha dois olhos jogando em \textbf{B}. Se Branco \textbf{B}, Preto \textbf{A}. Resistir, como no \emph{Dia. 21}, é uma perda de oito pontos para Branco.

\emph{Dia. 23.} O grupo preto possui um olho de cinco espaços. Ele está vivo ou morto, dependendo de quem for o turno.

\emph{dia. 24.} Depois de Branco 1, não mais nenhuma maneira com que Preto possa fazer dois olhos. Se necessário realmente capturar o grupo, Branco pode colocá-lo sob atari ocupando as intersecções marcadas com \textbf{X}. Se Preto, então, capturar com \textbf{A}, Preto acaba com um olho de quatro espaços, que é a mesma posição do \emph{Dia. 15}, onde mostramos que o grupo Preto está morto já.

\emph{Dia. 25.} Se for turno preto, ele poderá fazer dois olhos jogando no ponto-chave de 1.

O tópico de espaço de olho é minuciosamente contemplado no livro \emph{The Basics of Life and Death}~\cite{zeijst_bozulich_basics_of_life_and_death}, publicado pela Kiseido.