\chapter{Catálogo Kiseido em Português}\label{ap:pt}

\textbf{\emph{Note, porém, que, basicamente, nenhum dos livros abaixo foram traduzidos para o português. O catálogo abaixo é apenas uma tradução somente dos títulos em inglês.}}

\bigskip

\emph{Livros marcados com um asterisco também estão disponíveis em iOS através do aplicativo Go Books~\cite{gobooks} ou SmartGo~\cite{smartgo}.}

\section{Introdutório e Geral}

\begin{longtable}{l|p{50mm}|p{50mm}} 
    \hline
    \textbf{Código} & \textbf{Título} & \textbf{Autor(es)} \\
    \hline \hline
    K50* & Go --- Uma Introdução Completa ao Jogo & Cho Chikun \\
    \hline
    K40 & O Almanaque do Jogador de Go de 2001 & Richard Bozulich (editor) \\
    \hline
    K30 & Gravuras Japonesas e Mundo do Go & William Pinckard e Kitagawa Akiko \\
    \hline
\end{longtable}


\section{Livros de Problemas para Iniciantes}

\emph{Livros de Problemas Nivelados para Iniciantes}, por Kano Yoshinori:

\begin{longtable}{c|c|l} 
    \hline
    \textbf{Código} & \textbf{Volume} & \textbf{Título} \\
    \hline \hline
    K46* & 1 & Problemas Introdutórios (35-25 kyu) \\
    \hline
    K47* & 2 & Problemas Elementares (25-20 kyu) \\
    \hline
    K48* & 3 & Problemas Intermediários (20-15 kyu) \\
    \hline
    K49* & 4 & Problemas Avançados (15-1 kyy) \\
    \hline
\end{longtable}


\section{Série Go Elementar}

Por mais de 30 anos, a \emph{Série Go Elementar} tem sido o texto padrão de jogadores de Go que querem obter um domínio firme dos fundamentos de Go. Não é só a teoria que é desenvolvida nessa série, o leitor também receberá problemas que demonstram como esses conceitos teóricos são aplicados a jogos reais.

\begin{longtable}{c|c|p{45mm}|p{45mm}} 
    \hline
    \textbf{Código} & \textbf{Volume} & \textbf{Título} &\textbf{Autor(es)} \\
    \hline \hline
    K10* & 1 & No Início --- A Abertura no Jogo de Go & Ishigure Ikuro \\
    \hline
    K11* & 2 & 38 Josekis Básicos & Kosugi Kiyoshi e James Davies \\
    \hline
    K12* & 3 & Tesuji & James Davies \\
    \hline
    K13* & 4 & Vida e Morte & James Davies \\
    \hline
    K14* & 5 & Ataque e Defesa & Ishida Akira e James Davies \\
    \hline
    K15* & 6 & O Fim de Jogo & Ogawa Tomoko e James Davies \\
    \hline
    K16* & 7 & Go de Compensação & Nagahara Yoshiaki e Richard Bozulich \\
    \hline
\end{longtable}

\section{Série ``Fique Forte"}

Uma série de livros de problemas cobrindo cada etapa do jogo, desde a abertura até o fim de jogo. Cada livro contém 170 ou mais problemas, com dificuldade desde elementar à avançada. Portanto, eles podem ser utilizados por jogadores indo de 20 kyu até nível dan. Através do estudo desse formato, você poderá não só aprender os princípios básicos de como os movimentos se intercalam como também se treinar sobre como analisar e pensar sobre posições. Você irá se deparar padrões similares ou idênticos aos que surgirão em suas próprias partidas. Nós podemos garantir que o estudo diligente desta série inteira constituirá a fundação de um jogador verdadeiramente forte.

\bigskip

\emph{Todos os livros desta série foram autorados pelo Richard Bozulich.}

\begin{longtable}{l|c|l} 
    \hline
    \textbf{Código} & \textbf{Volume} & \textbf{Título} \\
    \hline \hline
    K51 & 1 & Fique Forte na Abertura \\
    \hline
    K52 & 2 & Fique Forte em Josekis I \\
    \hline
    K53 & 3 & Fique Forte em Josekis II \\
    \hline
    K54 & 4 & Fique Forte em Josekis III \\
    \hline
    K55 & 5 & Fique Forte em Invasões \\
    \hline
    K56 & 6 & Fique Forte em Tesujis \\
    \hline
    K57* & 7 & Fique Forte no Fim de Jogo \\
    \hline
    K58* & 8 & Fique Forte em Vida e Morte \\
    \hline
    K59* & 9 & Fique Forte em Go de Compensação \\
    \hline
    K60 & 10 & Fique Forte em Atacar \\
    \hline
\end{longtable}

\section{Dominando o Básico}

Uma série de livros, especificamente escrita para jogadores de kyu alto, para se dominar as técnicas básicas do Go. Cada livro nessa série consiste de centenas de problemas projetados para transmitir de maneira sólida os conceitos fundamentais da teoria e das técnicas do Go. Um estudo minucioso e detalhado dessa série é o caminho mais rápido para se avançar pelos níveis kyu.

\begin{longtable}{l|c|p{55mm}|p{35mm}} 
    \hline
    \textbf{Código} & \textbf{Volume} & \textbf{Título} & \textbf{Autor(es)} \\
    \hline \hline
    K71* & 1 & 501 Problemas de Abertura & Richard Bozulich e Rob van Zeijst \\
    \hline
    K72 & 2 & 1001 Problemas de Vida e Morte & Richard Bozulich \\
    \hline
    K73* & 3 & Criando Boa Forma & Rob van Zeijst e Richard Bozulich \\
    \hline
    K74* & 4 & 501 Problemas de Tesuji & Richard Bozulich \\
    \hline
    K75* & 5 & O Básico de Estratégia no Go & Richard Bozulich \\
    \hline
    K76* & 6 & Tudo sobre Ko & Rob van Zeijst e Richard Bozulich \\
    \hline
    K77* & 7 & Atacando e Defendendo Moyos & Rob van Zeijst e Richard Bozulich \\
    \hline
    K78* & 8 & Lute como um Profissional --- Os Segredos do Kiai & Rob van Zeijst e Richard Bozulich \\
    \hline
    K79* & 9 & Uma Enciclopédia de Princípios de Go & Richard Bozulich \\
    \hline
    K80 & 10 & Encontros Íntimos com o Meio de Jogo & Michiel Eijkhout \\
    \hline
\end{longtable}

\pagebreak

\section{Livros Elementares}

\begin{longtable}{l|p{55mm}|p{55mm}} 
    \hline
    \textbf{Código} & \textbf{Título} & \textbf{Autor(es)} \\
    \hline \hline
    K02* & Técnicas básicas de Go & Nagahara Yoshiaki e Haruyama Isamu \\
    \hline
    K28 & Lições nos Fundamentos de Go & Kageyama Toshiro \\
    \hline
    K36 & Teoria de Abertura, Facilitada & Otake Hideo \\
    \hline
    K82* & Estratégia de Go de Compensação e a Abertura Sanrensei & Rob van Zeijst e Richard Bozulich \\
    \hline
    K83 & Princípios Básicos de Abertura e Meio de Jogo & Rob van Zeijst e Richard Bozulich \\
    \hline
    K84* & O Básico de Vida e Morte & Rob van Zeijst e Richard Bozulich \\
    \hline
    K85 & Um Levantamento de Tesujis Básicos & Richard Bozulich \\
    \hline
\end{longtable}

\section{Advanced Books}

Da série \emph{Problemas Nivelados para Jogadores Dan}:

\begin{longtable}{c|c|l} 
    \hline
    \textbf{Código} & \textbf{Volume} & \textbf{Título} \\
    \hline \hline
    K61 & 1 & 300 Problemas de Vida e Morte 5 kyu to 3 dan \\
    \hline
    K62 & 2 & 300 Problemas de Tesuji 5 kyu to 3 dan \\
    \hline
    K63 & 3 & 300 Problemas de Joseki 1 dan to 3 dan \\
    \hline
    K64 & 4 & 300 Problemas de Vida e Morte 4 dan to 7 dan \\
    \hline
    K65 & 5 & 300 Problemas de Tesuji 4 dan to 7 dan \\
    \hline
    K66 & 6 & 300 Problemas de Joseki 4 dan to 7 dan \\
    \hline
    K67 & 7 & Problemas de Abertura e Meio de Jogo 1 dan to 7 dan \\
    \hline
\end{longtable}

\pagebreak

Outros livros avançados:

\begin{longtable}{l|p{55mm}|p{55mm}} 
    \hline
    \textbf{Código} & \textbf{Título} & \textbf{Autor(es)} \\
    \hline \hline
    K81 & Um Dicionário de Fusekis Modernos: O Estilo Coreano & Kim Song June \\
    \hline
    K29 & Reduzindo Armações Territoriais & Fujisawa Shuko \\
    \hline
    K41 & O Dicionário de Josekis Básicos do Século 21, Vol. 1 & Takao Shinji \\
    \hline
    K42 & O Dicionário de Josekis Básicos do Século 21, Vol. 2 & Takao Shinji \\
    \hline
\end{longtable}

\section{Coleções de Partidas}

\begin{longtable}{l|p{55mm}|p{55mm}} 
    \hline
    \textbf{Código} & \textbf{Título} & \textbf{Autor(es)} \\
    \hline \hline
    K01* & Invencível: As Partidas de Shusaku & John Power (editor e compilador) \\
    \hline
    K07 & O Torneio Honinbo de 1971 & Iwamoto Kaoru \\
    \hline
    K91* & Partidas de Mestres Modernos, Vol. 1, O Despertar do Go Competitivo & Rob van Zeijst e Richard Bozulich \\
    \hline
    K92 & Partidas de Mestres Modernos, Vol. 2, O Embate de Dez Partidas entre Gu Li e Lee Sedol de 2014, Parte I: Partidas Um a Cinco & Rob van Zeijst \\
    \hline
    K93 & Partidas de Mestres Modernos, Vol. 3, O Embate de Dez Partidas entre Gu Li e Lee Sedol de 2014, Parte II: Partidas Seis a Oito & Michael Redmond 9p e Rob van Zeijst \\
    \hline
    PP01 & Go Competitivo, 1992 & John Power (editor e compilador) \\
    \hline
\end{longtable}

\section{O Mundo do Go}

\emph{O Mundo do Go} foi uma revista trimestral que cubria o Go competitivo japonês e internacional. Ela continha análise de partidas de torneios e artigos instrucionais para jogadores de todos os níveis.

Um conjunto completo de O Mundo do Go desde a primeira edição até a \#129 está disponível em três DVDs pela Kiseido Digital. Para pedi-los, vá até \href{https://www.kiseidodigital.com}{\path{kiseidodigital.com}}.

Para pedir livros e equipamento de Go, vá até a loja online da Kiseido: \href{https://www.kiseido.com}{\path{kiseido.com}}.

\pagebreak

\section{Direto do Japão}

Segue o endereço da Kiseido no Japão:

\bigskip
\bigskip

\begin{tabular}{l}
    \hline
    Kiseido Publishing Company \\
    Kagawa 4-48-32 \\
    Chigasaki-shi \\
    Kanagawa-ken \\
    Japan 253-0082 \\
    FAX +81-467-81-0605 \\
    email: \href{mailto:kiseido@yk.rim.or.jp}{kiseido@yk.rim.or.jp} \\
    \href{https://www.kiseido.com}{kiseido.com} \\
    \hline
\end{tabular}