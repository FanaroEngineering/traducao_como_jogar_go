%-------------------------------------------------------------------------------
\documentclass[11pt]{book}
%-------------------------------------------------------------------------------
% Packages

% Tamanho Amazon mais próximo de A5:
\usepackage[paperheight=  8.5in,
            paperwidth=   5.5in,
            bindingoffset=0.5in,
            left=         0.0in,
            right=        0.25in,
            top=          0.7in,
            bottom=       0.6in,
            footskip=     0.25in]{geometry}

\usepackage{titlesec}
\titleformat{\chapter}[display]
    {\normalfont\huge\bfseries}
    {\chaptertitlename\ \thechapter}
    {20pt}
    {\Huge}
\titlespacing*{\chapter}{0pt}{0pt}{25pt}
\raggedbottom
\frenchspacing

\usepackage[utf8]{inputenc}
\usepackage[T1]{fontenc}
\usepackage[portuguese]{babel}

\usepackage[math-style=ISO]{unicode-math}
\usepackage{amstext}

\usepackage{graphicx}
\graphicspath{
  {./Diagramas EPS/1/}
  {./Diagramas EPS/2/}
  {./Diagramas EPS/3/}
  {./Diagramas EPS/4/}
  {./Diagramas EPS/5/}
  {./Diagramas EPS/6/}
  {./Diagramas EPS/7/}
  {./Diagramas EPS/8/}
  {./Diagramas EPS/9/}
  {./Diagramas EPS/10/}
  {./Diagramas EPS/11/}
}
\usepackage{subcaption}
\usepackage{wrapfig}
\usepackage[rightcaption,raggedright]{sidecap}
\sidecaptionvpos{figure}{t}
\usepackage[skip=0pt,justification=raggedright,singlelinecheck=false]{caption}

\usepackage[colorlinks= true,
            allcolors=  black,
            pdfsubject= {Go; Baduk; Weiqi},
            pdftitle=   {Como Jogar Go: Uma Introdução Concisa},
            pdfauthor=  {Richard Bozulich; James Davies; Philippe Fanaro},
            pdfkeywords={Go; Baduk; Weiqi; Introdução},
            pdfproducer={LaTeX},
            pdfcreator= {pdflatex; bibtex},
            bookmarksnumbered]{hyperref}

\usepackage{enumitem}
\setitemize{fullwidth}
\setenumerate{widest}
\newcommand{\RomanNumeralCaps}[1]
    {\MakeUppercase{\romannumeral #1}}

\usepackage{longtable}
\renewcommand{\arraystretch}{1.5}

\usepackage{indentfirst}

\usepackage[nottoc,numbib]{tocbibind}
\addto\captionsportuguese{
  \renewcommand{\bibname}{Referências}
  \renewcommand{\contentsname}{Índice}
}
\usepackage{shorttoc}
\addtocontents{toc}{\protect\thispagestyle{empty}}
\pagenumbering{gobble}

\usepackage[bottom]{footmisc}

% Necessita especificamente de XeLaTeX.
\usepackage{xeCJK}
\xeCJKDeclareSubCJKBlock{Kana}{"3040 -> "309F, "30A0 -> "30FF, "31F0 -> "31FF, "1B000 -> "1B0FF}
\xeCJKDeclareSubCJKBlock{Hangul}{"1100 -> "11FF, "3130 -> "318F, "A960 -> "A97F, "AC00 -> "D7AF, "D7B0 -> "D7FF}

\setCJKmainfont{SimSun}
\setCJKmainfont[Kana]{MS Mincho}
\setCJKmainfont[Hangul]{Batang} % É preciso instalar a fonte Batang no sistema
%-------------------------------------------------------------------------------
% Content

\begin{document}
  \begin{titlepage}
    \centering
    
    \scshape
    
    \vspace*{\baselineskip}
    
    % Título
    
    \rule{\textwidth}{1.6pt}\vspace*{-\baselineskip}\vspace*{2pt}
    \rule{\textwidth}{0.4pt}
    
    \vspace{0.7\baselineskip}
    
    \Huge{Como Jogar Go}\\
    \vspace*{10pt}
    \LARGE{Uma Introdução Concisa}
    
    \vspace{0.275\baselineskip}
    
    \rule{\textwidth}{0.4pt}\vspace*{-\baselineskip}\vspace{3.2pt}
    \rule{\textwidth}{1.6pt}
    
    % Contribuidores
    
    \vspace*{2cm}

    \large{por}

    \vspace*{0.125cm}
    
    \Large{Richard Bozulich \\ James Davies}
    
    \vspace*{1.5cm}

    \large{traduzido por}

    \vspace*{0.125cm}

    \Large{Philippe Fanaro}
    
    \vfill
    
    % Editora e Logos
    
    \large{Kiseido Publishing Company}

    \vspace*{0.25cm}
    
    \large{Fanaro.io}

    \vspace{1.5cm}

    \large{2021}
\end{titlepage}

  \cleardoublepage

  \shorttoc{Índice Resumido}{0}
  \addcontentsline{toc}{chapter}{Índice}
  \tableofcontents

  \cleardoublepage

  \frontmatter
  \chapter{Prefácio}

Go\footnote{Também conhecido como \emph{igo} (囲碁) no Japão, \emph{baduk} (바둑) na Coreia e \emph{weiqi} (围棋) na China.} é o jogo de tabuleiro oriental milenar jogado por milhões de pessoas ao redor do mundo. As regras são tão fáceis que uma criança de cinco anos pode aprendê-lo; porém, a verdadeira maestria demanda anos de estudos intensivos. Assim como a natureza, seus simples elementos --- madeira e pedra, linha e círculo, preto e branco --- constroem estruturas complexas.

Go é o jogo perfeito para se desenvolver habilidades tanto do lado esquerdo quanto direito do cérebro --- tanto intuição e reconhecimento de padrões quanto análise por força bruta. Adicionalmente, a riqueza de seus conceitos estratégicos permitem que o Go seja utilizado como modelo para tomadas de decisões na vida real.

Go também ensina o valor da paciência, além de demonstrar como balancear agressividade e prudência, para ganhos máximos.

\bigskip

\emph{Como Jogar Go: Uma Introdução Concisa}~\cite{bozulich_how_to_play_go} é uma iniciação simples e direta ao jogo. As 10 regras básicas são claramente explicadas em somente algumas páginas. Problemas são intercalados durante o livro, para ajudar na compreensão dos conceitos sendo apresentados. Após as regras terem sido detalhadamente exploradas, há uma seção sobre estratégias de abertura, seguida de uma outra sobre táticas. Tudo que é essencial à prática do jogo está incluído com diversas referências a recursos que ajudarão o leitor a continuar em sua jornada neste milenar e profundo jogo.

\bigskip
\bigskip

Richard Bozulich

20 de Dezembro de 2016
  \chapter{Nota do Tradutor}

% TODO: Citar mídias sociais
% TODO: Citar livro do Felipe com a secretaria municipal

Primeiramente, agradeço muito ao Richard Bozulich e ao James Davies por, não somente terem escrito este livro, mas, também, terem liberados os direitos de tradução. Espero que seja o primeiro de muitos livros traduzidos para o português da excelente editora Kiseido~\cite{kiseido}.

\bigskip

A tradução deste livro foi feita com base em meu conhecimento do jogo. Sou um jogador amador de força aproximada mínima de 1 dan, um ranking tido como de maestria do jogo, apesar de que, claro, sempre há um peixe maior. Desde que comecei a jogar no final de 2012, já participei de torneios tanto no Brasil quanto na Europa, na Coreia e na China, além de alguns torneios online também. Toda essa experiência me ajuda a discernir o que é bom do que é ruim, o que é útil do que é perda de tempo.

É por isso que corri atrás de conteúdo da editora Kiseido, que vem trazendo ao Ocidente, em inglês, conhecimento de Go já há décadas. Este livro é uma introdução sólida e sistemática ao jogo, com muita concisão. Nele, o leitor encontrará, além das regras, conceitos e técnicas utilizadas por desde iniciantes até grandes mestres do Go, táticas e estratégias que residem no coração jogo.

Porém, não tenha pressa. Go demanda muita prática, e esta é muito divertida e recompensadora. Internalizar os conceitos deste livro é algo que não só demanda tempo, mas que também é feito em ciclos, mesmo os grandes mestres aprendem com os fundamentos todos os dias.

\pagebreak

Após o término deste livro, infelizmente, não há muito mais conteúdo em português, pelo menos não em via impressa ou em livros. Tentarei continuar criando mais traduções de livro de Go, mas não sei a que velocidade ou até que ponto. No entanto, já há conteúdo online para variados níveis, em português, espalhado pela internet.

Algumas dessas outras fontes com mais conteúdo são:

% TODO: Adicionar aplicativo Go Books para iOS
% TODO: Adicionar SmartGo (ambos Go Books e SmartGo são mencionados bem no final do livro)
\begin{longtable}{l|p{60mm}} 
 \hline
 \textbf{URL} & \textbf{Descrição} \\
 \hline \hline
 \href{https://online-go.com}{\path{online-go.com}}~\cite{ogs} & o servidor OGS é essencialmente uma versão mais moderna do servidor KGS, citado no livro. A interface é mais bem feita e há muitos outros recursos, incluindo tutoriais e josekipédia. Este é o servidor que eu mais recomendo para iniciantes \\
 \hline
 \href{https://facebook.com/groups/gobrasil}{\path{facebook.com/groups/gobrasil}}~\cite{facebook_go_brasil} & grupo \emph{Go Brasil} no Facebook, com mais de 1200 integrantes. Também há um grupo de Whatsapp bastante ativo, mas será preciso demandar pelo grupo de Facebook como acessá-lo \\
 \hline
 \href{https://nihonkiin.com.br}{\path{nihonkiin.com.br}}~\cite{brasil_nihon_kiin} & site da Brasil Nihon Kiin, associação nipo-brasileira de Go, que, também, existe fisicamente em São Paulo-SP \\
 \hline
 \href{https://fanaro.io}{\path{fanaro.io}}~\cite{fanaroio} & meu site, com conteúdo não só de Go \\
 \hline
 \href{https://youtube.com/c/PhilippeFanaro}{\path{youtube.com/c/PhilippeFanaro}}~\cite{fanaro_youtube} & meu canal de YouTube, focado em Go \\
 \hline
 \href{https://twitch.tv/fanaro009}{\path{twitch.tv/fanaro009}}~\cite{fanaro_twitch} & meu canal de Twitch, onde jogo e ensino ao vivo, ocasionalmente \\
 \hline
\end{longtable}

Enfim, sua jornada está só começando, e um guia completo como este será de grande valia na aventura.

Para mais informações, ou caso queira fornecer feedback, melhorias ou comentários, é só enviar um email para \emph{\href{mailto:philippefanaro@gmail.com}{philippefanaro@gmail.com}}~\cite{fanaro_email}.

% TODO: Alguns comentários sobre a tradução (por exemplo, de termos peculiares)

\bigskip
\bigskip

Philippe Fanaro

1 de Outubro de 2021
\end{document}
%-------------------------------------------------------------------------------