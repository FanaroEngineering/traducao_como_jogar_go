%-------------------------------------------------------------------------------
\documentclass[11pt]{book}
%-------------------------------------------------------------------------------
% Packages

% Tamanho Amazon mais próximo de A5:
\usepackage[paperheight=  8.5in,
            paperwidth=   5.5in,
            bindingoffset=0.5in,
            left=         0.0in,
            right=        0.25in,
            top=          0.7in,
            bottom=       0.6in,
            footskip=     0.25in]{geometry}

\usepackage{titlesec}
\titleformat{\chapter}[display]
    {\normalfont\huge\bfseries}
    {\chaptertitlename\ \thechapter}
    {20pt}
    {\Huge}
\titlespacing*{\chapter}{0pt}{0pt}{25pt}
\raggedbottom
\frenchspacing

\usepackage[utf8]{inputenc}
\usepackage[T1]{fontenc}
\usepackage[portuguese]{babel}

\usepackage[math-style=ISO]{unicode-math}
\usepackage{amstext}

\usepackage{graphicx}
\graphicspath{
  {./Diagramas EPS/1/}
  {./Diagramas EPS/2/}
  {./Diagramas EPS/3/}
  {./Diagramas EPS/4/}
  {./Diagramas EPS/5/}
  {./Diagramas EPS/6/}
  {./Diagramas EPS/7/}
  {./Diagramas EPS/8/}
  {./Diagramas EPS/9/}
  {./Diagramas EPS/10/}
  {./Diagramas EPS/11/}
}
\usepackage{subcaption}
\usepackage{wrapfig}
\usepackage[rightcaption,raggedright]{sidecap}
\sidecaptionvpos{figure}{t}
\usepackage[skip=0pt,justification=raggedright,singlelinecheck=false]{caption}

\usepackage[colorlinks= true,
            allcolors=  black,
            pdfsubject= {Go; Baduk; Weiqi},
            pdftitle=   {Como Jogar Go: Uma Introdução Concisa},
            pdfauthor=  {Richard Bozulich; James Davies; Philippe Fanaro},
            pdfkeywords={Go; Baduk; Weiqi; Introdução},
            pdfproducer={LaTeX},
            pdfcreator= {pdflatex; bibtex},
            bookmarksnumbered]{hyperref}

\usepackage{enumitem}
\setitemize{fullwidth}
\setenumerate{widest}
\newcommand{\RomanNumeralCaps}[1]
    {\MakeUppercase{\romannumeral #1}}

\usepackage{longtable}
\renewcommand{\arraystretch}{1.5}

\usepackage{indentfirst}

\usepackage[nottoc,numbib]{tocbibind}
\addto\captionsportuguese{
  \renewcommand{\bibname}{Referências}
  \renewcommand{\contentsname}{Índice}
}
\usepackage{shorttoc}
\addtocontents{toc}{\protect\thispagestyle{empty}}
\pagenumbering{gobble}

\usepackage[bottom]{footmisc}

\usepackage{xeCJK}
%-------------------------------------------------------------------------------
% Content

\begin{document}
  \begin{titlepage}
    \centering
    
    \scshape
    
    \vspace*{\baselineskip}
    
    % Título
    
    \rule{\textwidth}{1.6pt}\vspace*{-\baselineskip}\vspace*{2pt}
    \rule{\textwidth}{0.4pt}
    
    \vspace{0.7\baselineskip}
    
    \Huge{Como Jogar Go}\\
    \vspace*{10pt}
    \LARGE{Uma Introdução Concisa}
    
    \vspace{0.275\baselineskip}
    
    \rule{\textwidth}{0.4pt}\vspace*{-\baselineskip}\vspace{3.2pt}
    \rule{\textwidth}{1.6pt}
    
    % Contribuidores
    
    \vspace*{2cm}

    \large{por}

    \vspace*{0.125cm}
    
    \Large{Richard Bozulich \\ James Davies}
    
    \vspace*{1.5cm}

    \large{traduzido por}

    \vspace*{0.125cm}

    \Large{Philippe Fanaro}
    
    \vfill
    
    % Editora e Logos
    
    \large{Kiseido Publishing Company}

    \vspace*{0.25cm}
    
    \large{Fanaro.io}

    \vspace{1.5cm}

    \large{2021}
\end{titlepage}

  \cleardoublepage

  \shorttoc{Índice Resumido}{0}
  \addcontentsline{toc}{chapter}{Índice}
  \tableofcontents

  \cleardoublepage

  \frontmatter
  \chapter{Prefácio}

Go\footnote{Também conhecido como \emph{igo} (囲碁) no Japão, \emph{baduk} (바둑) na Coreia e \emph{weiqi} (围棋) na China.} é o jogo de tabuleiro oriental milenar jogado por milhões de pessoas ao redor do mundo. As regras são tão fáceis que uma criança de cinco anos pode aprendê-lo; porém, a verdadeira maestria demanda anos de estudos intensivos. Assim como a natureza, seus simples elementos --- madeira e pedra, linha e círculo, preto e branco --- constroem estruturas complexas.

Go é o jogo perfeito para se desenvolver habilidades tanto do lado esquerdo quanto direito do cérebro --- tanto intuição e reconhecimento de padrões quanto análise por força bruta. Adicionalmente, a riqueza de seus conceitos estratégicos permitem que o Go seja utilizado como modelo para tomadas de decisões na vida real.

Go também ensina o valor da paciência, além de demonstrar como balancear agressividade e prudência, para ganhos máximos.

\bigskip

\emph{Como Jogar Go: Uma Introdução Concisa}~\cite{bozulich_how_to_play_go} é uma iniciação simples e direta ao jogo. As 10 regras básicas são claramente explicadas em somente algumas páginas. Problemas são intercalados durante o livro, para ajudar na compreensão dos conceitos sendo apresentados. Após as regras terem sido detalhadamente exploradas, há uma seção sobre estratégias de abertura, seguida de uma outra sobre táticas. Tudo que é essencial à prática do jogo está incluído com diversas referências a recursos que ajudarão o leitor a continuar em sua jornada neste milenar e profundo jogo.

\bigskip
\bigskip

Richard Bozulich

20 de Dezembro de 2016
  \chapter{Nota do Tradutor}

% TODO: Citar mídias sociais
% TODO: Citar livro do Felipe com a secretaria municipal

Primeiramente, agradeço muito ao Richard Bozulich e ao James Davies por, não somente terem escrito este livro, mas, também, terem liberados os direitos de tradução. Espero que seja o primeiro de muitos livros traduzidos para o português da excelente editora Kiseido~\cite{kiseido}.

\bigskip

A tradução deste livro foi feita com base em meu conhecimento do jogo. Sou um jogador amador de força aproximada mínima de 1 dan, um ranking tido como de maestria do jogo, apesar de que, claro, sempre há um peixe maior. Desde que comecei a jogar no final de 2012, já participei de torneios tanto no Brasil quanto na Europa, na Coreia e na China, além de alguns torneios online também. Toda essa experiência me ajuda a discernir o que é bom do que é ruim, o que é útil do que é perda de tempo.

É por isso que corri atrás de conteúdo da editora Kiseido, que vem trazendo ao Ocidente, em inglês, conhecimento de Go já há décadas. Este livro é uma introdução sólida e sistemática ao jogo, com muita concisão. Nele, o leitor encontrará, além das regras, conceitos e técnicas utilizadas por desde iniciantes até grandes mestres do Go, táticas e estratégias que residem no coração jogo.

Porém, não tenha pressa. Go demanda muita prática, e esta é muito divertida e recompensadora. Internalizar os conceitos deste livro é algo que não só demanda tempo, mas que também é feito em ciclos, mesmo os grandes mestres aprendem com os fundamentos todos os dias.

\pagebreak

Após o término deste livro, infelizmente, não há muito mais conteúdo em português, pelo menos não em via impressa ou em livros. Tentarei continuar criando mais traduções de livro de Go, mas não sei a que velocidade ou até que ponto. No entanto, já há conteúdo online para variados níveis, em português, espalhado pela internet.

Algumas dessas outras fontes com mais conteúdo são:

% TODO: Adicionar aplicativo Go Books para iOS
% TODO: Adicionar SmartGo (ambos Go Books e SmartGo são mencionados bem no final do livro)
\begin{longtable}{l|p{60mm}} 
 \hline
 \textbf{URL} & \textbf{Descrição} \\
 \hline \hline
 \href{https://online-go.com}{\path{online-go.com}}~\cite{ogs} & o servidor OGS é essencialmente uma versão mais moderna do servidor KGS, citado no livro. A interface é mais bem feita e há muitos outros recursos, incluindo tutoriais e josekipédia. Este é o servidor que eu mais recomendo para iniciantes \\
 \hline
 \href{https://facebook.com/groups/gobrasil}{\path{facebook.com/groups/gobrasil}}~\cite{facebook_go_brasil} & grupo \emph{Go Brasil} no Facebook, com mais de 1200 integrantes. Também há um grupo de Whatsapp bastante ativo, mas será preciso demandar pelo grupo de Facebook como acessá-lo \\
 \hline
 \href{https://nihonkiin.com.br}{\path{nihonkiin.com.br}}~\cite{brasil_nihon_kiin} & site da Brasil Nihon Kiin, associação nipo-brasileira de Go, que, também, existe fisicamente em São Paulo-SP \\
 \hline
 \href{https://fanaro.io}{\path{fanaro.io}}~\cite{fanaroio} & meu site, com conteúdo não só de Go \\
 \hline
 \href{https://youtube.com/c/PhilippeFanaro}{\path{youtube.com/c/PhilippeFanaro}}~\cite{fanaro_youtube} & meu canal de YouTube, focado em Go \\
 \hline
 \href{https://twitch.tv/fanaro009}{\path{twitch.tv/fanaro009}}~\cite{fanaro_twitch} & meu canal de Twitch, onde jogo e ensino ao vivo, ocasionalmente \\
 \hline
\end{longtable}

Enfim, sua jornada está só começando, e um guia completo como este será de grande valia na aventura.

Para mais informações, ou caso queira fornecer feedback, melhorias ou comentários, é só enviar um email para \emph{\href{mailto:philippefanaro@gmail.com}{philippefanaro@gmail.com}}~\cite{fanaro_email}.

% TODO: Alguns comentários sobre a tradução (por exemplo, de termos peculiares)

\bigskip
\bigskip

Philippe Fanaro

1 de Outubro de 2021
  \chapter[Glossário de Termos de Go]{Glossário de Termos de Go Utilizados Neste Livro}

\begin{longtable}{l|p{85mm}} 
 \hline
 \textbf{Termo} & \textbf{Significado} \\
 \hline \hline
 \textbf{atari} & quando todas as intersecções exceto uma diretamente adjacente à pedra ou grupo de pedras são ocupadas pelo oponente, a pedra ou grupo é dita estar sob ``atari"\footnote{Sim, este termo de Go inspirou o nome do console icônico de videogames, Atari.}, e o oponente pode capturá-la no próximo movimento \\ 
 \hline
 \textbf{dan} & uma classificação de força de jogadores indo de shodan (1 dan) até 9 dan \\
 \hline
 \textbf{gote} & um movimento que o oponente pode ignorar. Um movimento gote é geralmente defensivo, então não representa nenhuma ameaça ao adversário \\
 \hline
 \textbf{joseki} & durante a abertura, conflitos locais frequentemente surgem, começando tipicamente nos cantos e se desenvolvendo para os lados. Esses conflitos locais se chamam josekis \\
 \hline
 \textbf{komi} & compensação de pontos adicionada a um dos lados no final da partida. Tipicamente, 6.5 pontos ao território do Branco em uma partida igualitária, para compensar a desvantagem de ser a segunda cor a jogar. \\
 \hline
 \textbf{kyu} & uma classificação amadora de jogadores indo desde aproximadamente 35 kyu (um total iniciante) até 1 kyu, o mais alto nível antes de shodan (1 dan) \\
 \hline
 \textbf{nigiri} & um procedimento para escolher quem joga primeiro em uma partida igual. Veja o \autoref{chap:10:niveis_compensacoes}. \\
 \hline
 \textbf{Nihon Kiin} & a Associação de Go do Japão. Uma fundação sem fins lucrativos baseada em Tóquio com cerca de 300 profissionais ativos. Somente jogadores pertencentes a essa organização ou à Associação Ocidental de Go do Japão são permitidos a competir em torneios japoneses patrocinados por jornais ou empresas \\
 \hline
 \textbf{tesuji} & um movimento habilidoso que conquista um objetivo tático claro \\
 \hline
 \textbf{sanrensei} & um padrão de abertura em que um jogador ocupa três pontos-estrela em um dos lados do tabuleiro \\
 \hline
 \textbf{san-san} & o ponto 3-3 em qualquer um dos cantos \\
 \hline
 \textbf{sente} & um movimento que o oponente não pode ignorar, caso contrário ele sofrerá uma perda inaceitável \\
 \hline
 \textbf{seki} & vida mútua. Uma situação em que nenhum dos dois grupos de cores opostas possui dois olhos, mas nenhum dos lados pode atacar sem perder suas próprias pedras \\
 \hline
\end{longtable}

  \mainmatter
  \chapter{O Equipamento}

Go é geralmente jogado em um tabuleiro com 19 linhas verticais e 19 horizontais, constituindo uma grade de 361 intersecções, como se pode ver pelo \emph{Diagrama 1}. O tabuleiro possui nove pontos grifados com pequenos círculos. Esses círculos são chamados de pontos-estrela.

As peças ou pedras são tipicamente duplo-convexas, para facilitar o manuseio quando de muitas pedras em uma mesma área. Um conjunto de Go padrão possui 181 pedras pretas e 180 pedras brancas.

\begin{wrapfigure}{r}{60mm}
    \vspace{-20pt}
    \begin{center}
        \includegraphics[width=.5\textwidth]{1 - Dia 1}
        \captionsetup{justification=centering}
        \caption*{\emph{Dia.\@~1}}
    \end{center}
    \vspace{-21pt}
\end{wrapfigure}

As pedras são colocadas em containeres chamados de potes. Há dois potes em um conjunto. Um pote é para as pedras pretas, e o outro, para as brancas.

A foto na capa mostra um exemplo de tabuleiro de Go com pernas em um ambiente tradicional japonês. O equipamento de Go pode ser obtido em uma grande gama de qualidade, desde conjuntos muito baratos àqueles custando dezenas de milhares de reais. É possível encontrar uma lista de equipamentos visitando o site da Kiseido: \href{https://www.kiseido.com}{\path{kiseido.com}}~\cite{kiseido}. Na verdade, você nem sequer precisa de um conjunto de Go para estudar ou jogar. É possível simplesmente baixar gratuitamente um programa chamado \emph{Go Write 2} que permite jogar e analisar posições (\href{https://www.gowrite.net/GOWrite2_download.html}{\path{gowrite.net/GOWrite2_download.html}})~\cite{gowrite}. Você também pode jogar partidas online sem custos com oponentes do mundo inteiro, através do servidor KGS (\href{https://www.gokgs.com}{\path{gokgs.com}})~\cite{kgs}. A força dos jogadores lá abrange desde iniciantes até profissionais.

O jogo de Go também pode ser jogado em pequenos tabuleiros, sem nenhuma mudança nas regras, e iniciantes são encorajados a jogar suas primeiras partidas em tabuleiros 9\(\times\)9. Já que uma partida 9\(\times\)9 pode ser finalizada em aproximadamente 10 minutos, esta é uma boa maneira de iniciantes se familiarizarem com as regras e as táticas básicas. Você talvez queira, a partir daí, então jogar algumas partidas em um tabuleiro 13$\times$13 antes de progredir para o padrão oficial 19$\times$19.
  \chapter{As Regras}\label{chap:regras}

\begin{itemize}
    \item[\textbf{Regra 1}] O tabuleiro começa vazio.
    \item[\textbf{Regra 2}] Preto sempre começa, e, a partir daí, Branco e Preto se alternam. 
    \item[\textbf{Regra 3}] Uma jogada consiste de colocar uma pedra de sua própria cor em uma intersecção vazia do tabuleiro, contanto que tal jogada seja legal, isto é, não conflite com as outras regras.

    \begin{figure}
        \centering
        \begin{subfigure}{.3\textwidth}
            \centering
            \includegraphics[width=.9\textwidth]{Dia 1}
            \caption{\emph{Dia. 1}}
        \end{subfigure}
        \begin{subfigure}{.3\textwidth}
            \centering
            \includegraphics[width=.9\textwidth]{Dia 2}
            \caption{\emph{Dia. 2}}
        \end{subfigure}
        \begin{subfigure}{.3\textwidth}
            \centering
            \includegraphics[width=.9\textwidth]{Dia 3}
            \caption{\emph{Dia. 3}}
        \end{subfigure}
    \end{figure}

    Os \emph{Diagramas 1 a 3} demonstram uma abertura típica no tabuleiro 9\(\times\)9. Uma vez jogadas, as pedras permanecem onde estão --- a não ser que capturadas, vide a \emph{Regra 5} ---; elas não podem ser fisicamente movidas para outros pontos. Exceto com pouquíssimas exceções causadas pela \emph{Regra 6}, uma jogada é irrestrita, isto é, você pode jogá-la onde quiser.
    \item[\textbf{Regra 4}] Um jogador pode passar o seu turno a qualquer momento.
    
    Passar geralmente ocorre em somente duas situações:
        
    \begin{enumerate}
        \item Perto do fim da partida;
        \item No início de um partida com pedras de compensação (\emph{handicap}).
    \end{enumerate}
    \item[\textbf{Regra 5}] Uma pedra ou um grupo de uma só cor conectado solidamente é capturado e removido do tabuleiro e mantido como prisioneiro quando todas as suas intersecções diretamente adjacentes são ocupadas pelo inimigo. Os 3 diagramas seguintes demonstram como capturas são executadas.
    \item[\textbf{Regra 6}] Um jogador não pode capturar suas próprias pedras. Ou seja, suicídio é ilegal.
\end{itemize}

\emph{Dia. 4.} Branco ocupa 3 de 4 pontos diretamente adjacentes à pedra preta; isto é, 3 de 4 liberdades. Diz-se que a pedra preta está em atari.

\emph{Dia. 5.} Branco 1 captura a pedra preta através da ocupação de sua última liberdade e a remove do tabuleiro.

\emph{Dia. 6.} Este é o resultado. As pedras capturadas são mantidas separadamente, tipicamente na tampa dos potes, que fica virada do avesso durante a partida.

\emph{Dia. 7 a 9} ilustram a captura na borda do tabuleiro.

\emph{Dia. 10 a 12} ilustram a captura  no canto do tabuleiro.

\emph{Dia. 13 a 15} demonstram  como um grupo de duas pedras solidamente conectado é capturado.

Um grupo solidamente conectado de 5 pedras é capturado nos \emph{Dia. 16 a 18}.

De acordo com a \emph{Regra 6}, é ilegal capturar suas próprias pedras. No \emph{Dia. 19}, as pedras brancas estão em atari. Não é possível salvá-las através da conexão em 1 no \emph{Dia. 20} pois se joga em sua própria liberdade, e as 3 pedras são deixadas sem nenhuma liberdade. Uma jogada assim é ilegal. Em outras palavras, ``a auto-captura é ilegal" ou ``o suicídio é ilegal". Entretanto, se for o turno preto, ele pode capturar as duas pedras brancas jogando em 1 no \emph{Dia. 21}.

A captura das pedras adversárias toma precedência sobre a auto-captura. No \emph{Dia 22}, as duas pedras pretas estão sob atari. Quando Branco 1 é jogado em \emph{Dia 23}, nem não há liberdades para tanto para 1 quanto para as duas pedras pretas à direita, mas são as pedras pretas, não as brancas, que são capturadas. O resultado é mostrado no \emph{Dia 24}. Se Preto quiser capturar, imediatamente, em \textbf{A}, ele pode mas não é obrigatório.

\section{30 Problemas de Captura e Resgate de Pedras}

Aqui estão 30 problemas. Depois de pensar sobre eles, você entenderá completamente como capturar pedras assim como, também, resgatar pedras em risco.

\section{Respostas aos Problemas de 1 a 30}

\begin{itemize}
  \item[\textbf{Resposta ao Problema 1}]
      Preto 1 no \emph{Dia. 1} captura um pedra.

      Se Preto joga 1 no \emph{Dia. 2}, Branco pode resgatar sua pedra através da extensão de 2.
  \item[\textbf{Resposta ao Problema 2}]
      Preto 1 no \emph{Dia. 1} captura uma pedra.

      Se Preto conecta em 1 no \emph{Dia. 2}, Branco pode resgatar sua pedra conectando em 2.
  \item[\textbf{Resposta ao Problema 3}]
      Preto 1 no \emph{Dia. 1} captura uma pedra.

      Se Preto conecta em 1 no \emph{Dia. 2}, Branco pode resgatar sua pedra conectando em 2.
  \item[\textbf{Resposta ao Problema 4}]
      Preto 1 no \emph{Dia. 1} captura duas pedras.

      Se Preto joga 1 no \emph{Dia. 2}, Branco pode resgatar suas pedras estendendo em 2.
  \item[\textbf{Resposta ao Problema 5}]
      Preto 1 no \emph{Dia. 1} captura duas pedras.

      Se Preto estende para 1 no \emph{Dia. 2}, Branco pode resgatar suas pedras conectando em 2.
  \item[\textbf{Resposta ao Problema 6}]
      Preto 1 no \emph{Dia. 1} captura as duas pedras marcadas.

      Se Preto estende para 1 no \emph{Dia. 2}, Branco pode resgatar suas duas pedras capturando as duas pedras pretas com 2.
  \item[\textbf{Resposta ao Problema 7}]
      Preto 1 no \emph{Dia. 1} captura duas pedras.

      Se Preto faz atari com 1 em \emph{Dia. 2}, Branco pode resgatar suas pedras e capturar duas do Preto com 2.
  \item[\textbf{Resposta ao Problema 8}]
      Preto 1 no \emph{Dia. 1} captura duas pedras.

      \emph{Dia. 2} mostra o resultado desta captura.
  \item[\textbf{Resposta ao Problema 9}]
      Preto 1 no \emph{Dia. 1} captura três pedras.

      Se Preto conecta em 1 no \emph{Dia. 2}, Branco pode resgatar sua pedra conectando em 2.
  \item[\textbf{Resposta ao Problema 10}]
      Preto 1 no \emph{Dia. 1} captura três pedras.

      Se Preto joga 1 no \emph{Dia. 2}, Branco pode resgatar as pedras em risco com a conexão em 2.
  \item[\textbf{Resposta ao Problema 11}]
      Preto 1 no \emph{Dia. 1} captura 3 pedras.

      Se Preto joga 1 no \emph{Dia. 2}, Branco pode salvar suas pedras e capturar as 4 pretas com 2.
  \item[\textbf{Resposta ao Problema 12}]
      Preto 1 em \emph{Dia. 1} captura uma pedra (crucial).

      Se Preto estende para 1 no \emph{Dia. 2}, Branco pode resgatar sua pedra conectando em 2.
  \item[\textbf{Resposta ao Problema 13}]
      Preto 1 no \emph{Dia. 1} captura cinco pedras.

      Se Preto joga 1 em \emph{Dia. 2} para escapar do atari, Branco pode resgatar suas cinco pedras conectando em 2.
  \item[\textbf{Resposta ao Problema 14}]
      Preto 1 no \emph{Dia. 1} captura cinco pedras.

      Se Preto conecta em 1 no \emph{Dia. 2}, Branco pode resgatar suas cinco pedras capturando quatro pedras com 2.
  \item[\textbf{Resposta ao Problema 15}]
      Preto pode resgatar sua pedra sob atari conectando em 1 no \emph{Dia. 1}.
      
      Se Preto faz atari com 1 no \emph{Dia. 2}, Branco pode capturar com 2.
  \item[\textbf{Resposta ao Problema 16}]
      Preto 1 no \emph{Dia. 1} resgata sua pedra em atari.

      Se Preto faz atari com 1 no \emph{Dia. 2}, Branco pode capturar uma pedra (crucial) com 2.
  \item[\textbf{Resposta ao Problema 17}]
      Preto 1 no \emph{Dia. 1} salva suas duas pedras sob atari.

      Se Preto faz atari com 1 no \emph{Dia. 2}, Branco pode capturar duas pedras com 2.
  \item[\textbf{Resposta ao Problema 18}]
      Preto 1 no \emph{Dia. 1} resgata sua pedra em atari.

      Se Preto faz atari com 1 no \emph{Dia. 2}, Branco pode capturar uma pedra com 2.
  \item[\textbf{Resposta ao Problema 19}]
      Preto 1 no \emph{Dia. 1} resgata suas três pedras em atari.

      Se Preto faz atari com 1 no \emph{Dia. 2}, Branco pode capturar três pedras com 2.
  \item[\textbf{Resposta ao Problema 21}]
      Preto 1 no \emph{1} resgata suas duas pedras em atari.

      Se Preto faz atari com 1 no \emph{Dia. 2}, Branco captura duas pedras com 2.
  \item[\textbf{Resposta ao Problema 22}]
      Branco não consegue escapar. Se ele rastejar na primeira linha com 1 a 10 no \emph{Dia. 1}, Preto captura oito pedras com 12.

      Após Branco 1 no \emph{Dia 2}, se Preto faz atari a partir da direita com 2, Branco pode escapar com 3 e 5.
  \item[\textbf{Resposta ao Problema 23}]
      Preto precisa fazer atari com 1 no \emph{Dia. 1}. Se Branco 2, Preto captura duas pedras com 3.

      Se Preto faz atari a partir da direita com 1 no \emph{Dia. 2}, ele possui somente uma liberdade, então Branco pode capturar três pedras com 2.
  \item[\textbf{Resposta ao Problema 24}]
      Preto precisa fazer atari com 1 no \emph{Dia. 1}. Se Branco 2, Preto faz atari novamente e suas pedras estão seguras. Se Branco \textbf{A}, Preto captura em \textbf{B}.

      Se Preto faz atari pelo topo com 1 em \emph{Dia. 2}, após Branco 2, as pedras marcadas ficam encarceradas.
  \item[\textbf{Resposta ao Problema 25}]
      Preto precisa fazer atari com 1 no \emph{Dia. 1}. Ele pode agora capturar três pedras com \textbf{A}. Se Branco \textbf{A}, suas pedras ainda estão sob atari.

      Se Preto faz atari com 1 no \emph{Dia. 2}, Branco conecta com 2 e suas pedras escapam.
  \item[\textbf{Resposta ao Problema 26}]
      Se Preto faz atari em 1 no \emph{Dia. 1}, ele captura quatro pedras. Se Branco joga 2, ele não possuirá nenhuma maneira de escapar depois de Preto 3.

      Se Preto faz atari com 1 no \emph{Dia. 2}, Branco estende com 2 e suas pedras escapam.
  \item[\textbf{Resposta ao Problema 27}]
      Preto deveria fazer atari com 1 no \emph{Dia. 1}. Preto pode agora capturar três pedras jogando em \textbf{A}. Se Branco \textbf{A}, suas pedras ainda estão em atari.

      Se Preto faz atari com 1 no \emph{Dia. 2}, Branco conecta com 2 e suas pedras escaparam.
  \item[\textbf{Resposta ao Problema 28}]
      As pedras pretas à direita estão em perigo, então ele precisa fazer atari na pedra marcada com 1 no \emph{Dia. 1}.

      Se Preto faz atari vindo debaixo com 1 no \emph{Dia. 2}, Branco faz atari com 2 e pode capturar em seguida com \textbf{A}.
  \item[\textbf{Resposta ao Problema 29}]
      Preto precisa fazer um duplo-atari  nas pedras marcadas com 1 no \emph{Dia. 1}. Ele pode agora capturar uma pedra em \textbf{A} ou \textbf{B}.

      Preto 1 em \emph{Dia 2} faz atari em somente uma pedra branca. Branco conecta com 2 e Preto não pode capturar nada.
  \item[\textbf{Resposta ao Problema 30}]
      Preto 1 no \emph{Dia. 1} é um duplo-atari, então Branco pode capturar uma das pedras marcadas em \textbf{A} ou \textbf{B}.
      
      Preto 1 no \emph{Dia. 2} faz atari em somente uma pedra. Branco conecta com 2 e Preto não pode capturar nada.
\end{itemize}
  \chapter{A Regra do Ko}

\begin{itemize}
  \item[\textbf{Regra 7}] Nenhuma posição de tabuleiro pode ser recriada.
\end{itemize}

Esta regra requer que todo movimente cria uma nova posição no tabuleiro. Sua principal função é prevenir ciclos infinitos de captura e recaptura em posições de ko como a exibida no \emph{Dia.\@~25}.

% TODO: Adicionar a origem do ko?

Na posição do \emph{Dia.\@~25}, suponha que seja o turno branco. Ele pode capturar uma pedra preta jogando em 1 no \emph{Dia.\@~26}. O resultado é mostrado no \emph{Dia.\@~27}.

Se Preto agora captura com 2 no \emph{Dia.\@~28}\ldots

A posição do \emph{Dia.\@~29} é o resultado. Mas essa é a mesma posição do \emph{Dia.\@~25}. Pela \emph{Regra 7}, isso não é permitido, então Preto precisa jogar em outro lugar. Por exemplo, ele poderia jogar 2 no \emph{Dia.\@~30}. Isso oferece a Branco a oportunidade de conectar em 3, resolvendo o ko.

No \emph{Dia.\@~31}, é o turno Branco a jogar. Uma luta de ko está prestes a acontecer no entorno da pedra preta marcada, que está em atari.

Se Branco captura com 1 no \emph{Dia.\@~32}, Preto não pode imediatamente recapturar porque isso reverteria a posição de volta para o \emph{Dia.\@~31}, então ele jogará em outro lugar com Preto 2. Esse tipo de movimento é chamado de ameaça de ko.

Se Branco responde a essa ameaça com 3 no \emph{Dia.\@~33}, Preto pode recapturar com 4, pois a troca de Preto 2--Branco 3 faz com que a posição global do tabuleiro seja diferente do \emph{Dia.\@~31}. É agora a vez do Branco de encontrar uma ameaça de ko.

Branco corta com 5 no \emph{Dia.\@~34}. Preto poderia ignorar Branco 5 e capturar três pedras em \textbf{A}, e assim  resolver o ko, mas vamos supor qu eele responderia Branco 5 com 6. Branco pode agora recapturar o ko com 7. Preto precisa fazer outra ameaça de ko com 8. Talvez Branco ignorará essa ameaça e capturará quatro pedras com 9, finalizando a briga pelo ko. Preto obtém certa compensação com 10.
  \chapter{O Fim do Jogo}

\begin{itemize}
    \item[\textbf{Regra 8}] Duas passagens de turno consecutivas finalizam a partida. (Ou um dos jogadores pode desistir, e é possível desistir a qualquer momento, mesmo após um passe adversário.)
    \item[\textbf{Regra 9}] Ao final da partida, toda pedra que não conseguir se salvar é removida do tabuleiro como prisioneira do adversário.
    \item[\textbf{Regra 10}] A pontuação de um jogador é o número de intersecções vazias que ele cercou \emph{menos} o número de prisioneiros que ele perdeu para o adversário. A pontuação mais alta vence --- note que a diferença de pontos é irrelevante para o status final da partida.
\end{itemize}

\pagebreak

A partida no \emph{Dia.\@~35} está quase finalizada. Ambos Preto e Branco asseguraram seus respectivos territórios. Durante essa partida, Branco capturou diversos prisioneiros, mais precisamente, sete; e Preto capturou três pedras brancas.

\begin{figure}[h!]
    \centering
    \begin{subfigure}[t]{.3\textwidth}
        \centering
        \captionsetup{justification=raggedright,singlelinecheck=false,margin={.05in,.05in}}
        \includegraphics[width=1\textwidth]{4 - Dia 35}
        \caption*{\emph{Dia.\@~35. A partida está quase finalizada}}
    \end{subfigure}
    \hfill
    \begin{subfigure}[t]{.3\textwidth}
        \centering
        \captionsetup{justification=raggedright,singlelinecheck=false,margin={.05in,.05in}}
        \includegraphics[width=1\textwidth]{4 - Dia 36}
        \caption*{\emph{Dia.\@~36. Preenchendo os pontos neutros}}
    \end{subfigure}
    \hfill
    \begin{subfigure}[t]{.3\textwidth}
        \centering
        \captionsetup{justification=raggedright,singlelinecheck=false,margin={.05in,.05in}}
        \includegraphics[width=1\textwidth]{4 - Dia 37}
        \caption*{\emph{Dia.\@~37. Removendo a pedra morta}}
    \end{subfigure}
    \vspace*{.5cm}
    \captionsetup{justification=centering}
    \caption*{\emph{Como capturas, Preto possui três pedras; e Branco, sete.}}
\end{figure}
 
Preto 1 no \emph{Dia.\@~36} não conquista nenhum ponto, mas ameaça a captura de uma pedra branca e o início de um ko, então Branco conecta com 2. Preto 3 e Branco 4 não ganham pontos também. Há pontos neutros que são jogados ao final da partida, somente para mais claramente delimitar os territórios. Não há mais pontos a serem disputados, então Preto 5 e Branco 6 são passagens de turno. De acordo com a \emph{Regra 8}, isso finaliza a partida.

A pedra branca marcada no \emph{Dia.\@~36} está morta: ela não consegue evitar de ser capturada. Portanto, de acordo com a \emph{Regra 9}, ela é removida do tabuleiro como prisioneira. O resultado dessa operação é visualizado no \emph{Dia.\@~37}. Note que Preto possui quatro prisioneiros brancos.

Contemos a pontuação.

\pagebreak

\begin{figure}[h!]
    \centering
    \begin{subfigure}[t]{.3\textwidth}
        \centering
        \captionsetup{justification=raggedright,singlelinecheck=false,margin={.05in,.05in}}
        \includegraphics[width=1\textwidth]{4 - Dia 38}
        \caption*{\emph{Dia.\@~38. O território preto}}
    \end{subfigure}
    \hfill
    \begin{subfigure}[t]{.3\textwidth}
        \centering
        \captionsetup{justification=raggedright,singlelinecheck=false,margin={.05in,.05in}}
        \includegraphics[width=1\textwidth]{4 - Dia 39}
        \caption*{\emph{Dia.\@~39. O território branco}}
    \end{subfigure}
    \hfill
    \begin{subfigure}[t]{.3\textwidth}
        \centering
        \captionsetup{justification=raggedright,singlelinecheck=false,margin={.05in,.05in}}
        \includegraphics[width=1\textwidth]{4 - Dia 40}
        \caption*{\emph{Dia.\@~40. O território é preenchido com prisioneiros}}
    \end{subfigure}
    \vspace*{.5cm}
    \captionsetup{justification=centering}
    \caption*{\emph{Como capturas, Preto possui três pedras; e Branco, sete.}}
\end{figure}

Preto cercou os pontos vazios marcados com $\times$ no \emph{Dia.\@~38}, e Branco cercou os pontos vazios marcados com $\times$ no \emph{Dia.\@~39}. Esses pontos vazios cercados são denominados de \emph{território}.

Para subtrair prisioneiros do território, os prisioneiros são colocados no tabuleiro dentro dos territórios pretos e brancos, como mostrado pelas pedras marcadas no \emph{Dia.\@~40}. Conte os territórios para cada lado e verá que Branco possui 6 pontos enquanto que Preto possui 5 pontos. Branco vence por um ponto.

\bigskip

Essas são as dez regras necessárias para se jogar Go. Há, porém, algumas poucas posições especiais e kos que são determinados por adjudicação. Essas posições são conhecidas como \emph{Precedentes da Nihon Kiin}, mas elas ocorrem muito raramente em partidas.

As regras aqui apresentadas são as japonesas. Há outras também, a mais importante dentre essas outras sendo as regras chinesas. A maioria dos jogadores no ocidente utiliza as regras japonesas. Porém, se você for jogar Go na China, terá de aprender, claro sobre as regras chinesas. A principal diferença é o procedimento de contagem ao final da partida, mas, para basicamente quase todos os fins práticos, ambas são equivalentes. Isto é, elas não mudam a natureza do jogo. Para uma exposição dos diversos conjuntos de regras, assim como os Precedentes da Nihon Kiin, veja o livro \emph{The Go Player's Almanac}~\cite{bozulich_almanac}.
  \chapter{Partidas-Exemplo}\label{chap:cinco}

\section{Partida-Exemplo em um Tabuleiro \texorpdfstring{6$\times$6}{6x6}}

A partida a seguir ilustra jogadas perfeitas em um tabuleiro 6$\times$6, o menor tamanho em que Go é interessante de ser jogado.

\emph{Dia.\@~1}. Isso pode ser chamado de abertura. Preto começa por tentar controlar o lado direito com 1 e 3; e Branco, o lado esquerdo com 2 e 4. Preto se dobra em volta de Branco com 5 e 7, e Branco resiste com 6 e 8.

\begin{figure}[h!]
  \centering
  \begin{subfigure}[t]{.3\textwidth}
    \centering
    \includegraphics[width=1\textwidth]{5 - Dia 1}
    \captionsetup{justification=centering}
    \caption*{\emph{Dia.\@~1. (1-8)}}
  \end{subfigure}
  \hfill
  \begin{subfigure}[t]{.3\textwidth}
    \centering
    \includegraphics[width=1\textwidth]{5 - Dia 2}
    \captionsetup{justification=centering}
    \caption*{\emph{Dia.\@~2. (9)}}
  \end{subfigure}
  \hfill
  \begin{subfigure}[t]{.3\textwidth}
    \centering
    \includegraphics[width=1\textwidth]{5 - Dia 3}
    \captionsetup{justification=centering}
    \caption*{\emph{Dia.\@~3. (10-11)}}
  \end{subfigure}
\end{figure}

\emph{Dia.\@~2}. Preto pressiona sua vantagem através do corte em 9. Isso põe as duas pedras marcadas em atari --- as pedras pretas estão ocupando todas as liberdades brancas exceto uma. Note que as pedras brancas marcadas não estão diretamente conectadas às outras pedras brancas.

\emph{Dia.\@~3}. Branco resgata suas duas pedras pela conexão com 10, e a pedra preta marcada agora está em atari. Preto ignora isso e joga 11, colocando a pedra branca em atari.

\pagebreak

\emph{Dia.\@~4}. Branco escapa do atari descendo para 12. Preto conecta com 13 para prevenir que Branco o corte ali. A pedra preta marcada ainda está sob atari. Entretanto, ela não pode escapar, então Branco não se apressa em capturá-la e corta em 14, colocando outra pedra preta em atari.

\begin{figure}[h!]
  \centering
  \begin{subfigure}[t]{.3\textwidth}
    \centering
    \includegraphics[width=1\textwidth]{5 - Dia 4}
    \captionsetup{justification=centering}
    \caption*{\emph{Dia.\@~4. (12-14)}}
  \end{subfigure}
  \hfill
  \begin{subfigure}[t]{.3\textwidth}
    \centering
    \includegraphics[width=1\textwidth]{5 - Dia 5}
    \captionsetup{justification=centering}
    \caption*{\emph{Dia.\@~5. (15-18)}}
  \end{subfigure}
  \hfill
  \begin{subfigure}[t]{.3\textwidth}
    \centering
    \includegraphics[width=1\textwidth]{5 - Dia 6}
    \captionsetup{justification=centering}
    \caption*{\emph{Dia.\@~6. (19-21)}}
  \end{subfigure}
\end{figure}

\emph{Dia.\@~5}. Preto joga um atari com 15 e Branco captura com 16, tomando um prisioneiro. Preto agora desce para 17, ameaçando jogar em 18 e desprivilegiar Branco de um ponto. Sendo assim, Branco precisa conectar em 18. Esse é o último movimento da partida, valendo um ponto.

\emph{Dia.\@~6}. Preto conecta com 19, preparando para tomar o último ponto neutro em 21. Branco não pode jogar em 21, então ele captura com 20, tomando outro prisioneiro.

\pagebreak

\emph{Dia.\@~7}. Após Preto 21 no \emph{Dia.\@~6}, não há mais pontos a serem ganhos ou disputados, então Branco passa. Preto também passa. De acordo com a \emph{Regra 8}, o jogo termina.

\begin{figure}[h!]
  \centering
  \begin{subfigure}[t]{.3\textwidth}
    \centering
    \includegraphics[width=1\textwidth]{5 - Dia 7}
    \captionsetup{justification=raggedright,singlelinecheck=false,margin={.20in,.05in}}
    \caption*{\emph{Dia.\@~7. Partida finalizada}}
  \end{subfigure}
  \hspace{1cm}
  \begin{subfigure}[t]{.3\textwidth}
    \centering
    \includegraphics[width=1\textwidth]{5 - Dia 8}
    \captionsetup{justification=centering}
    \caption*{\emph{Dia.\@~8}}
  \end{subfigure}
\end{figure}

\emph{Dia.\@~8}. Branco possui dois prisioneiros, então ele os coloca no território preto --- as duas pedras marcadas. Os territórios são agora contados. Branco possui 6 pontos, e Preto, 9.

\textbf{Preto vence por 3 pontos.}

\pagebreak

\section{Perguntas e Respostas}\label{section:5.2:seki}

\begin{itemize}
  \item[\textbf{Pergunta}]
    Ao invés de 19 no \emph{Dia.\@~6}, será que Preto não poderia jogar atari em 21?
  \item[\textbf{Resposta}]
    Se Preto fizer atari imediatamente com 1 no \emph{Dia.\@~9}, ele se coloca em atari e perde duas pedras após a captura Branca com 2.

    \begin{figure}[h!]
      \centering
      \begin{subfigure}[t]{.3\textwidth}
        \centering
        \includegraphics[width=.9\textwidth]{5 - Dia 9}
        \captionsetup{justification=centering}
        \caption*{\emph{Dia.\@~9}}
      \end{subfigure}
      \hspace{1cm}
      \begin{subfigure}[t]{.3\textwidth}
        \centering
        \includegraphics[width=.9\textwidth]{5 - Dia 10}
        \captionsetup{justification=centering}
        \caption*{\emph{Dia.\@~10}}
      \end{subfigure}
    \end{figure}

  \item[\textbf{Pergunta}]
    Ao invés de capturar com 20 no \emph{Dia.\@~6}, será que Branco não poderia jogar no último ponto neutro em 21?
  \item[\textbf{Resposta}]
    Se Branco jogar imediatamente em 1 no \emph{Dia.\@~10}, ele coloca três de suas pedras em atari, e Preto procederia com a captura em 2.
  \item[\textbf{Pergunta}]
    Por que Branco passou no \emph{Dia.\@~7}? Por que ele não tenta invadir o território com 1 no \emph{Dia.\@~11}?

  \item[\textbf{Resposta}]
    Não há regra que previna Branco de invadir. Mas ele percebe que isso seria de nenhuma utilidade. Preto responderia com 2 no \emph{Dia.\@~12}, colocando a pedra invasora em atari, e, quando Preto captura com 6, Branco perderia sua força invasora completamente.

    \begin{figure}[h!]
      \centering
      \begin{subfigure}[t]{.3\textwidth}
        \centering
        \includegraphics[width=.9\textwidth]{5 - Dia 11}
        \captionsetup{justification=centering}
        \caption*{\emph{Dia.\@~11}}
      \end{subfigure}
      \hspace{1cm}
      \begin{subfigure}[t]{.3\textwidth}
        \centering
        \includegraphics[width=.9\textwidth]{5 - Dia 12}
        \captionsetup{justification=centering}
        \caption*{\emph{Dia.\@~12}}
      \end{subfigure}
    \end{figure}

  \item[\textbf{Pergunta}]
    Se Branco continuar insistindo na invasão, será que ele não conseguirá, no final, ocupar todos os pontos dentro do grupo Preto e, assim, capturá-lo?

    \begin{figure}[h!]
      \centering
      \begin{subfigure}[t]{.3\textwidth}
        \centering
        \includegraphics[width=.9\textwidth]{5 - Dia 13}
        \captionsetup{justification=centering}
        \caption*{\emph{Dia.\@~13}}
      \end{subfigure}
      \hfill
      \begin{subfigure}[t]{.3\textwidth}
        \centering
        \includegraphics[width=.9\textwidth]{5 - Dia 14}
        \captionsetup{justification=centering}
        \caption*{\emph{Dia.\@~14}}
      \end{subfigure}
      \hfill
      \begin{subfigure}[t]{.3\textwidth}
        \centering
        \includegraphics[width=.9\textwidth]{5 - Dia 15}
        \captionsetup{justification=centering}
        \caption*{\emph{Dia.\@~15}}
      \end{subfigure}
    \end{figure}

  \item[\textbf{Resposta}]
    Ele é bem-vindo a tentar, como no \emph{Dia.\@~13}, mas ele não será bem-sucedido. Após Branco 7 e 9, Preto esmaga aquelas duas pedras com 10 --- Branco não pode conectar no ponto 1-1, já que seria suicídio, um movimento ilegal. Branco põe o grupo preto de onze pedras em atari com 11 no \emph{Dia.\@~14} e Preto captura duas pedras com 14. No final, Branco esgota os possíveis pontos de invasão. A posição agora aparenta ser \emph{Dia.\@~15} e Preto ainda vence por 3 pontos. Lembre-se que as pedras brancas voltarão para o território Branco quando a partida for contabilizada. Branco pode tentar novamente em \textbf{A}, mas Preto capturará imediatamente com \textbf{B}, e vice-versa.

    O que faz com que o grupo preto inteiro no \emph{Dia.\@~15} seja invulnerável à captura são os vários buracos, ou olhos, que ele possui. Branco pode jogar somente uma pedra por vez, portanto ele jamais conseguirá preencher todos esses olhos simultaneamente, sem se suicidar, como ele deveria, para capturar o grupo preto.

  \pagebreak

  \item[\textbf{Pergunta}]
    Quantos ``buracos''  ou ``olhos''  são necessários para que um grupo esteja seguro?
  \item[\textbf{Resposta}]
    Ele precisa de dois, pelo menos. \emph{Dia.\@~16} mostra dois exemplos. Os dois grupos pretos estão vivos com dois olhos cada um. Na verdade, Branco não possui um movimento legal para atacar também. Branco, com seus dois grandes espaços de olhos, também está vivo. Preto pode jogar dentro do grupo branco, mas Branco pode facilmente capturar os invasores.

    \begin{figure}[h!]
      \centering
      \begin{subfigure}[t]{.3\textwidth}
        \centering
        \includegraphics[width=.9\textwidth]{5 - Dia 16}
        \captionsetup{justification=centering}
        \caption*{\emph{Dia.\@~16}}
      \end{subfigure}
      \hfill
      \begin{subfigure}[t]{.3\textwidth}
        \centering
        \includegraphics[width=.9\textwidth]{5 - Dia 17}
        \captionsetup{justification=centering}
        \caption*{\emph{Dia.\@~17}}
      \end{subfigure}
      \hfill
      \begin{subfigure}[t]{.3\textwidth}
        \centering
        \includegraphics[width=.9\textwidth]{5 - Dia 18}
        \caption*{\emph{Dia.\@~18. Branco 3 em 1}}
      \end{subfigure}
    \end{figure}

    Você deveria também notar que os olhos precisam estar separados. O grupo preto no canto inferior esquerdo do \emph{Dia.\@~17} não está vivo, mas morto. Ele não consegue evitar de ser capturado. Talvez possa parecer que ele possui dois olhos, mas estes, neste caso, não são separáveis ou distintos. Branco 1 no \emph{Dia.\@~18} coloca Preto em atari. Preto pode capturar com 2, mas Branco só joga novamente 3 em 1, privando o grupo preto de sua última liberdade e, assim, capturando todas as cinco pedras pretas.

    Outra coisa sobre a qual se atentar é olho falso, como o que o grupo preto possui no canto superior direito do \emph{Dia.\@~18} em \textbf{A}. Esse grupo também está morto. Branco \textbf{A} captura as três pedras marcadas e põe as outras duas pedras sob atari.

  \pagebreak

  \item[\textbf{Pergunta}]
    Um grupo sempre precisa estar apto a criar dois olhos para estar vivo?

    \begin{figure}[h!]
      \centering
      \begin{subfigure}[t]{.4\textwidth}
        \centering
        \includegraphics[width=1\textwidth]{5 - Dia 19}
        \captionsetup{justification=centering}
        \caption*{\emph{Dia.\@~19}}
      \end{subfigure}
    \end{figure}
  \item[\textbf{Resposta}]
    Geralmente, mas há exceções. No \emph{Dia.\@~19}, o grupo preto marcado e o grupo branco não possuem nenhum olho, mas ambos compartilham liberdades. Nenhum dos lados pode atacar o outro jogando \textbf{A} ou \textbf{B} sem se colocar em atari, então nenhum dos lados deveria jogar nessa região, ou seja, ambos os grupos estão vivos. Este tipo de impasse local é chamado de \emph{seki}. Os pontos vazios entre as pedras marcadas não contam como território para nenhum dos lados.

    Há outro seki no canto inferior esquerdo do \emph{Dia.\@~19}. O grupo preto de três pedras e o grupo branco cercando-o ambos possuem um olho só, e nenhum deles pode ocupar o ponto \textbf{C} entre eles sem se colocar em atari. (O ponto \textbf{C} também não é território.)
\end{itemize}

\pagebreak

\section{Partida-Exemplo em um Tabuleiro \texorpdfstring{9$\times$9}{9x9}}

\emph{Dia.\@~1}. A boa estratégia dita que movimentos de abertura sejam feitos na terceira linha ou acima, em relação às bordas do tabuleiro. Neste caso, no entanto, Preto 1 e Branco 2 são jogados nas quartas linhas, um dos motivos sendo as peculiaridades do tabuleiro 9$\times$9.

\begin{figure}[h!]
  \centering
  \begin{subfigure}[t]{.3\textwidth}
    \centering
    \includegraphics[width=1.05\textwidth]{5 - Exemplo - Dia 1}
    \captionsetup{justification=centering}
    \caption*{\emph{Dia.\@~1. (1-8)}}
  \end{subfigure}
  \hfill
  \begin{subfigure}[t]{.3\textwidth}
    \centering
    \includegraphics[width=1.05\textwidth]{5 - Exemplo - Dia 2}
    \captionsetup{justification=centering}
    \caption*{\emph{Dia.\@~2. (9-16)}}
  \end{subfigure}
  \hfill
  \begin{subfigure}[t]{.3\textwidth}
    \centering
    \includegraphics[width=1.05\textwidth]{5 - Exemplo - Dia 3}
    \captionsetup{justification=centering}
    \caption*{\emph{Dia.\@~3. (17-23)}}
  \end{subfigure}
\end{figure}

\emph{Dia.\@~2}. Preto 11 faz atari na pedra branca, então Branco conecta em 12. Tendo construído um grupo vivo à direita, Branco invade o lado esquerdo com 14 e inicia o estabelecimento de outro grupo vivo à esquerda com 16.

\emph{Dia.\@~3}. Branco defende seu grupo no lado esquerdo com 18 e 20, e, então, seu grupo à direita com 22. Preto pula para o lado direito com 23 e o fim de jogo se inicia. Pulos de um espaço como 19 e 23 são frequentemente boas jogadas.

\pagebreak

\emph{Dia.\@~4}. Branco detém a intrusão preta ao lado esquerdo com 24 e 26. As jogadas começam, a partir de agora, a focar nas bordas do tabuleiro.

\begin{figure}[h!]
  \centering
  \begin{subfigure}[t]{.3\textwidth}
    \centering
    \includegraphics[width=1.05\textwidth]{5 - Exemplo - Dia 4}
    \captionsetup{justification=centering}
    \caption*{\emph{Dia.\@~4. (24-34)}}
  \end{subfigure}
  \hspace{1cm}
  \begin{subfigure}[t]{.3\textwidth}
    \centering
    \includegraphics[width=1.05\textwidth]{5 - Exemplo - Dia 5}
    \captionsetup{justification=centering}
    \caption*{\emph{Dia.\@~5. (35-41)}}
  \end{subfigure}
\end{figure}

\emph{Dia.\@~5}. O fim de jogo continua com Preto 35 até o atari de Branco 40.

\emph{Dia.\@~6}. Branco 50 é um sacrifício que, mais tarde, forçará Preto a conectar em 55. No final da partida, Branco perdeu um prisioneiro --- a pedra em 50 --- e Preto 53 é uma pedra morta.

\begin{figure}[h!]
  \centering
  \begin{subfigure}[t]{.28\textwidth}
    \centering
    \includegraphics[width=1.0\textwidth]{5 - Exemplo - Dia 6}
    \captionsetup{justification=centering}
    \caption*{\emph{Dia.\@~6. (42-57)}}
  \end{subfigure}
  \hfill
  \begin{subfigure}[t]{.28\textwidth}
    \centering
    \includegraphics[width=1.0\textwidth]{5 - Exemplo - Dia 7}
    \captionsetup{justification=centering}
    \caption*{\emph{Dia.\@~7. Prisioneiros}}
  \end{subfigure}
  \hfill
  \begin{subfigure}[t]{.28\textwidth}
    \centering
    \includegraphics[width=1.0\textwidth]{5 - Exemplo - Dia 8}
    \captionsetup{justification=centering}
    \caption*{\emph{Dia.\@~8. Rearranjo}}
  \end{subfigure}
\end{figure}

\emph{Dia.\@~7}. O prisioneiro branco é substituído dentro do território branco, e a pedra preta morta --- 53 no \emph{Dia.\@~6} --- é removida e preenche o interior do território preto. Essas pedras são indicadas pelas pedras marcadas.

\emph{Dia.\@~8}. É costumeiro rearranjar os territórios durante a contagem, para torná-lo mais facilmente reconhecível em termos de múltiplos de 5 ou de 10. Nesta partida, Preto possui 11 pontos de território, e Branco, 13, portanto, Branco vence por dois pontos.
  \chapter{Estratégia de Abertura}\label{chap:6:estrat_abertura}

Para se tornar um jogador forte de Go, é necessário desenvolver, imperativamente, duas habilidades:

\vspace{.25cm}

\begin{enumerate}[leftmargin=1.5cm,rightmargin=1cm]
    \item \textbf{leitura} à frente do tabuleiro atual, movimento por movimento, e previsão de resultados de embates locais;
    \item \textbf{intuição} sobre o que está acontecendo no tabuleiro como um todo, isto é, avaliação global.
\end{enumerate}

\vspace{.25cm}

O balanço aproximadamente igualitário entre qualidades intuitivas e analíticas é grande parte da atratividade do Go. Na abertura, quando o tabuleiro consiste de majoritariamente espaço vazio, é intuição, sedimentada em uma base de dados de conhecimento geral, que toma um papel dominante.

\begin{wrapfigure}{r}{60mm}
    \vspace{-30pt}
    \begin{center}
        \includegraphics[width=.5\textwidth]{6 - Dia 1}
        \captionsetup{justification=centering}
        \caption*{\emph{Dia.\@~1. (1-5)}}
    \end{center}
    \vspace{-20pt}
\end{wrapfigure}

No tabuleiro 19$\times$19, o tamanho oficial, é difícil de se assegurar território no início, então a partida usualmente começa com os jogadores espaçando suas pedras para formar grandes armações dentro das quais eles poderão brigar vantajosamente no futuro.

\emph{Dias.\@~1 a 3} mostram uma abertura típica no tabuleiro 19$\times$19.

É geralmente muito mais fácil de se estabelecer bases nos cantos, como Preto e Branco o fazem com 1 a 4 no \emph{Dia.\@~1}. Uma ou duas pedras por canto é suficiente. Com 5, Preto estabelece um enclausuro de canto. Esse movimento delimita e vigia o território no canto. Não é ainda território seguro ou consolidado, uma vez que Branco possui múltiplas maneiras (condicionais) de invadi-lo. Mas Preto terá a vantagem em qualquer luta que se irromper por ali. O tempo para a invasão branca será, assim, um fator crítico.

Aproximar-se do canto com Branco 6 no \emph{Dia.\@~2}, onde Preto possui somente uma pedra, é uma boa jogada. Isso frequentemente provoca lutas, como o curto conflito que se segue. Nos movimentos de 7 a 12, Preto assegura  o canto enquanto Branco constrói uma posição à direita. Essa sequência é um dos padrões-referência conhecidos como \emph{josekis}.

\begin{figure}[h!]
    \centering
    \begin{subfigure}[t]{.495\textwidth}
        \centering
        \includegraphics[width=\textwidth]{6 - Dia 2}
        \captionsetup{justification=centering}
        \caption*{\emph{Dia.\@~2. (6-12)}}
    \end{subfigure}
    \hfill
    \begin{subfigure}[t]{.495\textwidth}
        \centering
        \includegraphics[width=\textwidth]{6 - Dia 3}
        \captionsetup{justification=centering}
        \caption*{\emph{Dia.\@~3. (13-17)}}
    \end{subfigure}
\end{figure}

Preto 13 no \emph{Dia.\@~3} é outro exemplo de outra aproximação. Branco 14 forma uma armação esparsa no quadrante inferior esquerdo do tabuleiro a partir da pedra branca marcada. A terceira ou a quarta linha são as melhores para extensões como esta. Preto desenvolve uma armação no topo com 15 e 17. Note Preto 17 na quarta linha, e Preto 15 e a pedra preta marcada na terceira linha. Isso constitui um balanço ideal de jogadas altas e baixas. As pedras na terceira linha defendem os flancos da posição preta enquanto que a pedra na quarta linha expande seu território para o centro.

\pagebreak

\section{A Primeira Prioridade: Presença no Canto}

A abertura de uma partida de Go geralmente se inicia com ambos os lados estabelecendo presenças nos cantos. Os pontos a seguir são os cinco de referência que um jogador usualmente ocupa com seus primeiros movimentos.

\emph{Dia.\@~1. O ponto 3-3.} O intuito de Preto 1 no ponto 3-3 --- também conhecido como \emph{san-san} --- é assegurar o território no canto, apesar de que esse movimento não fornece muita influência no centro.

\begin{figure}[h!]
    \centering
    \begin{subfigure}[t]{.3\textwidth}
        \centering
        \includegraphics{6 - Corner - Dia 1}
        \captionsetup{justification=centering}
        \caption*{\emph{Dia.\@~1. O ponto 3-3}}
    \end{subfigure}
    \hfill
    \begin{subfigure}[t]{.3\textwidth}
        \centering
        \includegraphics{6 - Corner - Dia 2}
        \captionsetup{justification=centering}
        \caption*{\emph{Dia.\@~2. O ponto-estrela}}
    \end{subfigure}
    \hfill
    \begin{subfigure}[t]{.3\textwidth}
        \centering
        \includegraphics{6 - Corner - Dia 3}
        \captionsetup{justification=centering}
        \caption*{\emph{Dia.\@~3. O ponto 3-4}}
    \end{subfigure}
\end{figure}

\emph{Dia.\@~2. O ponto-estrela.} Por ter jogado no ponto 4-4 --- também conhecido como ponto-estrela --- com 1, Preto almeja influência no centro.

\emph{Dia.\@~3. O ponto 3-4.} Quando Preto joga 1 no ponto 3-4 (\emph{komoku}), ele espera ganhar território ao longo do lado direito assim como algo no canto.

\emph{Dia.\@~4. O ponto 5-3.} Preto 1 no ponto 5-3 (\emph{mokuhazushi}) enfatiza o lado. Preto está disposto a conceder a maior parte do canto para Branco.


\begin{figure}[h!]
    \centering
    \begin{subfigure}[t]{.3\textwidth}
        \centering
        \includegraphics{6 - Corner - Dia 4}
        \captionsetup{justification=centering}
        \caption*{\emph{Dia.\@~4. O ponto 5-3}}
    \end{subfigure}
    \hspace{1cm}
    \begin{subfigure}[t]{.3\textwidth}
        \centering
        \includegraphics{6 - Corner - Dia 5}
        \captionsetup{justification=centering}
        \caption*{\emph{Dia.\@~5. O ponto 5-4}}
    \end{subfigure}
\end{figure}

\emph{Dia.\@~5. O ponto 5-4.} Preto 1 no ponto 5-4 (\emph{takamoku}) concede o canto ao Branco. Ele almeja influência no centro e ao longo dos lados.

\pagebreak

\section{Movimentos de Aproximação}

Depois de os jogadores terem ocupado os cantos, é praxe atacá-los ou desafiá-los. Tais movimentos são conhecidos como movimentos de aproximação.

Uma pedra no 3-3 vai geralmente ser atacada por um movimento no ponto 4-4 --- Branco 1 no \emph{Dia.\@~6}. Com movimentos até Branco 7, Preto assegura o território no canto, mas Branco ganha influência no centro. Esta sequência é um padrão estabelecido como joseki. Seria uma boa ideia memorizar este padrão básico e outros conforme continuamos, uma vez que eles são alguns dos mais comuns que você encontrará.


\begin{figure}[h!]
    \centering
    \begin{subfigure}[t]{.3\textwidth}
        \centering
        \includegraphics[width=.85\textwidth]{6 - Approach - Dia 6}
        \captionsetup{justification=centering}
        \caption*{\emph{Dia.\@~6. Um joseki do ponto 3-\@3}}
    \end{subfigure}
    \hfill
    \begin{subfigure}[t]{.3\textwidth}
        \centering
        \includegraphics[width=.85\textwidth]{6 - Approach - Dia 7}
        \captionsetup{justification=centering}
        \caption*{\emph{Dia.\@~7. Um joseki do ponto 4-\@4}}
    \end{subfigure}
    \hfill
    \begin{subfigure}[t]{.3\textwidth}
        \centering
        \includegraphics[width=.85\textwidth]{6 - Approach - Dia 8}
        \captionsetup{justification=centering}
        \caption*{\emph{Dia.\@~8. Um joseki do ponto 4-\@4}}
    \end{subfigure}
\end{figure}

Contra uma pedra jogada no ponto 4-4, é usual se aproximar com um movimento do cavaleiro --- similar ao do cavalo no xadrez --- de Branco 1 no \emph{Dia.\@~7}. Preto 2 é a resposta típica, mas Preto 2 em \textbf{A} ou \textbf{B} também são frequentemente jogados. Branco agora continua deslizando para 3, Preto defende o canto com 4, e Branco estende dois espaços até 5. Isso também é um joseki básico.

Ao invés de 1 no \emph{Dia.\@~7}, Branco poderia também invadir o ponto 3-3 com 1 no \emph{Dia.\@~8}. A sequência até Preto 12 é outro joseki básico. Branco consegue um território blindado no canto, mas Preto obtém uma posição muito densa no exterior. Se não houver muitas pedras no tabuleiro, a densidade preta é julgada como melhor do que o território branco, portanto não é aconselhável que Branco jogue essa sequência até que o meio de jogo comece. Esse joseki enfatiza a principal fraqueza da pedra situada no ponto 4-4: ela deixa o canto aberto para invasões. No entanto, um jogador que queira prosseguir com uma estratégia que enfatize influência central vai geralmente jogar uma pedra ou no ponto 4-4 ou no ponto 5-4.

\pagebreak

Contra uma pedra no ponto 3-4, há duas aproximações básicas: o movimento (curto) do cavaleiro em \textbf{A} no \emph{Dia.\@~9} e a aproximação de um espaço em \textbf{B}.

\begin{figure}[h!]
    \centering
    \begin{subfigure}[t]{.3\textwidth}
        \centering
        \includegraphics[width=\textwidth]{6 - Approach - Dia 9}
        \captionsetup{justification=centering}
        \caption*{\emph{Dia.\@~9}}
    \end{subfigure}
    \hfill
    \begin{subfigure}[t]{.3\textwidth}
        \centering
        \includegraphics[width=\textwidth]{6 - Approach - Dia 10}
        \captionsetup{justification=centering}
        \caption*{\emph{Dia.\@~10}}
    \end{subfigure}
    \hfill
    \begin{subfigure}[t]{.3\textwidth}
        \centering
        \includegraphics[width=\textwidth]{6 - Approach - Dia 11}
        \captionsetup{justification=centering}
        \caption*{\emph{Dia.\@~11}}
    \end{subfigure}
\end{figure}

Contra a aproximação curta do cavaleiro de Branco 1 no \emph{Dia.\@~10}, Preto 2 é uma resposta sólida. Branco pode responder estendendo até 3 ou mesmo tão distantemente quanto \textbf{A}.

Se Branco joga a aproximação de um espaço de 1 no \emph{Dia.\@~11}, contatar-se com Preto 2 é uma resposta popular. A sequência até Branco 7 é um joseki básico. Preto assegura o canto e Branco estabelece uma posição no topo. Preto poderia também jogar 6 em \textbf{A}, e Branco poderia jogar 7 em \textbf{B}. Compare este joseki a movimentos como 6 a 12 no \emph{Dia.\@~2} no início desta seção sobre estratégia de abertura.

\pagebreak

Quando Preto possui uma pedra no ponto 5-3 (a pedra marcada), Branco 1 no \emph{Dia.\@~12} é a aproximação padrão. Preto responde achatando Branco com 2 e 4. Branco assegura o território à direita com 3 e 5 e Preto estende a partir de sua parede com 6, estabelecendo uma posição no topo.

\begin{figure}[h!]
    \centering
    \begin{subfigure}[t]{.35\textwidth}
        \centering
        \includegraphics{6 - Approach - Dia 12}
        \captionsetup{justification=centering}
        \caption*{\emph{Dia.\@~12}}
    \end{subfigure}
    \hspace{1cm}
    \begin{subfigure}[t]{.35\textwidth}
        \centering
        \includegraphics{6 - Approach - Dia 13}
        \captionsetup{justification=centering}
        \caption*{\emph{Dia.\@~13}}
    \end{subfigure}
\end{figure}

O intuito de se jogar uma pedra preta no ponto 5-4 (a pedra marcada) no \emph{Dia.\@~13} é ganhar influência no centro. Quando Branco se aproxima com 1, Preto joga 2, confinando Branco ao canto. Com a sequência até 7, Branco garante o território no canto, e Preto ganha a influência no centro.

\pagebreak

\section{Pinças}

Quando um lado faz um movimento de aproximação, o outro frequentemente segue com uma pinça.

Após Branco se aproximar com a pedra no 4-4 com 1 no \emph{Dia.\@~14}, Branco pode jogar uma pinça com 2. (Pinças de \textbf{A} a \textbf{E} também são possíveis e frequentemente jogadas.) Em resposta, Branco geralmente invade o canto com 3 no \emph{Dia.\@~15}. A sequência até Branco 11 é um joseki. Branco ganha território no canto, e Preto consegue uma posição densa no entorno.

\begin{figure}[h!]
    \centering
    \begin{subfigure}[t]{.3\textwidth}
        \centering
        \includegraphics[width=1.2\textwidth]{6 - Pincers - Dia 14}
        \captionsetup{justification=centering}
        \caption*{\emph{Dia.\@~14}}
    \end{subfigure}
    \hspace{1cm}
    \begin{subfigure}[t]{.3\textwidth}
        \centering
        \includegraphics[width=1.2\textwidth]{6 - Pincers - Dia 15}
        \captionsetup{justification=centering}
        \caption*{\emph{Dia.\@~15}}
    \end{subfigure}
\end{figure}

\pagebreak

Quando Preto possui a pedra marcada, como no \emph{Dia.\@~16}, a direção correta para Preto bloquear é por baixo, com 4. Os movimentos até Preto 8 também constituem um joseki. Novamente, Branco consegue o canto, mas Preto conseguiu mapear uma armação territorial no canto superior direito com sua parede e a pedra marcada.

\begin{figure}[h!]
    \centering
    \hspace{-2cm}
    \begin{subfigure}[t]{.3\textwidth}
        \centering
        \includegraphics[width=1.1\textwidth]{6 - Pincers - Dia 16}
        \captionsetup{justification=centering}
        \caption*{\emph{Dia.\@~16}}
    \end{subfigure}
    \hspace{1cm}
    \begin{subfigure}[t]{.3\textwidth}
        \centering
        \includegraphics[width=1.625\textwidth]{6 - Pincers - Dia 17}
        \captionsetup{justification=centering}
        \caption*{\emph{Dia.\@~17}}
    \end{subfigure}
\end{figure}

Se Branco não estiver satisfeito com o resultado no \emph{Dia.\@~16}, em que Preto possui uma presença esmagadora no centro enquanto que suas próprias pedras se confinam ao canto, ele pode optar pela resposta 3 à Preto 2 no \emph{Dia.\@~17}. Após a sequência joseki até 13, Preto assegurou uma posição no topo e, ainda, mapeou uma esfera de influência no lado direito, mas Branco conseguiu estabelecer uma presença no centro. Branco 13 em \textbf{A} também é uma jogada tida como joseki.

No \emph{Dia.\@~18}, Branco se aproxima da pedra marcada no ponto 3-4 com 1, e Preto joga a pinça com 2. Branco sai para o centro com o movimento diagonal de 3, e Preto defende o lado direito com 4. Finalmente, Branco estabelece uma base para suas pedras deslizando até 5, e Preto delimita seu território no lado direito com a extensão até 6.

\begin{figure}[h!]
    \centering
    \begin{subfigure}[t]{.3\textwidth}
        \centering
        \includegraphics[width=1.2\textwidth]{6 - Pincers - Dia 18}
        \captionsetup{justification=centering}
        \caption*{\emph{Dia.\@~18}}
    \end{subfigure}
    \hspace{1cm}
    \begin{subfigure}[t]{.3\textwidth}
        \centering
        \includegraphics[width=1.2\textwidth]{6 - Pincers - Dia 19}
        \captionsetup{justification=centering}
        \caption*{\emph{Dia.\@~19}}
    \end{subfigure}
\end{figure}

Apesar de que um pouco distante, Preto 2 no \emph{Dia.\@~19} é também uma pinça contra 1. Branco sai para o centro com seu pulo de dois espaços em 3. Preto defende o lado direito com 4, e Branco faz uma base com suas pedras de 5 a 9.

\pagebreak

Preto 2 no \emph{Dia.\@~19} é o mais distante --- cinco intersecções de distância --- que uma pedra pode ser jogada para ser denominada como pinça. Por exemplo, Preto 2 no \emph{Dia.\@~20} não é uma pinça contra a pedra branca em 1 pois Branco pode jogar a extensão ideal de 3.

\begin{figure}[h!]
    \centering
    \begin{subfigure}[t]{.3\textwidth}
        \centering
        \includegraphics[width=1.2\textwidth]{6 - Pincers - Dia 20}
        \captionsetup{justification=centering}
        \caption*{\emph{Dia.\@~20}}
    \end{subfigure}
    \hspace{1cm}
    \begin{subfigure}[t]{.3\textwidth}
        \centering
        \includegraphics[width=1.2\textwidth]{6 - Pincers - Dia 21}
        \captionsetup{justification=centering}
        \caption*{\emph{Dia.\@~21}}
    \end{subfigure}
\end{figure}

Contra a aproximação alta de Branco 1 no \emph{Dia.\@~21}, há diversas pinças que podem ser jogadas. Preto 2, \textbf{A} e \textbf{B} são as mais comuns. Se Branco responder Preto 2 pulando ao centro com 3, Preto defenderá de maneira disciplinada e apertada o lado direito com 4.

\pagebreak

\section{Josekis}

Nas seções sobre aproximações e pinças, vários josekis básicos foram introduzidos. Esses josekis são tão básicos que constantemente surgem em partidas, então é uma boa ideia memorizá-los. Josekis provêm exemplos de bons movimentos que você pode aplicar a posições em seus próprios jogos, sendo assim, ao estudá-los, você gradativamente afiará sua intuição sobre o que constitui uma boa jogada.

Um bom livro para se começar a estudar josekis é \emph{38 Basic Josekis}~\cite{kosugi_bozulich_38_basic_joseki}. Conforme você se tornar mais forte, desejará um livro de referência mais completo no assunto. Para esse propósito, nós recomendamos a coleção de dois volumes \emph{21st Century Dictionary of Basic Josekis}~\cite{takao_shinji_21st_century_joseki_dictionary}, por Takao Shinji 9p. Apesar de que ele não contém algumas inovações recentes, outro excelente livro é o \emph{A Dictionary of Basic Josekis}~\cite{ishida_yoshio_basic_joseki_dictionary}, por Ishida Yoshio 9p. Ele contém muitos exemplos de jogos nos quais os josekis foram estudados e utilizados. Todos esses livros foram publicados pela Kiseido~\cite{kiseido}.

\pagebreak

\section{Enclausuros de Canto}

\begin{wrapfigure}{r}{60mm}
    \vspace{-30pt}
    \begin{center}
        \includegraphics[width=.5\textwidth]{6 - Corner Enclosures}
        \captionsetup{justification=centering}
        \caption*{\emph{Dia.\@~22}}
    \end{center}
    \vspace{-20pt}
\end{wrapfigure}

Ao invés de fazer uma jogada de aproximação, um jogador talvez escolha reforçar uma pedra que ele já jogou em um canto, com um enclausuro. Na abertura mostrada no \emph{Dia.\@~22}, após Branco 4, Preto faz um enclausuro de movimento de cavaleiro curto com com 5. Preto poderia também fazer o enclausuro de um espaço jogando em \textbf{A} ou o de cavaleiro longo com \textbf{B}. Essas são todas boas jogadas, mas o de cavaleiro curto é o que mais vem sendo jogado ultimamente pois tende a melhor defender o canto. O enclausuro de um espaço de Branco 6 é jogado para enfatizar o centro, mas, já que possui um flanco aberto, está vulnerável a um ataque ao redor de \textbf{C}.

\pagebreak

\section{Extensões}\label{sec:6.6:extensions}

Até aqui, consideramos movimentos feitos primariamente em torno do canto. Conforme a partida progride, entretanto, extensões ao longo dos lados precisarão ser feitas. Extensões precisam trabalhar eficientemente. Elas deveriam não ser demasiado curtas nem demasiado longas. Há três princípios similares que oferecem diretivas para a criação de extensões eficientes.

\begin{itemize}
    \item[\textbf{Princípio 1}] De uma pedra só, estenda dois espaços.
        
        De uma pedra só na terceira linha, uma extensão de dois espaços é a mais eficiente. Por exemplo, Branco faz um aproximação de cavaleiro longo contra a pedra preta no ponto 3-4 com 1 no \emph{Dia.\@~23}. Se Preto defende o canto com 2, a extensão de dois espaços de Branco 3 é a resposta ideal. Isso é um joseki e possui a distinção de ser um dos mais curtos. (Outro joseki curto é mostrado no \emph{Dia.\@~10}.)
    
    \begin{figure}[h!]
        \centering
        \begin{subfigure}[t]{.3\textwidth}
            \centering
            \includegraphics[width=1\textwidth]{6 - Extensions - Dia 23}
            \captionsetup{justification=centering}
            \caption*{\emph{Dia.\@~23}}
        \end{subfigure}
        \hfill
        \begin{subfigure}[t]{.3\textwidth}
            \centering
            \includegraphics[width=1\textwidth]{6 - Extensions - Dia 24}
            \captionsetup{justification=centering}
            \caption*{\emph{Dia.\@~24}}
        \end{subfigure}
        \hfill
        \begin{subfigure}[t]{.3\textwidth}
            \centering
            \includegraphics[width=1\textwidth]{6 - Extensions - Dia 25}
            \captionsetup{justification=centering}
            \caption*{\emph{Dia.\@~25}}
        \end{subfigure}
    \end{figure}

    \item[\textbf{Princípio 2}] De uma parede de duas pedras, estenda três espaços.
    
        No \emph{Dia.\@~24}, Branco possui uma parede de duas pedras (as pedras marcadas). A partir dessa parede, estender três espaço para 1 é o ideal. (Essa posição foi gerada a partir do joseki mostrado no \emph{Dia.\@~11}, apesar de que em uma orientação diferente.)
    \item[\textbf{Princípio 3}] De uma parede de três pedras, estenda quatro espaços. 

        No \emph{Dia.\@~25}, Branco fez uma parede de três pedras (as pedras marcadas), então ele pode estender quatro espaços até 1.
\end{itemize}

\pagebreak

\begin{wrapfigure}{r}{60mm}
    \vspace{-15pt}
    \begin{center}
        \includegraphics[width=.5\textwidth]{6 - Extensions - Dia 26}
        \captionsetup{justification=centering}
        \caption*{\emph{Dia.\@~26}}
    \end{center}
    \vspace{-20pt}
\end{wrapfigure}

Esses princípios não são tão rígidos; eles deveriam ser interpretados como diretivas. Suas extensões precisam trabalhar não somente em consonância com suas próprias pedras mas também de modo a efetivamente frustrar os planos do oponente.

Na abertura no \emph{Dia.\@~26}, nenhum movimento de aproximação foi feito, e cada lado tomou dois enclausuros de canto. A atenção agora se desloca para as extensões ao longo dos lados. Após Branco 8, onde está a maior extensão? O que deveria imediatemente saltar aos seus olhos são os dois enclausuros de canto de um espaço no lado direito, olhando um para o outro. O lado que estender de seu enclausuro primeiro tomará a iniciativa.

\begin{wrapfigure}{l}{60mm}
    \vspace{-27.5pt}
    \begin{center}
        \includegraphics[width=.5\textwidth]{6 - Extensions - Dia 27}
        \captionsetup{justification=centering}
        \caption*{\emph{Dia.\@~27}}
    \end{center}
    \vspace{-20pt}
\end{wrapfigure}

\bigskip

Por estender cinco espaços de seu enclausuro no canto superior direito com 9 no \emph{Dia.\@~27}, Preto toma a iniciativa no lado direito. Este é o ponto em que Branco também gostaria de jogar, uma vez que está na direção em que ambos os enclausuros emanam influência. Em geral, estender cinco espaços de um enclausuro é a norma.

Branco responde estendendo no lado inferior com 10. Esse movimento faz duas coisas: pára uma extensão preta a partir do enclausuro no canto inferior esquerdo; e protege o flanco do enclausuro branco. Preto 11 possui o mesmo significado. Branco 12 no lado esquerdo é a extensão de menor valor, dado que os enclausuros preto e branco não projetam muita influência nesta direção.

\pagebreak

\begin{wrapfigure}{r}{60mm}
    \vspace{-15pt}
    \begin{center}
        \includegraphics[width=.5\textwidth]{6 - Extensions - Dia 28}
        \captionsetup{justification=centering}
        \caption*{\emph{Dia.\@~28}}
    \end{center}
    \vspace{-10pt}
\end{wrapfigure}

Não é sempre possível fazer uma extensão de um enclausuro de canto. No \emph{Dia.\@~28}, por exemplo, Branco joga no meio da esfera de influência preta com 1. Preto 2 é o mais longe que Preto pode estender a partir de seu enclausuro. Branco precisa construir uma base para suas pedras estendendo para 3, e Preto reforça seu canto com 4.

\begin{wrapfigure}{l}{60mm}
    \vspace{-27.5pt}
    \begin{center}
        \includegraphics[width=.5\textwidth]{6 - Extensions - Dia 29}
        \captionsetup{justification=centering}
        \caption*{\emph{Dia.\@~29}}
    \end{center}
    \vspace{-10pt}
\end{wrapfigure}

No \emph{Dia.\@~29}, Branco quer prevenir a segunda extensão de enclausuro preta no canto inferior direito, então ele faz a aproximação alta de 1. Preto contata com 2, e isso tudo resulta no joseki até Branco 7. Preto 8 talvez pareça uma extensão curta, mas, ainda assim, é uma boa jogada, porque contém a ameaça da invasão em \textbf{A}. Além disso, ela previne a jogada branca \textbf{B}, que reforçaria a posição branca no lado direito enquanto ameaçaria o enclausuro preto acima.

\bigskip

Essa é somente uma breve introdução à abertura e sua teoria. Para um tratamento mais completo dessa etapa da partida, recomendamos os seguintes dois livros publicados pela Kiseido: \emph{Opening Theory Made Easy}~\cite{otake_opening_theory_made_easy}, por Otake Hideo 9p; e \emph{In the Beginning}~\cite{ikure_in_the_beginning}, por Ikuro Ishigure.
  \chapter{Táticas Elementares}

\section{Escadas}

Escadas constituem uma das técnicas mais básicas\footnote{Uma técnica ser básica não garante que ela seja fácil de dominar. Mesmo profissionais encontram posições de escadas difíceis e complicadas em suas partidas.} de captura. Elas também são mais do que uma técnica local, dado que requerem consciência das condições no resto do tabuleiro.

\emph{Dia.\@~1.} A pedra preta marcada está cortando Branco em dois grupos, então Branco gostaria de capturá-la. Como ele poderia fazê-lo?

\emph{Dia.\@~2.} Começa-se pelo atari em 1.

\begin{figure}[h!]
    \centering
    \begin{subfigure}[t]{.3\textwidth}
        \centering
        \includegraphics[width=.9\textwidth]{7 - Ladders - Dia 1}
        \captionsetup{justification=centering}
        \caption*{\emph{Dia.\@~1}}
    \end{subfigure}
    \hfill
    \begin{subfigure}[t]{.3\textwidth}
        \centering
        \includegraphics[width=.9\textwidth]{7 - Ladders - Dia 2}
        \captionsetup{justification=centering}
        \caption*{\emph{Dia.\@~2}}
    \end{subfigure}
    \hfill
    \begin{subfigure}[t]{.3\textwidth}
        \centering
        \includegraphics[width=.9\textwidth]{7 - Ladders - Dia 3}
        \captionsetup{justification=centering}
        \caption*{\emph{Dia.\@~3}}
    \end{subfigure}
\end{figure}

\emph{Dia.\@~3.} Se Preto tentar escapar, Branco continua com os ataris de 3 a 11 até onde Preto esgota suas possibilidades. Se Preto \textbf{A}, Preto ainda estará sob atari, então Branco poderá ignorar tal jogada. Assim que as pedras brancas entrarem em perigo, Branco poderá capturar em \textbf{B}. Esse tipo de sequência, na qual o inimigo é mantido sob atari e dirigido à borda do tabuleiro, onde ele esgota suas liberdades, é chamado de escada.

\pagebreak

\begin{figure}[h!]
    \centering
    \begin{subfigure}[t]{.4\textwidth}
        \centering
        \includegraphics[width=.9\textwidth]{7 - Ladders - Dia 4}
        \captionsetup{justification=centering}
        \caption*{\emph{Dia.\@~4}}
    \end{subfigure}
    \hspace{1cm}
    \begin{subfigure}[t]{.4\textwidth}
        \centering
        \includegraphics[width=.9\textwidth]{7 - Ladders - Dia 5}
        \captionsetup{justification=centering}
        \caption*{\emph{Dia.\@~5}}
    \end{subfigure}
\end{figure}

\emph{Dia.\@~4.} Aqui está a mesma escada em um tabuleiro maior. São precisos mais movimentos, mas, similarmente, Preto não consegue escapar.

\emph{Dia.\@~5.} Preto possui a pedra triangulada no caminho da escada. O que acontece se Branco tentar capturar, com 1, a pedra circulada?

\emph{Dia.\@~6.} Preto foge com 2. Branco faz atari com 3 a 11, mas, após 12, Preto se conecta com a pedra marcada, ganhando uma liberdade extra. Se Branco continua como anteriormente, Preto agora possui liberdades extras e terá tempo de jogar um duplo-atari em 14. Se Branco conectar com 15, Preto captura uma pedra com 16, colocando outra pedra sob atari. Branco defende com 17, mas Preto pode jogar outro duplo-atari com 18, do outro lado.

\begin{figure}[h!]
    \centering
    \begin{subfigure}[t]{.4\textwidth}
        \centering
        \includegraphics[width=.9\textwidth]{7 - Ladders - Dia 6}
        \captionsetup{justification=centering}
        \caption*{\emph{Dia.\@~6}}
    \end{subfigure}
\end{figure}

Em suma, se Preto possui uma pedra dentro ou em cima do caminho definido pelos pontos $\times$ no \emph{Dia.\@~5}, a escada não será bem sucedida.

Para uma prática adicional sobre escadas, aqui se seguem quatro problemas.

\pagebreak

\subsection{Problemas 31 a 34}

\begin{figure}[h!]
    \centering
    \begin{subfigure}[t]{.35\textwidth}
        \includegraphics[width=1.25\textwidth]{7 - Problem 31}
        \captionsetup{margin={.15in,0in}}
        \caption*{\textbf{Problema 31}: \emph{Como Preto pode capturar as cinco pedras brancas?}}
    \end{subfigure}
    \hspace{1.5cm}
    \begin{subfigure}[t]{.35\textwidth}
        \includegraphics[width=1.25\textwidth]{7 - Problem 32}
        \captionsetup{margin={.15in,0in}}
        \caption*{\textbf{Problema 32}: \emph{Como Branco deveria responder ao duplo-atari preto em 1?}}
    \end{subfigure}
    \par\bigskip
    \par\bigskip
    \begin{subfigure}[t]{.35\textwidth}
        \includegraphics[width=1.25\textwidth]{7 - Problem 33}
        \captionsetup{margin={.15in,0in}}
        \caption*{\textbf{Problema 33}: \emph{Preto possui duas maneiras de fazer atari. Qual é a maneira correta?}}
    \end{subfigure}
    \hspace{1.5cm}
    \begin{subfigure}[t]{.35\textwidth}
        \includegraphics[width=1.25\textwidth]{7 - Problem 34}
        \captionsetup{margin={.15in,0in}}
        \caption*{\textbf{Problema 34}: \emph{Preto possui duas maneiras de fazer atari. Qual é a escolha correta?}}
    \end{subfigure}
\end{figure}

\pagebreak

\subsection{Respostas aos Problemas 31 a 34}

\subsubsection*{Resposta ao Problema 31}

Preto pode capturar as pedras brancas em uma escada com 1 e 3 no \emph{Dia.\@~1}. Depois de Branco 4, Preto direciona as pedras brancas à borda do tabuleiro com 5. Após Preto 7, não há mais nenhuma maneira de Branco evitar ser capturado.

\begin{figure}[h!]
    \centering
    \begin{subfigure}[t]{.425\textwidth}
        \includegraphics[width=1\textwidth]{7 - Problem 31 - Dia 1}
        \captionsetup{justification=centering}
        \caption*{\emph{Dia.\@~1. Correto}}
    \end{subfigure}
    \hspace{1cm}
    \begin{subfigure}[t]{.425\textwidth}
        \includegraphics[width=1\textwidth]{7 - Problem 31 - Dia 2}
        \captionsetup{justification=centering}
        \caption*{\emph{Dia.\@~2. Errado}}
    \end{subfigure}
\end{figure}

Preto 1 no \emph{Dia.\@~2} não configura uma escada, então Branco pode aumentar, em duas, suas liberdades descendo para 2. As pedras brancas escapam.

\pagebreak

\subsubsection*{Resposta ao Problema 32}

Branco deveria responder ao duplo-atari de Preto 1 capturando as dez pedras com 2. Isso seria uma perda enorme para Preto.

\begin{figure}[h!]
    \centering
    \begin{subfigure}[t]{.425\textwidth}
        \includegraphics[width=1\textwidth]{7 - Problem 32 - Dia 1}
        \captionsetup{justification=centering}
        \caption*{\emph{Dia.\@~1. Correto}}
    \end{subfigure}
    \hspace{1cm}
    \begin{subfigure}[t]{.425\textwidth}
        \includegraphics[width=1\textwidth]{7 - Problem 32 - Dia 2}
        \captionsetup{justification=centering}
        \caption*{\emph{Dia.\@~2. Errado}}
    \end{subfigure}
\end{figure}

Se Branco responder Preto 1 no \emph{Dia.\@~2} tentando resgatar as duas pedras com 2, Preto captura com 3 e ameaça capturar a pedra marcada. Ele também ameaça capturar três pedras em uma escada jogando em \textbf{A}. Preto também possui múltiplas escolhas de duplos-ataris à esquerda, como \textbf{B}.

\pagebreak

\subsubsection*{Resposta ao Problema 33}

Preto deveria dirigir Branco à borda do tabuleiro com o atari de 1 no \emph{Dia.\@~1}. Se Preto fizer atari de novo com 3, não há maneira de Branco sair do atari.

\begin{figure}[h!]
    \centering
    \begin{subfigure}[t]{.425\textwidth}
        \includegraphics[width=1\textwidth]{7 - Problem 33 - Dia 1}
        \captionsetup{justification=centering}
        \caption*{\emph{Dia.\@~1. Correto}}
    \end{subfigure}
    \hspace{1cm}
    \begin{subfigure}[t]{.425\textwidth}
        \includegraphics[width=1\textwidth]{7 - Problem 33 - Dia 2}
        \captionsetup{justification=centering}
        \caption*{\emph{Dia.\@~2. Errado}}
    \end{subfigure}
\end{figure}

Jogar o atari na primeira linha com 1 no \emph{Dia.\@~2} é um erro. Branco vira com 2, colocando as pedras marcadas em atari. Preto é forçado a conectar com 3. Ao invés de 4, Branco também pode jogar os duplo-ataris em \textbf{A} e \textbf{B}.

\pagebreak

\subsubsection*{Resposta ao Problema 34}

Preto deveria fazer primeiro atari pela esquerda com 1, então iniciar a escada com 3. A escada agora não colide com a pedra marcada.
        
\begin{figure}[h!]
    \centering
    \begin{subfigure}[t]{.425\textwidth}
        \includegraphics[width=1\textwidth]{7 - Problem 34 - Dia 1}
        \captionsetup{justification=centering}
        \caption*{\emph{Dia.\@~1. Correto}}
    \end{subfigure}
    \hspace{1cm}
    \begin{subfigure}[t]{.425\textwidth}
        \includegraphics[width=1\textwidth]{7 - Problem 34 - Dia 2}
        \captionsetup{justification=centering}
        \caption*{\emph{Dia.\@~2. Errado}}
    \end{subfigure}
\end{figure}

Se Preto imediatamente começar a escada com a sequência de 1 a 5 no \emph{Dia.\@~2}, devido à presença da pedra marcada, Branco 6 se torna atari na pedra preta em 3. Preto precisa defender conectando em \textbf{A} e, assim, Branco pode jogar \textbf{B}, e suas pedras se salvam.

\pagebreak

\section{Redes}

\emph{Dia.\@~7.} Novamente, Branco quer capturar a pedra marcada.

\begin{figure}[h!]
    \centering
    \begin{subfigure}[t]{.31\textwidth}
        \includegraphics[width=1\textwidth]{7 - Nets - Dia 7}
        \captionsetup{justification=centering}
        \caption*{\emph{Dia.\@~7}}
    \end{subfigure}
    \hfill
    \begin{subfigure}[t]{.31\textwidth}
        \includegraphics[width=1\textwidth]{7 - Nets - Dia 8}
        \captionsetup{justification=centering}
        \caption*{\emph{Dia.\@~8}}
    \end{subfigure}
    \hfill
    \begin{subfigure}[t]{.31\textwidth}
        \includegraphics[width=1\textwidth]{7 - Nets - Dia 9}
        \captionsetup{justification=centering}
        \caption*{\emph{Dia.\@~9}}
    \end{subfigure}
\end{figure}

\emph{Dia.\@~8.} Entretanto, a escada correria em uma pedra preta no canto superior esquerdo e colapsaria. Um jogador que tenta uma escada que não funciona, como esta, incorre perdas enormes pois é deixado com vários pontos de corte, como \textbf{A} e \textbf{B} nessa posição. Branco precisa jogar diferentemente.

\emph{Dia.\@~9.} Nesta posição, Branco possui uma alternativa para capturar Preto: é possível contê-lo com 1.

\emph{Dia.\@~10.} Se Preto irrazoavelmente tentar escapar com 2 e 4, Branco capturará com 3 e 5.

\begin{figure}[h!]
    \centering
    \begin{subfigure}[t]{.31\textwidth}
        \includegraphics[width=1\textwidth]{7 - Nets - Dia 10}
        \captionsetup{justification=centering}
        \caption*{\emph{Dia.\@~10}}
    \end{subfigure}
    \hspace{1cm}
    \begin{subfigure}[t]{.31\textwidth}
        \includegraphics[width=1\textwidth]{7 - Nets - Dia 11}
        \captionsetup{justification=centering}
        \caption*{\emph{Dia.\@~11}}
    \end{subfigure}
\end{figure}

\emph{Dia.\@~11.} Este é outro exemplo de captura com uma rede. Branco pode capturar a pedra marcada jogando primeiro um atari com 1 e, então, enredando-o com 3.

Aqui seguem quatro problemas para se praticar redes.

\pagebreak

\subsection{Problemas 35 a 38}

\begin{figure}[h!]
    \centering
    \begin{subfigure}[t]{.35\textwidth}
        \includegraphics[width=1.25\textwidth]{7 - Problem 35}
        \captionsetup{margin={.15in,0in}}
        \caption*{\textbf{Problema 35}: \emph{Como Preto pode capturar a pedra marcada?}}
    \end{subfigure}
    \hspace{1.5cm}
    \begin{subfigure}[t]{.35\textwidth}
        \includegraphics[width=1.25\textwidth]{7 - Problem 36}
        \captionsetup{margin={.15in,0in}}
        \caption*{\textbf{Problema 36}: \emph{Como Preto pode capturar as duas pedras marcadas?}}
    \end{subfigure}
    \par\bigskip
    \begin{subfigure}[t]{.35\textwidth}
        \includegraphics[width=1.25\textwidth]{7 - Problem 37}
        \captionsetup{margin={.15in,0in}}
        \caption*{\textbf{Problema 37}: \emph{Como Preto pode capturar a pedra marcada?}}
    \end{subfigure}
    \hspace{1.5cm}
    \begin{subfigure}[t]{.35\textwidth}
        \includegraphics[width=1.25\textwidth]{7 - Problem 38}
        \captionsetup{margin={.15in,0in}}
        \caption*{\textbf{Problema 38}: \emph{Como Preto pode capturar as duas pedras marcadas?}}
    \end{subfigure}
\end{figure}

\pagebreak

\subsection{Respostas aos Problemas 35 a 38}

\subsubsection*{Resposta ao Problema 35}

Preto deveria enredar a pedra branca pulando com 1 no \emph{Dia.\@~1}. Branco não consegue escapar. Se ele jogar \textbf{A}, Preto bloqueia com \textbf{B}, colocando as pedras brancas em um atari do qual não há escapatória.

\begin{figure}[h!]
    \centering
    \begin{subfigure}[t]{.425\textwidth}
        \includegraphics[width=1\textwidth]{7 - Problem 35 - Dia 1}
        \captionsetup{justification=centering}
        \caption*{\emph{Dia.\@~1. Correto}}
    \end{subfigure}
    \hspace{1cm}
    \begin{subfigure}[t]{.425\textwidth}
        \includegraphics[width=1\textwidth]{7 - Problem 35 - Dia 2}
        \captionsetup{justification=centering}
        \caption*{\emph{Dia.\@~2. Errado}}
    \end{subfigure}
\end{figure}

Se Preto fizer atari com 1 no \emph{Dia.\@~2}, Branco estenderá em 2. Preto tentará contê-lo com 3 a 7, mas Branco escapa para o centro com 8.

\pagebreak

\subsubsection*{Resposta ao Problema 36}

Preto deveria jogar uma rede nas pedras brancas com 1 no \emph{Dia.\@~1}. Branco não escapará. Se Branco jogar 2, Preto bloqueia com 3, colocando as pedras brancas em um inescapável atari.

\begin{figure}[h!]
    \centering
    \begin{subfigure}[t]{.425\textwidth}
        \includegraphics[width=1\textwidth]{7 - Problem 36 - Dia 1}
        \captionsetup{justification=centering}
        \caption*{\emph{Dia.\@~1. Correto}}
    \end{subfigure}
    \hspace{1cm}
    \begin{subfigure}[t]{.425\textwidth}
        \includegraphics[width=1\textwidth]{7 - Problem 36 - Dia 2}
        \captionsetup{justification=centering}
        \caption*{\emph{Dia.\@~2. Errado}}
    \end{subfigure}
\end{figure}

Se Preto fizer atari com 1 no \emph{Dia.\@~1}, Branco estenderá para 2. Preto tenta contê-lo com 3 a 5, mas Branco escapa ao centro com 6.

\pagebreak

\subsubsection*{Resposta ao Problema 37}

Preto deveria jogar a rede de 1 nas pedras brancas no \emph{Dia.\@~1}. Branco não tem para onde fugir. Se ele estender para 2, Preto bloqueará com 3, e as pedras brancas estarão capturadas de qualquer maneira.

\begin{figure}[h!]
    \centering
    \begin{subfigure}[t]{.425\textwidth}
        \includegraphics[width=1\textwidth]{7 - Problem 37 - Dia 1}
        \captionsetup{justification=centering}
        \caption*{\emph{Dia.\@~1. Correto}}
    \end{subfigure}
    \hspace{1cm}
    \begin{subfigure}[t]{.425\textwidth}
        \includegraphics[width=1\textwidth]{7 - Problem 37 - Dia 2}
        \captionsetup{justification=centering}
        \caption*{\emph{Dia.\@~2. Errado}}
    \end{subfigure}
\end{figure}

Se Preto fizer atari com 1 no \emph{Dia.\@~2}, Branco estende em 2. Preto tentará contê-lo com 3 a 5, mas Branco escapa ao centro com 6.

\pagebreak

\subsubsection*{Resposta ao Problema 38}

Preto deveria enredar Branco com 1 no \emph{Dia.\@~1}. Branco não escapará. Se ele jogar \textbf{A}, Preto bloqueará em \textbf{B}, submetendo as pedras brancas a um atari sem escapatória.
    
Se Preto fizer atari com 1 e 3 no \emph{Dia.\@~2}, Branco estende com 2 e 4 e conecta suas pedras com aquelas à direita.

\begin{figure}[h!]
    \centering
    \begin{subfigure}[t]{.425\textwidth}
        \includegraphics[width=1\textwidth]{7 - Problem 38 - Dia 1}
        \captionsetup{justification=centering}
        \caption*{\emph{Dia.\@~1. Correto}}
    \end{subfigure}
    \hspace{1cm}
    \begin{subfigure}[t]{.425\textwidth}
        \includegraphics[width=1\textwidth]{7 - Problem 38 - Dia 2}
        \captionsetup{justification=centering}
        \caption*{\emph{Dia.\@~2. Errado}}
    \end{subfigure}
\end{figure}

Se Preto fizer atari com 1 e 3 no \emph{Dia.\@~2}, Branco estende com 2 e 4 e conecta suas pedras com aquelas à direita.

\pagebreak

\section{Táticas de Sacrifício}

Go abunda em termos de táticas de sacrifício que levam a capturas de pedras adversárias. Aqui seguem alguns exemplos.

\emph{Dia.\@~12.} Branco possui uma maneira de capturar as duas pedras marcadas, o que propiciará a conexão das quatro pedras no canto com aquelas isoladas no exterior.

\begin{figure}[h!]
    \centering
    \begin{subfigure}[t]{.31\textwidth}
        \includegraphics[width=1\textwidth]{7 - Sacrifice - Dia 12}
        \captionsetup{justification=centering}
        \caption*{\emph{Dia.\@~12}}
    \end{subfigure}
    \hspace{1cm}
    \begin{subfigure}[t]{.31\textwidth}
        \includegraphics[width=1\textwidth]{7 - Sacrifice - Dia 13}
        \captionsetup{justification=centering}
        \caption*{\emph{Dia.\@~13}}
    \end{subfigure}
\end{figure}

\emph{Dia.\@~13.} Branco inicia o sacrifício com 1, colocando as duas pedras sob atari. Preto poderá capturar com 2.

\emph{Dia.\@~14.} Esse é o resultado. As três pedras marcadas estão em atari, então\ldots

\begin{figure}[h!]
    \centering
    \begin{subfigure}[t]{.31\textwidth}
        \includegraphics[width=1\textwidth]{7 - Sacrifice - Dia 14}
        \captionsetup{justification=centering}
        \caption*{\emph{Dia.\@~14}}
    \end{subfigure}
    \hspace{1cm}
    \begin{subfigure}[t]{.31\textwidth}
        \includegraphics[width=1\textwidth]{7 - Sacrifice - Dia 15}
        \captionsetup{justification=centering}
        \caption*{\emph{Dia.\@~15}}
    \end{subfigure}
\end{figure}

\emph{Dia.\@~15.} Preto as captura com 3. Esse padrão, no qual um dos lados captura, e o outro recaptura, é denominado de \emph{ricochete}\footnote{Em inglês, o termo já é muito sedimentado, e é conhecido como \emph{snapback}.}.

\pagebreak

\emph{Dia.\@~16.} Como é que Preto poderia capturar as pedras brancas?

\begin{figure}[h!]
    \centering
    \begin{subfigure}[t]{.31\textwidth}
        \includegraphics[width=1\textwidth]{7 - Sacrifice - Dia 16}
        \captionsetup{justification=centering}
        \caption*{\emph{Dia.\@~16}}
    \end{subfigure}
    \hfill
    \begin{subfigure}[t]{.31\textwidth}
        \includegraphics[width=1\textwidth]{7 - Sacrifice - Dia 17}
        \captionsetup{justification=centering}
        \caption*{\emph{Dia.\@~17}}
    \end{subfigure}
    \hfill
    \begin{subfigure}[t]{.31\textwidth}
        \includegraphics[width=1\textwidth]{7 - Sacrifice - Dia 18}
        \captionsetup{justification=centering}
        \caption*{\emph{Dia.\@~18}}
    \end{subfigure}
\end{figure}

\emph{Dia.\@~17.} Se Preto simplesmente fizer atari com 1, Branco conectará com 2. Se Preto cortar com 3, Branco fará atari com 4, e as pedras brancas não poderão ser capturadas.

\emph{Dia.\@~18.} Se Preto quiser capturar algumas pedras, ele terá de sacrificar uma com 1. Se Branco capturar com 2\ldots

\emph{Dia.\@~19.} Preto faz atari com 3. Se Branco conecta com 4, Preto corta com 5, colocando as nove pedras Brancas em atari. Branco não consegue evitar de ser capturado no próximo movimento. Ao invés de Branco 4\ldots

\begin{figure}[h!]
    \centering
    \begin{subfigure}[t]{.31\textwidth}
        \includegraphics[width=1\textwidth]{7 - Sacrifice - Dia 19}
        \captionsetup{justification=centering}
        \caption*{\emph{Dia.\@~19}}
    \end{subfigure}
    \hfill
    \begin{subfigure}[t]{.31\textwidth}
        \includegraphics[width=1\textwidth]{7 - Sacrifice - Dia 20}
        \captionsetup{justification=centering}
        \caption*{\emph{Dia.\@~20}}
    \end{subfigure}
    \hfill
    \begin{subfigure}[t]{.31\textwidth}
        \includegraphics[width=1\textwidth]{7 - Sacrifice - Dia 21}
        \captionsetup{justification=centering}
        \caption*{\emph{Dia.\@~21}}
    \end{subfigure}
\end{figure}

\emph{Dia.\@~20.} Já que Branco não pode evitar de perder quatro pedras, ele não deveria defendê-las. Em seu lugar, ele deveria jogar jogar 4 em outro lugar. Quando Preto capturar com 5\ldots

\emph{Dia.\@~21.} Branco pode recapturar uma pedra com 6, desta maneira, limitando sua perda.

\pagebreak

\subsection{Problemas 39 a 46}

\begin{figure}[h!]
    \centering
    \begin{subfigure}[t]{.35\textwidth}
        \includegraphics[width=1.25\textwidth]{7 - Problem 39}
        \captionsetup{margin={0in,0.15in}}
        \caption*{\textbf{Problema 39}: \emph{Como Preto pode capturar duas pedras?}}
    \end{subfigure}
    \hspace{1.5cm}
    \begin{subfigure}[t]{.35\textwidth}
        \includegraphics[width=1.25\textwidth]{7 - Problem 40}
        \captionsetup{margin={0in,.15in}}
        \caption*{\textbf{Problema 40}: \emph{Como Preto pode capturar três pedras?}}
    \end{subfigure}
    \par\bigskip
    \begin{subfigure}[t]{.35\textwidth}
        \includegraphics[width=1.25\textwidth]{7 - Problem 41}
        \captionsetup{margin={0in,.15in}}
        \caption*{\textbf{Problema 41}: \emph{Como Preto pode capturar três pedras?}}
    \end{subfigure}
    \hspace{1.5cm}
    \begin{subfigure}[t]{.35\textwidth}
        \includegraphics[width=1.25\textwidth]{7 - Problem 42}
        \captionsetup{margin={0in,.15in}}
        \caption*{\textbf{Problema 42}: \emph{Como Preto pode capturar três pedras?}}
    \end{subfigure}
\end{figure}

\pagebreak

\begin{figure}[h!]
    \centering
    \begin{subfigure}[t]{.35\textwidth}
        \includegraphics[width=1.25\textwidth]{7 - Problem 43}
        \captionsetup{margin={.15in,0in}}
        \caption*{\textbf{Problema 43}: \emph{Como Preto pode capturar três pedras?}}
    \end{subfigure}
    \hspace{1.5cm}
    \begin{subfigure}[t]{.35\textwidth}
        \includegraphics[width=1.25\textwidth]{7 - Problem 44}
        \captionsetup{margin={.15in,0in}}
        \caption*{\textbf{Problema 44}: \emph{Como Preto pode capturar seis pedras?}}
    \end{subfigure}
    \par\bigskip
    \begin{subfigure}[t]{.35\textwidth}
        \includegraphics[width=1.25\textwidth]{7 - Problem 45}
        \captionsetup{margin={.15in,0in}}
        \caption*{\textbf{Problema 45}: \emph{Como Preto pode conectar todas as suas pedras?}}
    \end{subfigure}
    \hspace{1.5cm}
    \begin{subfigure}[t]{.35\textwidth}
        \includegraphics[width=1.25\textwidth]{7 - Problem 46}
        \captionsetup{margin={.15in,0in}}
        \caption*{\textbf{Problema 46}: \emph{Como Preto pode conectar todas as suas pedras?}}
    \end{subfigure}
\end{figure}

\pagebreak

\subsection{Respostas aos Problemas 39 a 46}

\subsubsection*{Resposta ao Problema 39}

Preto deveria fazer atari nas duas pedras com 1 no \emph{Dia.\@~1}. Se Branco responder capturando a pedra marcada em \textbf{A}, Preto responderá jogando na pedra marcada e capturando três pedras.
    
\begin{figure}[h!]
    \centering
    \begin{subfigure}[t]{.425\textwidth}
        \includegraphics[width=1\textwidth]{7 - Problem 39 - Dia 1}
        \captionsetup{justification=centering}
        \caption*{\emph{Dia.\@~1. Correto}}
    \end{subfigure}
    \hspace{1cm}
    \begin{subfigure}[t]{.425\textwidth}
        \includegraphics[width=1\textwidth]{7 - Problem 39 - Dia 2}
        \captionsetup{justification=centering}
        \caption*{\emph{Dia.\@~2. Errado}}
    \end{subfigure}
\end{figure}

Se Preto defender a pedra em atari conectando em 1 no \emph{Dia.\@~2}, Branco conectará, e Preto não estará apto a capturar nenhuma pedra.

\pagebreak

\subsubsection*{Resposta ao Problema 40}

Preto deveria sacrificar uma pedra com 1 no \emph{Dia.\@~1}. Se Branco capturar em \textbf{A}, ele ainda estará sob atari, portanto Preto jogará de volta em 1, e capturará quatro pedras.

\begin{figure}[h!]
    \centering
    \begin{subfigure}[t]{.425\textwidth}
        \includegraphics[width=1\textwidth]{7 - Problem 40 - Dia 1}
        \captionsetup{justification=centering}
        \caption*{\emph{Dia.\@~1. Correto}}
    \end{subfigure}
    \hspace{1cm}
    \begin{subfigure}[t]{.425\textwidth}
        \includegraphics[width=1\textwidth]{7 - Problem 40 - Dia 2}
        \captionsetup{justification=centering}
        \caption*{\emph{Dia.\@~2. Errado}}
    \end{subfigure}
\end{figure}

Conectar com Preto 1 no \emph{Dia.\@~2} oferece a oportunidade de Branco consertar seu defeito em sua forma através da conexão em 2.

\pagebreak

\subsubsection*{Resposta ao Problema 41}

Preto deveria sacrificar uma pedra em 1 no \emph{Dia.\@~1}. Se Branco capturar em \textbf{A}, ele ainda estará sob atari, então Preto jogará de volta em 1 e capturará quatro pedras.
    
\begin{figure}[h!]
    \centering
    \begin{subfigure}[t]{.425\textwidth}
        \includegraphics[width=1\textwidth]{7 - Problem 41 - Dia 1}
        \captionsetup{justification=centering}
        \caption*{\emph{Dia.\@~1. Correto}}
    \end{subfigure}
    \hspace{1cm}
    \begin{subfigure}[t]{.425\textwidth}
        \includegraphics[width=1\textwidth]{7 - Problem 41 - Dia 2}
        \captionsetup{justification=centering}
        \caption*{\emph{Dia.\@~2. Errado}}
    \end{subfigure}
\end{figure}

Se Preto fizer atari com 1 no \emph{Dia.\@~2}, Branco conectará em 2, e Preto não conseguirá capturar nenhuma pedra.

\pagebreak

\subsubsection*{Resposta ao Problema 42}

Preto deveria sacrificar uma pedra com 1 no \emph{Dia.\@~1}. Se Branco capturar em \textbf{A}, ele ainda estará sob atari, e Preto jogará em 1 para capturar as quatro pedras.
    
\begin{figure}[h!]
    \centering
    \begin{subfigure}[t]{.425\textwidth}
        \includegraphics[width=1\textwidth]{7 - Problem 42 - Dia 1}
        \captionsetup{justification=centering}
        \caption*{\emph{Dia.\@~1. Correto}}
    \end{subfigure}
    \hspace{1cm}
    \begin{subfigure}[t]{.425\textwidth}
        \includegraphics[width=1\textwidth]{7 - Problem 42 - Dia 2}
        \captionsetup{justification=centering}
        \caption*{\emph{Dia.\@~2. Errado}}
    \end{subfigure}
\end{figure}

Se Preto capturar uma pedra com 1 no \emph{Dia.\@~2}, Branco conectará em 2. Agora, as quatro pedras marcadas não têm como escapar e serão capturadas.

\pagebreak

\subsubsection*{Resposta ao Problema 43}

Nenhum sacrifício é necessário. Preto deveria simplesmente fazer atari com 1 no \emph{Dia.\@~1}. Se Branco conectar em \textbf{A}, Preto capturará cinco pedras jogando em \textbf{B}. Se Branco conectar em \textbf{B}, Preto \textbf{A} capturará três pedras.
    
\begin{figure}[h!]
    \centering
    \begin{subfigure}[t]{.425\textwidth}
        \includegraphics[width=1\textwidth]{7 - Problem 43 - Dia 1}
        \captionsetup{justification=centering}
        \caption*{\emph{Dia.\@~1. Correto}}
    \end{subfigure}
    \hspace{1cm}
    \begin{subfigure}[t]{.425\textwidth}
        \includegraphics[width=1\textwidth]{7 - Problem 43 - Dia 2}
        \captionsetup{justification=centering}
        \caption*{\emph{Dia.\@~2. Errado}}
    \end{subfigure}
\end{figure}

Sacrificar uma pedra em 1 no \emph{Dia.\@~2} é um erro. Após os movimentos até Preto 5, Branco conectará em 1, e todas as suas pedras estarão seguras. Este exemplo demonstra que sacrifícios nem sempre funcionam\footnote{Especialmente quando próximos ao canto, em que as liberdades de cada pedra se comportam de maneiras mais peculiares.}.

\pagebreak

\subsubsection*{Resposta ao Problema 44}

Preto deveria fazer atari com 1 no \emph{Dia.\@~1}, sacrificando a pedra marcada. Se Branco capturar com 2, ele ainda estará em atari, então Preto jogará de volta onde a pedra marcada estava e capturará as sete pedras.
 
\begin{figure}[h!]
    \centering
    \begin{subfigure}[t]{.425\textwidth}
        \includegraphics[width=1\textwidth]{7 - Problem 44 - Dia 1}
        \captionsetup{justification=centering}
        \caption*{\emph{Dia.\@~1. Correto}}
    \end{subfigure}
    \hspace{1cm}
    \begin{subfigure}[t]{.425\textwidth}
        \includegraphics[width=1\textwidth]{7 - Problem 44 - Dia 2}
        \captionsetup{justification=centering}
        \caption*{\emph{Dia.\@~2. Errado}}
    \end{subfigure}
\end{figure}

Preto 1 no \emph{Dia.\@~1} também é atari, mas as pedras pretas possuem uma escassez de liberdades, então Branco conseguirá capturar três pedras com 2, e suas seis pedras estarão seguras.

\pagebreak

\subsubsection*{Resposta ao Problema 45}

Se Preto jogar 1 no \emph{Dia.\@~1}, Branco não consegue escapar do atari, já que possui falta de liberdades. Se ele conectar em \textbf{A}, Preto capturará as oito pedras jogando em \textbf{B}. Se Branco conectar em \textbf{B}, Preto capturará seis pedras jogando em \textbf{A}.
 
\begin{figure}[h!]
    \centering
    \begin{subfigure}[t]{.425\textwidth}
        \includegraphics[width=1\textwidth]{7 - Problem 45 - Dia 1}
        \captionsetup{justification=centering}
        \caption*{\emph{Dia.\@~1. Correto}}
    \end{subfigure}
    \hspace{1cm}
    \begin{subfigure}[t]{.425\textwidth}
        \includegraphics[width=1\textwidth]{7 - Problem 45 - Dia 2}
        \captionsetup{justification=centering}
        \caption*{\emph{Dia.\@~2. Errado}}
    \end{subfigure}
\end{figure}

Preto 1 no \emph{Dia.\@~2} é devagar demais. Branco conectará no ponto-chave de 2. Se Preto fizer atari em \textbf{A}, Branco resgata suas pedras conectando em \textbf{B}.

\pagebreak

\subsubsection*{Resposta ao Problema 46}

Se preto sacrificar em 1 no \emph{Dia.\@~1}, Branco precisará capturar em 2. Preto agora faz atari com 3. Se Branco conectar em 1, Preto captura seis pedras jogando em \textbf{A}. Se Branco conectar em \textbf{A}, Preto captura quatro pedras jogando em 1.
 
\begin{figure}[h!]
    \centering
    \begin{subfigure}[t]{.425\textwidth}
        \includegraphics[width=1\textwidth]{7 - Problem 46 - Dia 1}
        \captionsetup{justification=centering}
        \caption*{\emph{Dia.\@~1. Correto}}
    \end{subfigure}
    \hspace{1cm}
    \begin{subfigure}[t]{.425\textwidth}
        \includegraphics[width=1\textwidth]{7 - Problem 46 - Dia 2}
        \captionsetup{justification=centering}
        \caption*{\emph{Dia.\@~2. Errado}}
    \end{subfigure}
\end{figure}

Se Preto jogar qualquer outro movimento, em 1 no \emph{Dia.\@~2} por exemplo, Branco conectará em 2. Se Preto agora jogar em \textbf{A}, Branco conectará em \textbf{B}, e suas pedras estarão seguras. Entretanto, as pedras pretas ainda estão separadas.
  \chapter{Vida e Morte}

Nos \emph{Dias. 15 a 18} no \autoref{chap:cinco}, nós brevemente explicamos a diferença entre olhos reais e olhos falsos. Neste capítulo, mostraremos técnicas para a criação de olhos falsos nos grupos adversários, e explicaremos o conceito de olhos falsos.

\section{Olhos Falsos}

\emph{Dia. 1.} O grupo preto, que está confinado ao canto, possui somente um olho real, e um olho falso --- o ponto em \textbf{A} ---, portanto ele está morto. No final da partida, se Preto se recusar a aceitar que este grupo está morto, Branco pode demonstrá-lo através dos movimentos no \emph{Dia. 2}.

\emph{Dia. 2.} Teoricamente, no final da partida, Branco poderá capturar Preto com 1 a 5. Jogadores experientesnão jogariam tais movimentos, eles reconheceriam tal grupo como morto.

\emph{Dia. 3.} Esse grupo preto não está morto ainda, mas Branco pode matá-lo.

\emph{Dia. 4.} Branco sacrifica uma pedra com 1, fazendo com que o segundo olho preto se torne falso. Se Preto capturar essa pedra com \textbf{A}, ele acabará com uma posição que é essencialmente idêntica ao \emph{Dia. 1}.

\emph{Dia. 5.} Para fazer dois olhos, Preto precisa conectar em 1. O grupo preto não poderá, então, mais ser morto.

\subsection{Exemplo 1}

O grupo preto no \emph{Dia. 1} está instável. Ele viverá ou morrerá, dependendo de qual será o próximo movimento.

Se Branco jogar primeiro, ele matará Preto com a colocação de 1 no \emph{Dia. 2}. Se Preto conectar com 2, Branco toma o ponto-chave de 3. Após a captura preta com 4\ldots

Branco sacrifica uma pedra com 5 no \emph{Dia. 3}, criando um olho falso naquele ponto. Preto não consegue mais fazer um segundo olho, portanto seu grupo está morto.

Se Preto jogar primeiro, ele poderá fazer um segundo olhopara seu grupo pela conexão de 1 no \emph{Dia. 4}. Se Branco \textbf{A}, Preto captura em \textbf{B} e obtém seus dois olhos.

\subsection{Exemplo 2}

O grupo preto no \emph{Dia. 1} está instável, incompleto. Ele viverá ou morrerá baseado no próximo movimento.

Se Branco jogar primeiro, ele poderá matar o grupo preto com o sacrifício de 1 no \emph{Dia. 2}. O ponto 1 agora é um olho falso. Capturar com 2 não ajuda Preto a criar um olho. Seu grupo está morto a partir daí.

O ponto \textbf{A} no \emph{Dia. 3} é um olho falso. Apesar de que Branco não precisa jogar lá, ele pode forçar Preto a jogar naquele ponto com o atari em \textbf{B}, se desafiado a demonstrar que o grupo Preto está morto.

Se Preto jogar primeiro, ele pode fazer um segundo olho para seu grupo conectando em 1 no \emph{Dia. 4}.

\subsection{Exemplo 3}

O grupo preto no \emph{Dia. 1} está incompleto. Vida ou morte dependem da próxima jogada.

Se Branco jogar primeiro, ele poderá jogar 1 no \emph{Dia 2} e transformar o olho em \textbf{A} em um olho falso. Preto está morto.

Se for turno preto, ele pode tornar o ponto \textbf{A} no \emph{Dia. 3} em um olho verdadeiro conectando em 1. 

\section{Espaço de Olho e Forma de Olho}

Quando um grupo cerca um espaço contíguo aberto de vários pontos, a questão de se esse espaço será suficientemente grande para assegurar dois olhos surge.

\emph{Dia. 1.} O grupo preto possui um espaço de três intersecções como olho, e seu status é instável.

\emph{Dia. 2.} Se Preto jogar 1, ele está vivo, pois possui dois olhos separados.

\emph{Dia. 3.} Porém, se Branco jogar 1, Preto está morto.

\emph{Dias. 4 e 5.} Os movimentos até Branco 11 nesses dois diagramas provam que o grupo Preto está morto. Branco simplesmente preenche todas as liberdades pretas mostradas. Preto não tem como se defender.

Aqui seguem mais alguns exemplos.

\emph{Dia. 6.} O grupo preto possui um olho de quatro espaços, e está vivo já.

\emph{Dia. 7.} Se Branco jogar em 1, Preto joga 2 e, novamente, ele está vivo com dois olhos separados.

\emph{Dia. 8.} Similarmente, se Branco jogar 1, Preto jogará 2 e, novamente, ele estará vivo com dois olhos separados.

\emph{Dia. 9.} Após Branco 1, Preto não pode passar ou ignorar o movimento branco. Branco continuará com 3 e Preto morrerá.

\emph{Dia. 10.} Branco pode demonstrar que Preto está morto jogando 5 a 9. Depois da captura por Preto em 10\ldots

\emph{Dia. 11.} Branco joga em 11, e a situação se torna clara: Preto não possui dois olhos.

\emph{Dia. 12.} O grupo preto está vivo ou morto, dependendo de quem for a vez.

\emph{Dia. 13.} Se for turno branco, e ele jogar em 1, o grupo preto está morto.

\emph{Dia. 14.} Se for turno preto, ele pode fazer três olhos jogando em 1.

\emph{Dia. 15.} O grupo preto é um quadrado de quatro espaços, e está morto já.

\emph{Dia. 16.} Mesmo se Preto jogar primeiro, tudo que ele pode fazer é jogar em 1 (ou qualquer outro ponto simétrico) e ameaçar fazer dois olhos jogando em 2. No entanto, Branco pode jogar 2 primeiro, e Preto é deixado sem uma resposta. Se ele jogar qualquer um entre \textbf{A} ou \textbf{B}, ele coloca todo seu grupo em atari.

\emph{Dia. 17.} Se Branco for desafiado a provar que o grupo Preto está morto, tudo que ele precisa fazer é jogar o atari em 4. Se Preto 5 capturar, Branco 6 joga novamente em 4, e todo o grupo preto inevitavelmente será capturado.

\emph{Dia. 18.} O grupo preto está vivo ou morto, dependendo de quem for o turno.

\emph{Dia. 19.} Se Branco jogar primeiro, ele pode matar Preto com 1. Esse movimento põe as cinco pedras pretas em atari. Se Preto conectar em \textbf{A}, todas as pedras estarão em atari. Após Branco 1, o grupo preto está morto, e a posição já está sedimentada.

\emph{Dia. 20.} Se for turno preto, ele pode viver jogando no ponto-chave de 1. Preto está vivo em seki. A posição fica como está. No final da partida, as três pedras brancas permanecerão no tabuleiro, e tanto Preto quanto Branco receberão zero pontos de território nesta região.

\emph{Dia. 21.} Se Branco resistir respondendo Preto 1, no \emph{Dia. 20}, com 2, Preto captura as quatro pedras brancas com 3.

\emph{Dia. 22.} Esse é o resultado da captura das pedras brancas no \emph{Dia. 21}. Preto está vivo com quatro pontos de território mais quatro pedras capturadas. Se Branco \textbf{A}, Preto ganha dois olhos jogando em \textbf{B}. Se Branco \textbf{B}, Preto \textbf{A}. Resistir, como no \emph{Dia. 21}, é uma perda de oito pontos para Branco.

\emph{Dia. 23.} O grupo preto possui um olho de cinco espaços. Ele está vivo ou morto, dependendo de quem for o turno.

\emph{dia. 24.} Depois de Branco 1, não mais nenhuma maneira com que Preto possa fazer dois olhos. Se necessário realmente capturar o grupo, Branco pode colocá-lo sob atari ocupando as intersecções marcadas com \textbf{X}. Se Preto, então, capturar com \textbf{A}, Preto acaba com um olho de quatro espaços, que é a mesma posição do \emph{Dia. 15}, onde mostramos que o grupo Preto está morto já.

\emph{Dia. 25.} Se for turno preto, ele poderá fazer dois olhos jogando no ponto-chave de 1.

O tópico de espaço de olho é minuciosamente contemplado no livro \emph{The Basics of Life and Death}~\cite{zeijst_bozulich_basics_of_life_and_death}, publicado pela Kiseido.
  \chapter{Estratégia de Go com Compensação}\label{chap:estrat_comp}

Depois de ter estudado, nos capítulos anteriores deste livro, e jogado algumas partidas em tabuleiros menores, você já adquiriu um conhecimento básico das regras e táticas básicas, então já está bem encaminhado para se tornar um jogador de Go. O próximo tópico que você precisará aprender é o de princípios básicos de estratégia do jogo. A forma mais segura de aprender tais princípios é estudando Go com pedras de compensação. Isso significa procurar por oponentes mais fortes e, com prazer, tomar grandes compensações contra eles. Jogar Go com compensação vai lhe ensinar como utilizar suas pedras de vantagem para ganhar influência no centro. Apesar de que o objetivo do Go é assegurar mais território do que o adversário, é influência que, lenta mas certamente, se torna (mais) território.

A maior compensação que existe tradicionalmente é a de nove pedras. É na verdade uma compensação enorme, equivalente a uma rainha no xadrez, ou talvez mais. Ainda assim, como iniciante, levará um bom tempo antes que você possa derrotar até mesmo um 1 dan razoavelmente forte, com essa compensação.

Encontrar oponentes hoje em dia é fácil. Há um excelente servidor chamado KGS (\href{https://www.gokgs.com}{\path{gokgs.com}})~\cite{kgs} onde você pode jogar online gratuitamente e, ainda, encontrar adversários desde a sua própria força até a de profissionais.

Como introdução, nós restringiremos esta discussão à compensação de nove pedras e somente estudaremos alguns padrões básicos que surgem com frequência neste nível de compensação.

\pagebreak

\begin{wrapfigure}{r}{60mm}
    \vspace{-15pt}
    \begin{center}
        \includegraphics[width=.5\textwidth]{9 - Dia 1}
        \captionsetup{justification=centering}
        \caption*{\emph{Dia.\@~1}}
    \end{center}
    \vspace{-20pt}
\end{wrapfigure}

Branco geralmente começa se aproximando de uma das pedras de compensação no canto com 1 no \emph{Dia.\@~1}. A resposta preta mais forte é contatar com o movimento diagonal de 2. Branco responde, na maioria das vezes, estendendo em 3, e Preto pode delimitar seu território no topo com 4.

A razão pela qual a troca de Preto 2 por Branco 3 é boa para Preto é que a pedra marcada está atacando as duas pedras brancas. Em outras palavras, ela está atuando como uma pinça em conjuntura com as três pedras pretas no topo.

Idealmente, Branco gostaria de estender tão longe quanto \textbf{A}, mas a pedra marcada está no caminho. (Vide os princípios de extensão discutidos na seção \autoref{sec:6.6:extensions}.)

\begin{wrapfigure}{l}{60mm}
    \vspace{-25pt}
    \begin{center}
        \includegraphics[width=.5\textwidth]{9 - Dia 2}
        \captionsetup{justification=centering}
        \caption*{\emph{Dia.\@~2}}
    \end{center}
    \vspace{-30pt}
\end{wrapfigure}

Para compreender a razão que faz com que Preto primeiro contate em 2, suponhamos que Preto omita tal jogada e simplesmente estenda para 2 no \emph{Dia.\@~2}. Branco poderia, então, deslizar a 3, Preto defende com a diagonal de 4, e Branco estende para 5. Comparado à posição em \emph{Dia.\@~1}, as pedras brancas terminam por ficar muito mais resilientes. Elas estão tomando território e estão muito mais avançadas no critério de fazer dois olhos e viver.

\pagebreak

\section{A Estratégia Branca de Chapelação}

\begin{wrapfigure}{r}{60mm}
    \vspace{-27.5pt}
    \begin{center}
        \includegraphics[width=.5\textwidth]{9 - Dia 3}
        \captionsetup{justification=centering}
        \caption*{\emph{Dia.\@~3}}
    \end{center}
    \vspace{-30pt}
\end{wrapfigure}

Após Preto 4 no \emph{Dia.\@~1}, Branco talvez faça a aproximação de dois espaços contra a pedra preta no canto inferior direito com 5 no \emph{Dia.\@~3}. Preto 6, delimitando o território no lado inferior, é um bom movimento e a resposta padrão. Branco poderá talvez, então, chapelar a pedra preta no lado direito com 7, ameaçando ostensivamente fazer território ele mesmo. Na realidade, ele está na esperança de atiçar Preto para uma briga, em que a habilidade de luta branca, sendo este o jogador mais forte, será muito superior à preta.

Como Preto deveria responder?

A melhor estratégia é resolutamente escapar ao centro com a pedra marcada, jogando a diagonal de 8.

\begin{wrapfigure}{l}{60mm}
    \vspace{-22.5pt}
    \begin{center}
        \includegraphics[width=.5\textwidth]{9 - Dia 4}
        \captionsetup{justification=centering}
        \caption*{\emph{Dia.\@~4}}
    \end{center}
    \vspace{-20pt}
\end{wrapfigure}

Branco tenta manter Preto confinado ao lado direito com 9 a 11 no \emph{Dia.\@~4}, mas, depois de 12, Branco precisa cortar em 13 para eliminar a conexão preta ali. Preto então faz atari com 14 a 16, e suas pedras agora escaparam. Branco faz atari com 17, e assegura sua posição com 19 --- Branco talvez também jogue em \textbf{A}. Após 20, Preto já quase assegurou a maior parte do território na parte baixa do lado direito. Exceto pela captura de 12, Branco não possui nenhum território.

\pagebreak

\begin{wrapfigure}{r}{60mm}
    \vspace{-17.5pt}
    \begin{center}
        \includegraphics[width=.5\textwidth]{9 - Dia 5}
        \captionsetup{justification=centering}
        \caption*{\emph{Dia.\@~5}}
    \end{center}
    \vspace{-37.5pt}
\end{wrapfigure}

Outra estratégia que Preto pode empregar contra a estratégia branca de chapelação é a de ceder território no lado direito. Por exemplo, Preto pode começar por contra-chapelar em 8 no \emph{Dia.\@~5}, ameaçando se conectar com a pedra marcada. Branco defende em 9. Preto agora pula a 10, induzindo Branco a defender em 11. Preto agora expande seu território no canto com 12, ameaçando reduzir o território e expandir o seu ainda mais, através do deslize em \textbf{A}.

Branco provavelmente defenderá com 13 no \emph{Dia.\@~6}. Preto continuará ameaçando reduzir o território branco com 14 a 16, enquanto expande seus cantos direitos superior e inferior. A continuação até 25 ainda também termina em sente.

\begin{wrapfigure}{l}{60mm}
    \vspace{-22.5pt}
    \begin{center}
        \includegraphics[width=.5\textwidth]{9 - Dia 6}
        \captionsetup{justification=centering}
        \caption*{\emph{Dia.\@~6}}
    \end{center}
    \vspace{-30pt}
\end{wrapfigure}

Ao avaliar essa posição, podemos perceber que Branco assegurou aproximadamente vinte e cinco pontos em um território já fortemente delimitado. Entretanto, os cantos Pretos são estimados em mais de quarenta pontos. Adicionalmente, Preto terminou em sente, então ele ainda pôde reforçar sua posição no canto superior esquerdo com 26. Esse movimento tornará uma invasão branca nessa região exponencialmente mais difícil.

\pagebreak

\section{Evitando a Estratégia Branca de Chapelação}

\begin{wrapfigure}{r}{60mm}
    \vspace{-27.5pt}
    \begin{center}
        \includegraphics[width=.5\textwidth]{9 - Dia 7}
        \captionsetup{justification=centering}
        \caption*{\emph{Dia.\@~7}}
    \end{center}
    \vspace{-37.5pt}
\end{wrapfigure}

Ao invés de responder a Branco 5 com 6 no \emph{Dia.\@~3}, uma jogada mais positiva é o pulo até 6 no \emph{Dia.\@~7}. Há duas razões para que esta seja uma boa jogada.

Primeiramente, Preto 6 cria uma barreira entre a pedra em 5 e as duas pedras brancas marcadas acima, assim mantendo os grupos separados. Branco agora está fragmentado em dois grupos sob ataque.

Em segundo lugar, Preto 6 se certifica de que a pedra marcada poderá se conectar à sua pedra no centro.

É esperado que Branco faça uma dupla-aproximação contra a pedra de compensação no canto inferior direito. Preto deveria então responder com o movimento diagonal em 8. Esse movimento segue o mesmo princípio de Preto 6: ele mantém as pedras brancas de 5 a 7 separadas e oferece acesso ao centro para a pedra 4-4, fazendo com que seja mais fácil conectá-la a seus aliados no exterior.

Há várias maneiras de Branco responder a Preto 8. Ele talvez ignore tal movimento e jogue em um lugar totalmente diferente; ele talvez, também, construa uma posição no lado inferior; ou ele talvez invada o canto como no \emph{Dia.\@~8} da próxima página.

\pagebreak

\section{Construindo Armações em Formato de Caixa}

\begin{wrapfigure}{r}{60mm}
    \vspace{-27.5pt}
    \begin{center}
        \includegraphics[width=.5\textwidth]{9 - Dia 8}
        \captionsetup{justification=centering}
        \caption*{\emph{Dia.\@~8}}
    \end{center}
    \vspace{-32.5pt}
\end{wrapfigure}

Branco geralmente responde a Preto 8 invadindo o canto com 9 no \emph{Dia.\@~8}, mas isso não terá um bom resultado para ele. Preto bloqueia com 10, e Branco cria um grupo vivo no canto com os movimentos até 19. Porém, com 20, Preto consegue uma parede densa virada para o centro, constituindo uma armação de território em formato de caixa, juntamente com as pedras marcadas. Além disso, as duas pedras brancas marcadas são basicamente inúteis, à sombra da parede preta.

\begin{wrapfigure}{l}{60mm}
    \vspace{-27.5pt}
    \begin{center}
        \includegraphics[width=.5\textwidth]{9 - Dia 9}
        \captionsetup{justification=centering}
        \caption*{\emph{Dia.\@~9}}
    \end{center}
    \vspace{-80pt}
\end{wrapfigure}

\bigskip

Branco talvez tente deslizar  para 9 no \emph{Dia.\@~9}. Novamente, Preto almeja o centro ao jogar 10. Branco cria um grupo vivo com 11 e 13, mas Preto mapeia uma grande armação de território em formato de caixa com 12 e 14. Com jogo claro e limpo, isso deveria gerar mais de 40 pontos garantidos de território para Preto.

\pagebreak

Branco poderia estender para 9, no \emph{Dia.\@~10}, esperando parar o avanço preto rumo à grande armação. Preto ancorará suas pedras no canto com 10. Se Branco pular para 11, Preto mantém Branco separado em dois. Branco terá muitas dificuldades vivendo com ambos os grupos.

\begin{figure}[h!]
    \centering
    \begin{subfigure}[t]{.5\textwidth}
        \centering
        \includegraphics[width=1\textwidth]{9 - Dia 10}
        \captionsetup{justification=centering}
        \caption*{\emph{Dia.\@~10}}
    \end{subfigure}
\end{figure}

\pagebreak

\section{Expanda o Seu Território Atacando}

\begin{wrapfigure}{r}{60mm}
    \vspace{-27.5pt}
    \begin{center}
        \includegraphics[width=.5\textwidth]{9 - Dia 11}
        \captionsetup{justification=centering}
        \caption*{\emph{Dia.\@~11}}
    \end{center}
    \vspace{-25pt}
\end{wrapfigure}

Depois da troca de 1 por Preto 4, Branco talvez tente criar uma base para suas pedras estendendo até 5 (ou \textbf{A}) em \emph{Dia.\@~11}. A resposta preta mais forte é fazer o ``pilar de ferro'' com 6. Essa jogada previne a base branca no contato em 6 ou o deslize em \textbf{B}. Esse pilar de ferro também reforça a formação de caixa delineada pelas pedras marcadas.

Branco não possui espaço para expansão no topo, então ele se expande ao centro. Preto mantém a pressão nas pedras brancas pulando a 8 e continuando a expandir e fortalecer sua armação no lado esquerdo.

\begin{wrapfigure}{l}{60mm}
    \vspace{-25pt}
    \begin{center}
        \includegraphics[width=.5\textwidth]{9 - Dia 12}
        \captionsetup{justification=centering}
        \caption*{\emph{Dia.\@~12}}
    \end{center}
    \vspace{-22.5pt}
\end{wrapfigure}

\bigskip

Já que Preto reforçou sua formação de caixa no canto superior esquerdo com as pedras marcadas, Branco estará em desvantagem se ali invadir. Portanto, ele talvez opte por jogar em uma parte do tabuleiro onde Preto possui menos pedras, esperando que qualquer posição que ele construa ali negue a influência preta da armação no canto superior esquerdo. No entanto, se ele tentar invadir essa área com 9 no \emph{Dia.\@~12}, Preto simplesmente enclausurará o canto com 10. Pular para Branco 11 é uma resposta razoável, mas Preto delimita o território no canto superior esquerdo com 12. Jogando-se corretamente, Preto pode contar com pelo menos 25 pontos nessa região se tornando território seguro. No meio-tempo, Branco assegurou quase nenhum território e ele ainda possui duas pedras no lado esquerdo que precisam se estabilizar.

\begin{wrapfigure}{r}{60mm}
    \vspace{-27.5pt}
    \begin{center}
        \includegraphics[width=.5\textwidth]{9 - Dia 13}
        \captionsetup{justification=centering}
        \caption*{\emph{Dia.\@~13}}
    \end{center}
    \vspace{-10pt}
\end{wrapfigure}

Se Branco se aproximar do outro lado com 9 no \emph{Dia.\@~13}, Preto irá novamente enclausurar o canto com 10, dessa maneira assegurando o território no lado esquerdo com 12, após o pulo branco até 11. Este resultado é ainda pior para Branco do que \emph{Dia.\@~12}, visto que suas pedras de 9 a 11 estão espremidas  entre duas posições fortes pretas. Será uma tarefa árdua se estabilizar com tal grupo branco.

\begin{wrapfigure}{l}{60mm}
    \vspace{-25pt}
    \begin{center}
        \includegraphics[width=.5\textwidth]{9 - Dia 14}
        \captionsetup{justification=centering}
        \caption*{\emph{Dia.\@~14}}
    \end{center}
    \vspace{-25pt}
\end{wrapfigure}

Por fim, Branco talvez invada diretamente o canto com 9 no \emph{Dia.\@~14}. Ele certamente pode criar um grupo vivo com quase dez pontos de território, mas à concessão de quase 30 pontos de potencial território, que muito provavelmente será concretizado, com jogadas corretas.

\bigskip

Este capítulo é somente uma breve introdução à estratégia de compensação, em particular, a de nove pedras. Para uma discussão mais detalhada dessas posições, referenciam-se ao leitor \emph{Handicap Go Strategy and the Sanrensei Opening}~\cite{zeijst_bozulich_handicap_sanrensei}, disponível em forma impressa pela Kiseido (veja o \autoref{ap:pt} de catálogo de livros); ou, em sistemas iOS, o aplicativo SmartGo~\cite{smartgo}. Esse livro também é uma boa introdução a Go com compensação, assim como à abertura sanrensei, que é uma abertura de partidas em igualdade. Para um livro ainda mais detalhado sobre esse tópico, veja \emph{Handicap Go}~\cite{nagahara_bozulich_handicap_go}, também publicado pela Kiseido e disponível no aplicativo \href{apps.apple.com/us/app/smartgo-player/id314506629}{SmartGo}.
  \chapter{Níveis e Compensações}\label{chap:10:niveis_compensacoes}

O sistema de nivelamento amador utilizado no Go funciona da seguinte maneira. Jogadores iniciantes recebem automaticamente o nível de 35 kyu e, conforme aprendem técnicas básicas do jogo, eles rapidamente avançam a escada kyu até chegar a algo entre, aproximadamente 20 a 10 kyu. Essa melhora rápida presume, claro, que se estude e se jogue algumas vezes na semana por alguns meses. De 10 a 1 kyu, o progresso é usualmente muito mais devagar. Assim que um jogador amador se torna um expert, ele recebe o nível de 1 dan (\emph{shodan} em japonês), ou de uma organização ou por simplesmente medir sua força contra outros jogadores de força dan. Níveis amadores dan vão até 7 dan em geral. Níveis profissionais vão de 1 a 9 dan, mas eles estão em uma escala diferente, portanto não possuem nenhum nível correspondente em níveis amadores. Um profissional 1 dan deveria estar apto a dar perto de três pedras de compensação a um amador de nível 5 dan, e ainda ter pelo menos 50\% de chance de vitória. O sistema de compensação no Go oferece ao jogador mais fraco uma chance realística de vitória quando se joga com alguém mais forte, dado que a compensação esteja correta, claro.

Jogadores de força igual geralmente utilizam um método chamado de \emph{nigiri} para determinar quem tomará branco ou preto. Quando o nigiri é utilizado, um dos jogadores toma uma quantidade arbitrária de pedras brancas enquanto que o jogador preto escolhe uma ou duas pedras, essencialmente uma variante do jogo de par-ou-ímpar. Então ambas as respectivas pedras escolhidas são postas em cima do tabuleiro. Se o número de pedras brancas corresponder à escolha de par ou ímpar preta, então cada jogador procede com suas cores como estão, senão os jogadores trocam suas cores. Dali em diante, o jogo procede normalmente. Porém, caso um dos jogadores comece a ganhar consistentemente, a compensação deve mudar.

O jogador com as pedras pretas sempre possui o primeiro movimento, então ele possui uma vantagem crítica. Essa vantagem é estimada como de valia de 6 pontos. Portanto, quando de jogadores de força similar, o jogador com pedras pretas dá ao de pedras brancas seis prisioneiros no início da partida, como adiantamento. Adicionalmente, se o resultado da partida for empate, Branco vence. Esse total de prisioneiros é referenciado como \emph{komi}, e dizemos algo como ``Branco recebe um komi de 6.5 pontos''.

\begin{wrapfigure}{r}{60mm}
    \vspace{-30pt}
    \begin{center}
        \includegraphics[width=.5\textwidth]{10 - Dia 15}
        \captionsetup{justification=centering}
        \caption*{\emph{Dia.\@~15}}
    \end{center}
    \vspace{-25pt}
\end{wrapfigure}

Quando jogadores de diferentes níveis jogam uma partida, o jogador mais forte sempre toma as pedras brancas, e a compensação é determinada pela diferença de nível entre eles. Por exemplo, se um jogador for 1 kyu, e o outro, 2 kyu, então o jogador 2 kyu não recebe nenhuma pedra de compensação. Em seu lugar, ele joga como preto sem nenhum komi, fazendo o primeiro movimento.

Quando um 1 kyu joga contra um 3 kyu, o 3 kyu recebe uma compensação de duas pedras; quando um 1 kyu joga contra um 5 kyu, o 5 kyu recebe uma compensação de quatro pedras, etc., até a compensação de nove pedras, que é em geral a maior dada.

\begin{wrapfigure}{l}{60mm}
    \vspace{-25pt}
    \begin{center}
        \includegraphics[width=.5\textwidth]{10 - Dia 16}
        \captionsetup{justification=centering}
        \caption*{\emph{Dia.\@~16}}
    \end{center}
    \vspace{-30pt}
\end{wrapfigure}

O mesmo sistema de compensação é utilizado para jogadores de nível dan também. A princípio, é possível determinar sua força no Go jogando contra jogadores de níveis já bem estabelecidos. No entanto, até um amador se tornar aproximadamente 5 dan, sua força pode oscilar fortemente. Por fim, apesar de que o sistema de nível do Go é virtualmente o mesmo ao redor do mundo, o valor dado a tais níveis pode variar de lugar para lugar.

\pagebreak

\begin{wrapfigure}{r}{60mm}
    \vspace{-20pt}
    \begin{center}
        \includegraphics[width=.5\textwidth]{10 - Dia 17}
        \captionsetup{justification=centering}
        \caption*{\emph{Dia.\@~17}}
    \end{center}
    \vspace{-31pt}
\end{wrapfigure}

Assim que a compensação para o jogador mais fraco for estabelecida, ele coloca pedras de compensação no tabuleiro na ordem exibida nos \emph{Dias. 15 a 17}. Branco então faz o primeiro movimento real da partida, e ambos os jogadores se alternam. Por exemplo, se a compensação for de três pedras, Preto posicionaria as pedras 1, 2 e 3 (a ordem tradicional) no \emph{Dia.\@~15}, e Branco faria o primeiro movimento real da partida.

Seguindo estritamente as regras (\emph{Regra 4} em particular, no \autoref{chap:regras}), o que realmente está acontecendo é que, após Preto posicionar 1, Branco passa. Preto então coloca a pedra 2 (na realidade, o terceiro movimento), e, novamente, Branco passa. Preto joga 3 (movimento 4), mas Branco não passa e faz seu primeiro movimento, iniciando realmente a partida.

De acordo com a \emph{Regra 3}, um jogador pode jogar em qualquer ponto que quiser, então ele poderia jogar pedras de compensação em qualquer intersecção. É possível jogar assim\footnote{Compensação livre, em que o jogador mais fraco determina totalmente a distribuição da vantagem, é tipicamente conhecida como compensação chinesa.}, desde que negociado entre os jogadores. Porém, de um ponto de vista pedagógico, nós recomendamos seguir a maneira tradicional de se jogar Go, pois ela vai lhe ensinar como jogar com pedras no 4-4 e criar posições fortes e densas.
  \chapter{Movimentos de Fim de Jogo}

A posição no \emph{Dia.\@~1} abaixo é a mesma que o \emph{Dia.\@~6} que no \autoref{chap:estrat_comp} após Branco ter jogado 13 (a pedra marcada). Quando Preto desce para 1 (14 no \emph{Dia.\@~6} do \autoref{chap:estrat_comp}), Branco não deveria jogar em outro lugar, com 2 por exemplo, já que Preto invadirá com 3, e Branco não conseguirá defender seu território.

O melhor que ele pode fazer é empurrar com 4 no \emph{Dia.\@~2} e pular para 6. Porém, Preto pode reduzir Branco rastejando com 7 a 9. Após os movimentos até 13, o território branco basicamente desapareceu. Branco investiu movimentos demais nesta área para deixar isso acontecer.

Sendo assim, quando Preto desce para 1 no \emph{Dia.\@~3}, Branco não possui nenhuma outra escolha senão defender com 2. Quando Preto desce para 3, ele precisa bloquear com 4 também.

Movimentos como Preto 1 e 3 são geralmente jogados no final do meio de jogo quando o território de cada lado já essencialmente determinado. Eles são utilizados para expandir seu território ao mesmo tempo que se diminiu o do adversário. Em seguida\ldots

Preto joga 5 e 7 no \emph{Dia.\@~4}, forçando Preto a defender com 6 e 8. Ele continua com o mesmo tipo de movimento na parte inferior do lado direito com 9 e 11, e Branco precisa defender com 12, terminando em \emph{gote}. Em outras palavras, Branco precisa fazer o último movimento defensivo.

Por que Branco 8 e 12 seria necessário?

Depois de Branco 12 no \emph{Dia.\@~4}, Preto possui sente. Isto é, ele possui a iniciativa --- ele não precisa defender e pode jogar onde quiser. Nesta posição, ele talvez queira jogar em \textbf{A} ou \textbf{B} e conseguiu uma boa vantagem nesta parte do tabuleiro.

Se Branco omitir 8 no \emph{Dia.\@~4} e se aproximar da pedra preta no canto inferior esquerdo com 1 no \emph{Dia.\@~5}, Preto cortará com 2. Se Branco tentar escapar com 3, Preto faz atari de novo com 4 e, então, enreda com 6. As pedras pretas não conseguem fugir. Se Branco \textbf{A}, Preto faz atari em \textbf{B}; se Branco \textbf{C}, Preto \textbf{D}.

\section{Cálculos de Fim de Jogo}

Restrinjamos nossa atenção ao lado direito do tabuleiro, assumindo que os outros territórios no tabuleiro já foram decididos. Qual é o valor dos movimentos Preto 5 e 7? E qual é o valor dos movimentos Preto 9 e 11?

Supondo que Branco jogue como aqui, no \emph{Dia.\@~7} antes de Preto. Se Branco fizer 5, Preto precisa responder com 6. Branco conecta com 7 e Preto precisa defender com 8. Comparado a \emph{Dia.\@~6}, Branco ganhou dois pontos marcados com \textbf{X}, e Preto perdeu dois pontos em 6 e 8. Isso é um ganho de quatro pontos para Branco.

Branco também termina com sente, então ele pode ainda jogar 9 e 11, forçando Preto a responder com 10 e 12. Comparado ao \emph{Dia.\@~6}, Branco ganha outros dois pontos, marcados com \(\increment\), e Preto perde dois pontos 10 e 12. Novamente, Branco ganha quatro pontos.

Portanto, qual seja o lado que jogue seus movimentos primeiro, o ganho será de oito pontos.

No entanto, a posição no \emph{Dia.\@~1} sugiu na abertura, então Preto talvez não desça para 1. Em seu lugar, ele talvez reforce sua posição no canto superior esquerdo com 1 no \emph{Dia.\@~8}. Branco talvez pense que descer para 2 é um grande movimento, já que ameaça a invasão do canto Preto, mas Preto ignoraria  tal ameaça e asseguraria o canto superior esquerdo com 3 ou \textbf{A}.

Se Branco der continuidade à ameaça de invasão ao canto superior direito preto com 4 e 6, Preto assegurará o canto superior esquerdo com 5 e 7. A invasão branca no canto superior direito é de aproximadamente 30 pontos, mas Preto 1 a 7 delimitam um território enorme à esquerda. É difícil estimar quão grande será esse território. Branco provavelmente conseguirá invadi-lo e criar um grupo vivo, mas Preto será capaz de garantir mais de 60 pontos no processo. Sendo assim, Preto fica feliz com a troca.

O que estamos querendo enfatizer é que movimentos como Preto 1 no \emph{Dia.\@~1} e Branco 2 no \emph{Dia.\@~8} são geralmente deixados para o fim de jogo. Na abertura, há sempre movimentos maiores a serem feitos. É claro que Preto 1 no \emph{Dia.\@~1} é uma grande ameaça, então, do ponto de vista do Branco, ele não pode ignorá-lo. Mas, do ponto de vista do Preto, ele pode facilmente se dar ao luxo de ignorar Branco 2 no \emph{Dia.\@~8}, já que há pontos maiores em disputa do que simplesmente defender o canto.

O fim de jogo é um aspecto importante do jogo a se dominar. A margem de vitória ou derrota é frequentemente muito pequena, e manobras hábeis de fim de jogo podem virar a partida a seu favor.

A Kiseido já publicou dois excelentes livros sobre esse assunto. Para uma introdução detalhada ao assunto, você deveria ler o \emph{The Endgame}~\cite{tomoko_bozulich_endgame}, o Volume 6 da \emph{Elementary Go Series} da Kiseido. Outro livro é o \emph{Get Strong at the Endgame}~\cite{bozulich_endgame}, Volume 7 da \emph{Ge Strong at Go Series} da Kiseido. Esse livro provê cálculos de fim de jogo para 101 posições básicas. Há também 120 problemas onde você deverá encontrar o tesuji de fim de jogo --- um movimento brilhante que cumpre um papel tático claro --- que maximizará seu lucro. Finalmente, há 70 problemas em tabuleiros 9\(\times\)9 e 13\(\times\)13 que lhe possiblitarão um teste de habilidade no fim de jogo.
  \chapter{Como Continuar os Estudos}

Depois de ter lido este livro e ter jogado vários jogos, você terá adquirido conhecimento básico do jogo e estará bem encaminhado para se tornar um jogador de Go. Porém, para se tornar um jogador forte, você precisará de duas coisas: aprender princípios básicos de estratégia de abertura; e desenvolver sua habilidade analítica. Para a abertura, você deveria começar com o Volume 1 da \emph{Elementary Go Series} da Kiseido, \emph{In the Beginning: The Opening in the Game of Go}~\cite{ikure_in_the_beginning}. Após o estudo desse livro, você terá aprendido os princípios básicos da abertura.

Juntamente com o livro citado acima, você também deveria estudar todos os problemas da série de três volumes \emph{Graded Go Problems for Beginners}~\cite{yoshinori_bozulich_ggpb}. (Os 46 problemas na primeira parte deste livro foram tirados do Volume 1 dessa série.) Os primeiros problemas são extremamente fáceis, mas, gradualmente, eles se tornam mais difíceis. Após ter estudado os quatro volumes dessa série, se você for capaz de voltar e resolver todos os 1,387 problemas só de relance, sua habilidade analítica será de 1 dan.

Você também deveria continuar seu estudo da abertura aprendendo um básico de josekis. Durante a abertura, conflitos críticos frequentemente surgem, iniciando-se nos cantos e se desenvolvendo ao longo dos lados. Esses conflitos são denominados josekis. Nós apresentamos diversos josekis no \autoref{chap:6:estrat_abertura} sobre estratégia de abertura, mas há muitos outros a serem estudados. O segundo volume da \emph{Elementary Go Series}, \emph{38 Basic Josekis}~\cite{ishida_yoshio_basic_joseki_dictionary}, fornecerá a fundação sobre a qual você poderá expandir seu conhecimento de joseki. Você poderá, então, utilizar os dois volumes do \emph{The 21st Century Dictionary of Basic Josekis}~\cite{takao_shinji_21st_century_joseki_dictionary} como referência para incrementar seu repertório de josekis, conforme necessário.

Para o meio de jogo, a melhor introdução é o \emph{Attack and Defense}~\cite{ishida_akira_james_davies}, o Volume 5 da \emph{Elementary Go Series} da Kiseido. Tesujis --- movimentos hábeis que cumprem um objetivo tático claro, aos quais jogadores de xadrez talvez se refiram como ``brilhantismos'' --- também constituem um aspecto importante do meio de jogo. Uma boa introdução ao assunto é o terceiro volume da série \emph{Elementary Go Series}, \emph{Tesuji}~\cite{davies_tesuji}. Você deveria em seguida trabalhar sobre os \emph{501 Tesuji Problems}, no quarto volume da \emph{Mastering the Basics Series}~\cite{bozulich_501_tesuji}, também da Kiseido. Os problemas nesse livro cobrem todos os 45 tesujis básicos que talvez apareçam nos seus jogos. A maestria desses problemas aumentará grandemente a sua força no Go.

Para melhorar o seu poder de meio de jogo, assim como sua compreensão geral de estratégia no Go, é necessário conhecer os princípios fundamentais do jogo. Estes são cobertos no nono volume da \emph{Mastering the Basics Series} da Kiseido, \emph{An Encyclopedia of Go Principles}~\cite{bozulich_encyclopedia_principles}.

Se você quiser se tornar um jogador seriamente comprometido, estudo constante de vida e morte é primordial. O estudo de vida e morte vai lhe treinar a ``ler'' a posição três ou quatro movimentos à frente. Nós recomendamos que você comece com o \emph{The Basics of Life and Death}~\cite{zeijst_bozulich_basics_of_life_and_death}. Esse livro é uma introdução minuciosa ao tópico. Na Parte \RomanNumeralCaps{1}, o conceito de vida e morte --- espaço de olho --- é explicado. Em seguida, os três tesujis mais importantes de vida e morte são introduzidos, isto é, os tesujis de: posicionamento, hane e sacrifício. Ao longo do caminho, outro conceito importante é apresentado: escassez de liberdades. Esse conceito é importante não somente em vida e morte, mas, também, em lutas de meio de jogo. Finalmente, a regra do quatro-curvado-no-canto é analisada. Intercalados entre as seções, temos 51 problemas que reforçarão esses conceitos e lhe ajudarão a internalizá-los. A Parte \RomanNumeralCaps{2} consiste de 177 posições de vida e morte básicas que surgem em josekis e conflitos de canto e de lado. Essas posições não são uma coleção aleatória de problemas de vida e morte. Cada uma é um padrão básico que pode facilmente surgir nos seus jogos. Internalizar essas posições e saber como gerenciá-las vai aumentar dramaticamente a sua força.

A Kiseido e o SmartGo~\cite{smartgo} publicam um número grande de livros de problemas de vida e morte, então há material suficiente para manter até mesmo o mais ávido solucionador de problemas entretido por muitos anos.

Enquanto você estuda todos esses livros e adquire o conhecimento teórico que eles provêm, você deveria jogar quantas partidas puder, para ganhar experiência prática no tabuleiro. Hoje em dia, um jogador pode convenientemente jogar Go na internet através do KGS (\href{https://www.gokgs.com}{gokgs.com})~\cite{kgs} gratuitamente. Você conseguirá encontrar jogadores desde iniciantes até profissionais. Há também um sistema de nível por lá, então você conseguirá traçar o seu progresso conforme escala a montanha kyu até chegar aos níveis dans e além.

Por fim, reproduzir partidas de profissionais é outra boa maneira de se melhorar. Isso vai lhe oferecer uma intuição para o fluxo da partida, além de bons exemplos de jogadas. O melhor livro para esse propósito é \emph{Invincible: The Games of Shusaku}~\cite{power_invincible}. Todos os aprendizes de profissionais são demandados a estudar as partidas de Shusaku.

  \appendix
  \addtocontents{toc}{\newpage\noindent\textbf{Apêndice}\par}
  \input{Conteudo/Apendice_Catalogo_EN}
  \chapter{Catálogo Kiseido em Português}\label{ap:pt}

\emph{Note, porém, que, basicamente, nenhum dos livros abaixo foram traduzidos para o português. O catálogo abaixo é apenas uma tradução somente dos títulos em inglês.}

\bigskip
  \chapter{Jogue Go Online com o KGS}

Jogue Go online gratuitamente através do Kiseido Go Server (KGS). Ele pode ser acessado através do site da Kiseido (\href{https://www.kiseido.com}{\path{kiseido.com}}~\cite{kiseido}), ou você pode ir diretamente ao \href{https://www.gokgs.com}{\path{gokgs.com}}~\cite{kgs}. Não há custo para se jogar partidas neste servidor, e ele funciona com Linux/UNIX, Mac e Windows. A força dos jogadores vai desde iniciantes até profissionais 9 dan.

O KGS possui as seguintes funcionalidades:

\begin{enumerate}
    \item Edição de partidas online.
    \item O KGS é multilingual. Todas as mensagens que os jogadores digitam e enviam são feitas em Unicode, o que possibilita conversas em qualquer linguagem.
    \item Suporte primariamente de regras japonesas, mas também há como se utilizar regras chinesas, AGA (americanas) e neozelandesas.
    \item É possível jogar sem tempo limite, com um limite de tempo absoluto ou com vários sistemas de byo-yomi.
    \item Habilidade de salvar e exportar seus jogos gratuitamente, possibilitando revisões e estudos futuros.
    \item Sistema de nivelamento que possibilita, dentre outros, histórico de progresso. 
\end{enumerate}

Entre no KGS por \href{https://www.gokgs.com}{\path{gokgs.com}}~\cite{kgs}.

  \backmatter
  \bibliographystyle{unsrt}
  \bibliography{bibliografia}
\end{document}
%-------------------------------------------------------------------------------