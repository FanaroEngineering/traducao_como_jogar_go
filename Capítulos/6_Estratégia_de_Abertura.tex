\chapter{Estratégia de Abertura}

Para se tornar um jogador forte de Go, é necessário desenvolver duas habilidades:

\begin{enumerate}
    \item a habilidade de ler à frente movimento por movimento e prever os resultados de embates locais;
    \item a habilidade de intuir o que está acontecendo no tabuleiro como um todo.
\end{enumerate}

O balanço aproximadamente igualitário entre qualidades intuitivas e analíticas é grande parte da atratividade do Go. Na abertura, quando o tabuleiro consiste de majoritariamente espaço vazio, é intuição, e uma base de dados de conhecimento geral, que toma um papel dominante.

No tabuleiro 19x19, o tamanho oficial, é difícil de se assegurar território no início, então a partida usualmente começa com os jogadores espaçando suas pedras para formar grandes armações dentro das quais eles poderão brigar vantajosamente no futuro.

\emph{Dia. 1 a 3} mostra uma abertura típica no tabuleiro 19x19.

É geralmente muito mais fácil de se estabelecer bases nos cantos, como Preto e Branco o fazem com 1 a 4 no \emph{Dia. 1}. Uma ou duas pedras por canto é suficiente. Com 5, Preto estabelece um enclausuro de canto. Esse movimento delimita e vigia o território no canto. Não é ainda território seguro, uma vez que Branco possui múltiplas maneiras de invadi-lo. Mas Preto terá a vantagem em qualquer luta que se irromper ali. O tempo para a invasão branca será, assim, um fator crítico.

Aproximar-se do canto com Branco 6 no \emph{Dia. 2}, onde Preto possui somente uma pedra, é uma boa jogada. Isso frequentemente provoca lutas, como o curto conflito que se segue. Nos movimentos de 7 a 12, Preto assegura  o canto enquanto Branco constrói uma posição à direita. Essa sequência é um dos padrões-referência conhecidos como josekis.

Preto 13 no \emph{Dia. 13} é outro exemplo de outra aproximação. Branco 14 forma uma armação esparsa no canto inferior esquerdo do tabuleiro a partir da pedra branca marcada. A terceira ou a quarta linha é a melhor para extensões como esta. Preto desenvolve uma armação no topo com 15 e 17. Note Preto 17 na quarta linha, e Preto 15 e a pedra preta marcada na terceira linha. Isso constitui um balanço ideal de jogadas altas e baixas. As pedras na terceira linha defendem os flancos da posição preta enquanto que a pedra na quarta linha expande seu território para o centro.

\section{A Primeira Prioridade: Estabelecer uma Presença no Canto}

A abertura de uma partida de Go geralmente se inicia com ambos os lados estabelecendo presenças nos cantos. Os pontos a seguir são os cinco de referência que um jogador usualmente ocupa com seus primeiros movimentos.

\emph{Dia. 1. O ponto 3-3.} O intuito de Preto 1 no ponto 3-3 --- também conhecido como \emph{san-san} --- é assegurar o território no canto, apesar de que esse movimento não fornece muita influência no centro.

\emph{Dia. 2. O ponto-estrela.} Por ter jogado no ponto 4-4 --- também conhecido como ponto-estrela --- com 1, Preto almeja influência no centro.

\emph{Dia. 3. O ponto 3-4.} Quando Preto joga 1 no ponto 3-4 (\emph{komoku}), ele espera ganhar território ao longo do lado direito assim como algo no canto.

\emph{Dia. 4. O ponto 5-3.} Preto 1 no ponto 5-3 (\emph{mokuhazushi}) enfatiza o lado. Preto está disposto a conceder a maior parte do canto para Branco.

\emph{Dia. 5. O ponto 5-4.} Preto 1 no ponto 5-4 (\emph{takamoku}) concede o canto ao Branco. Ele almeja influência no centro e ao longo dos lados.

\section{Movimentos de Aproximação}

