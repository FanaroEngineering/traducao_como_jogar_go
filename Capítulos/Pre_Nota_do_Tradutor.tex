\chapter{Nota do Tradutor}

% TODO: Citar mídias sociais
% TODO: Citar livro do Felipe com a secretaria municipal

Primeiramente, agradeço muito ao Richard Bozulich e ao James Davies por, não somente terem escrito este livro, mas, também, terem liberados os direitos de tradução. Espero que seja o primeiro de muitos livros traduzidos para o português da excelente editora Kiseido~\cite{kiseido}.

\bigskip

A tradução deste livro foi feita com base em meu conhecimento do jogo. Sou um jogador amador de força aproximada mínima de 1 dan, um ranking tido como de maestria do jogo, apesar de que, claro, sempre há um peixe maior. Desde que comecei a jogar no final de 2012, já participei de torneios tanto no Brasil quanto na Europa, na Coreia e na China, além de alguns torneios online também. Toda essa experiência me ajuda a discernir o que é bom do que é ruim, o que é útil do que é perda de tempo.

É por isso que corri atrás de conteúdo da editora Kiseido, que vem trazendo ao Ocidente, em inglês, conhecimento de Go já há décadas. Este livro é uma introdução sólida e sistemática ao jogo, com muita concisão. Nele, o leitor encontrará, além das regras, conceitos e técnicas utilizadas por desde iniciantes até grandes mestres do Go, táticas e estratégias que residem no coração jogo.

Porém, não tenha pressa. Go demanda muita prática, e esta é muito divertida e recompensadora. Internalizar os conceitos deste livro é algo que não só demanda tempo, mas que também é feito em ciclos, mesmo os grandes mestres aprendem com os fundamentos todos os dias.

Após o término deste livro, infelizmente, não há muito mais conteúdo em português, pelo menos não em via impressa. Tentarei continuar criando mais traduções de livro de Go, mas não sei a que velocidade ou até que ponto. No entanto, já há conteúdo online para variados níveis, em português, espalhado pela internet.

\bigskip
\bigskip

Philippe Fanaro

1 de Outubro de 2021